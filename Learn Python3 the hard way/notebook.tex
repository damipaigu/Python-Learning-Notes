
% Default to the notebook output style

    


% Inherit from the specified cell style.




    
\documentclass[11pt]{article}

    
    
    \usepackage[T1]{fontenc}
    % Nicer default font (+ math font) than Computer Modern for most use cases
    \usepackage{mathpazo}

    % Basic figure setup, for now with no caption control since it's done
    % automatically by Pandoc (which extracts ![](path) syntax from Markdown).
    \usepackage{graphicx}
    % We will generate all images so they have a width \maxwidth. This means
    % that they will get their normal width if they fit onto the page, but
    % are scaled down if they would overflow the margins.
    \makeatletter
    \def\maxwidth{\ifdim\Gin@nat@width>\linewidth\linewidth
    \else\Gin@nat@width\fi}
    \makeatother
    \let\Oldincludegraphics\includegraphics
    % Set max figure width to be 80% of text width, for now hardcoded.
    \renewcommand{\includegraphics}[1]{\Oldincludegraphics[width=.8\maxwidth]{#1}}
    % Ensure that by default, figures have no caption (until we provide a
    % proper Figure object with a Caption API and a way to capture that
    % in the conversion process - todo).
    \usepackage{caption}
    \DeclareCaptionLabelFormat{nolabel}{}
    \captionsetup{labelformat=nolabel}

    \usepackage{adjustbox} % Used to constrain images to a maximum size 
    \usepackage{xcolor} % Allow colors to be defined
    \usepackage{enumerate} % Needed for markdown enumerations to work
    \usepackage{geometry} % Used to adjust the document margins
    \usepackage{amsmath} % Equations
    \usepackage{amssymb} % Equations
    \usepackage{textcomp} % defines textquotesingle
    % Hack from http://tex.stackexchange.com/a/47451/13684:
    \AtBeginDocument{%
        \def\PYZsq{\textquotesingle}% Upright quotes in Pygmentized code
    }
    \usepackage{upquote} % Upright quotes for verbatim code
    \usepackage{eurosym} % defines \euro
    \usepackage[mathletters]{ucs} % Extended unicode (utf-8) support
    \usepackage[utf8x]{inputenc} % Allow utf-8 characters in the tex document
    \usepackage{fancyvrb} % verbatim replacement that allows latex
    \usepackage{grffile} % extends the file name processing of package graphics 
                         % to support a larger range 
    % The hyperref package gives us a pdf with properly built
    % internal navigation ('pdf bookmarks' for the table of contents,
    % internal cross-reference links, web links for URLs, etc.)
    \usepackage{hyperref}
    \usepackage{longtable} % longtable support required by pandoc >1.10
    \usepackage{booktabs}  % table support for pandoc > 1.12.2
    \usepackage[inline]{enumitem} % IRkernel/repr support (it uses the enumerate* environment)
    \usepackage[normalem]{ulem} % ulem is needed to support strikethroughs (\sout)
                                % normalem makes italics be italics, not underlines
    

    
    
    % Colors for the hyperref package
    \definecolor{urlcolor}{rgb}{0,.145,.698}
    \definecolor{linkcolor}{rgb}{.71,0.21,0.01}
    \definecolor{citecolor}{rgb}{.12,.54,.11}

    % ANSI colors
    \definecolor{ansi-black}{HTML}{3E424D}
    \definecolor{ansi-black-intense}{HTML}{282C36}
    \definecolor{ansi-red}{HTML}{E75C58}
    \definecolor{ansi-red-intense}{HTML}{B22B31}
    \definecolor{ansi-green}{HTML}{00A250}
    \definecolor{ansi-green-intense}{HTML}{007427}
    \definecolor{ansi-yellow}{HTML}{DDB62B}
    \definecolor{ansi-yellow-intense}{HTML}{B27D12}
    \definecolor{ansi-blue}{HTML}{208FFB}
    \definecolor{ansi-blue-intense}{HTML}{0065CA}
    \definecolor{ansi-magenta}{HTML}{D160C4}
    \definecolor{ansi-magenta-intense}{HTML}{A03196}
    \definecolor{ansi-cyan}{HTML}{60C6C8}
    \definecolor{ansi-cyan-intense}{HTML}{258F8F}
    \definecolor{ansi-white}{HTML}{C5C1B4}
    \definecolor{ansi-white-intense}{HTML}{A1A6B2}

    % commands and environments needed by pandoc snippets
    % extracted from the output of `pandoc -s`
    \providecommand{\tightlist}{%
      \setlength{\itemsep}{0pt}\setlength{\parskip}{0pt}}
    \DefineVerbatimEnvironment{Highlighting}{Verbatim}{commandchars=\\\{\}}
    % Add ',fontsize=\small' for more characters per line
    \newenvironment{Shaded}{}{}
    \newcommand{\KeywordTok}[1]{\textcolor[rgb]{0.00,0.44,0.13}{\textbf{{#1}}}}
    \newcommand{\DataTypeTok}[1]{\textcolor[rgb]{0.56,0.13,0.00}{{#1}}}
    \newcommand{\DecValTok}[1]{\textcolor[rgb]{0.25,0.63,0.44}{{#1}}}
    \newcommand{\BaseNTok}[1]{\textcolor[rgb]{0.25,0.63,0.44}{{#1}}}
    \newcommand{\FloatTok}[1]{\textcolor[rgb]{0.25,0.63,0.44}{{#1}}}
    \newcommand{\CharTok}[1]{\textcolor[rgb]{0.25,0.44,0.63}{{#1}}}
    \newcommand{\StringTok}[1]{\textcolor[rgb]{0.25,0.44,0.63}{{#1}}}
    \newcommand{\CommentTok}[1]{\textcolor[rgb]{0.38,0.63,0.69}{\textit{{#1}}}}
    \newcommand{\OtherTok}[1]{\textcolor[rgb]{0.00,0.44,0.13}{{#1}}}
    \newcommand{\AlertTok}[1]{\textcolor[rgb]{1.00,0.00,0.00}{\textbf{{#1}}}}
    \newcommand{\FunctionTok}[1]{\textcolor[rgb]{0.02,0.16,0.49}{{#1}}}
    \newcommand{\RegionMarkerTok}[1]{{#1}}
    \newcommand{\ErrorTok}[1]{\textcolor[rgb]{1.00,0.00,0.00}{\textbf{{#1}}}}
    \newcommand{\NormalTok}[1]{{#1}}
    
    % Additional commands for more recent versions of Pandoc
    \newcommand{\ConstantTok}[1]{\textcolor[rgb]{0.53,0.00,0.00}{{#1}}}
    \newcommand{\SpecialCharTok}[1]{\textcolor[rgb]{0.25,0.44,0.63}{{#1}}}
    \newcommand{\VerbatimStringTok}[1]{\textcolor[rgb]{0.25,0.44,0.63}{{#1}}}
    \newcommand{\SpecialStringTok}[1]{\textcolor[rgb]{0.73,0.40,0.53}{{#1}}}
    \newcommand{\ImportTok}[1]{{#1}}
    \newcommand{\DocumentationTok}[1]{\textcolor[rgb]{0.73,0.13,0.13}{\textit{{#1}}}}
    \newcommand{\AnnotationTok}[1]{\textcolor[rgb]{0.38,0.63,0.69}{\textbf{\textit{{#1}}}}}
    \newcommand{\CommentVarTok}[1]{\textcolor[rgb]{0.38,0.63,0.69}{\textbf{\textit{{#1}}}}}
    \newcommand{\VariableTok}[1]{\textcolor[rgb]{0.10,0.09,0.49}{{#1}}}
    \newcommand{\ControlFlowTok}[1]{\textcolor[rgb]{0.00,0.44,0.13}{\textbf{{#1}}}}
    \newcommand{\OperatorTok}[1]{\textcolor[rgb]{0.40,0.40,0.40}{{#1}}}
    \newcommand{\BuiltInTok}[1]{{#1}}
    \newcommand{\ExtensionTok}[1]{{#1}}
    \newcommand{\PreprocessorTok}[1]{\textcolor[rgb]{0.74,0.48,0.00}{{#1}}}
    \newcommand{\AttributeTok}[1]{\textcolor[rgb]{0.49,0.56,0.16}{{#1}}}
    \newcommand{\InformationTok}[1]{\textcolor[rgb]{0.38,0.63,0.69}{\textbf{\textit{{#1}}}}}
    \newcommand{\WarningTok}[1]{\textcolor[rgb]{0.38,0.63,0.69}{\textbf{\textit{{#1}}}}}
    
    
    % Define a nice break command that doesn't care if a line doesn't already
    % exist.
    \def\br{\hspace*{\fill} \\* }
    % Math Jax compatability definitions
    \def\gt{>}
    \def\lt{<}
    % Document parameters
    \title{Python3 Official Tutorial Notebook}
    
    
    

    % Pygments definitions
    
\makeatletter
\def\PY@reset{\let\PY@it=\relax \let\PY@bf=\relax%
    \let\PY@ul=\relax \let\PY@tc=\relax%
    \let\PY@bc=\relax \let\PY@ff=\relax}
\def\PY@tok#1{\csname PY@tok@#1\endcsname}
\def\PY@toks#1+{\ifx\relax#1\empty\else%
    \PY@tok{#1}\expandafter\PY@toks\fi}
\def\PY@do#1{\PY@bc{\PY@tc{\PY@ul{%
    \PY@it{\PY@bf{\PY@ff{#1}}}}}}}
\def\PY#1#2{\PY@reset\PY@toks#1+\relax+\PY@do{#2}}

\expandafter\def\csname PY@tok@w\endcsname{\def\PY@tc##1{\textcolor[rgb]{0.73,0.73,0.73}{##1}}}
\expandafter\def\csname PY@tok@c\endcsname{\let\PY@it=\textit\def\PY@tc##1{\textcolor[rgb]{0.25,0.50,0.50}{##1}}}
\expandafter\def\csname PY@tok@cp\endcsname{\def\PY@tc##1{\textcolor[rgb]{0.74,0.48,0.00}{##1}}}
\expandafter\def\csname PY@tok@k\endcsname{\let\PY@bf=\textbf\def\PY@tc##1{\textcolor[rgb]{0.00,0.50,0.00}{##1}}}
\expandafter\def\csname PY@tok@kp\endcsname{\def\PY@tc##1{\textcolor[rgb]{0.00,0.50,0.00}{##1}}}
\expandafter\def\csname PY@tok@kt\endcsname{\def\PY@tc##1{\textcolor[rgb]{0.69,0.00,0.25}{##1}}}
\expandafter\def\csname PY@tok@o\endcsname{\def\PY@tc##1{\textcolor[rgb]{0.40,0.40,0.40}{##1}}}
\expandafter\def\csname PY@tok@ow\endcsname{\let\PY@bf=\textbf\def\PY@tc##1{\textcolor[rgb]{0.67,0.13,1.00}{##1}}}
\expandafter\def\csname PY@tok@nb\endcsname{\def\PY@tc##1{\textcolor[rgb]{0.00,0.50,0.00}{##1}}}
\expandafter\def\csname PY@tok@nf\endcsname{\def\PY@tc##1{\textcolor[rgb]{0.00,0.00,1.00}{##1}}}
\expandafter\def\csname PY@tok@nc\endcsname{\let\PY@bf=\textbf\def\PY@tc##1{\textcolor[rgb]{0.00,0.00,1.00}{##1}}}
\expandafter\def\csname PY@tok@nn\endcsname{\let\PY@bf=\textbf\def\PY@tc##1{\textcolor[rgb]{0.00,0.00,1.00}{##1}}}
\expandafter\def\csname PY@tok@ne\endcsname{\let\PY@bf=\textbf\def\PY@tc##1{\textcolor[rgb]{0.82,0.25,0.23}{##1}}}
\expandafter\def\csname PY@tok@nv\endcsname{\def\PY@tc##1{\textcolor[rgb]{0.10,0.09,0.49}{##1}}}
\expandafter\def\csname PY@tok@no\endcsname{\def\PY@tc##1{\textcolor[rgb]{0.53,0.00,0.00}{##1}}}
\expandafter\def\csname PY@tok@nl\endcsname{\def\PY@tc##1{\textcolor[rgb]{0.63,0.63,0.00}{##1}}}
\expandafter\def\csname PY@tok@ni\endcsname{\let\PY@bf=\textbf\def\PY@tc##1{\textcolor[rgb]{0.60,0.60,0.60}{##1}}}
\expandafter\def\csname PY@tok@na\endcsname{\def\PY@tc##1{\textcolor[rgb]{0.49,0.56,0.16}{##1}}}
\expandafter\def\csname PY@tok@nt\endcsname{\let\PY@bf=\textbf\def\PY@tc##1{\textcolor[rgb]{0.00,0.50,0.00}{##1}}}
\expandafter\def\csname PY@tok@nd\endcsname{\def\PY@tc##1{\textcolor[rgb]{0.67,0.13,1.00}{##1}}}
\expandafter\def\csname PY@tok@s\endcsname{\def\PY@tc##1{\textcolor[rgb]{0.73,0.13,0.13}{##1}}}
\expandafter\def\csname PY@tok@sd\endcsname{\let\PY@it=\textit\def\PY@tc##1{\textcolor[rgb]{0.73,0.13,0.13}{##1}}}
\expandafter\def\csname PY@tok@si\endcsname{\let\PY@bf=\textbf\def\PY@tc##1{\textcolor[rgb]{0.73,0.40,0.53}{##1}}}
\expandafter\def\csname PY@tok@se\endcsname{\let\PY@bf=\textbf\def\PY@tc##1{\textcolor[rgb]{0.73,0.40,0.13}{##1}}}
\expandafter\def\csname PY@tok@sr\endcsname{\def\PY@tc##1{\textcolor[rgb]{0.73,0.40,0.53}{##1}}}
\expandafter\def\csname PY@tok@ss\endcsname{\def\PY@tc##1{\textcolor[rgb]{0.10,0.09,0.49}{##1}}}
\expandafter\def\csname PY@tok@sx\endcsname{\def\PY@tc##1{\textcolor[rgb]{0.00,0.50,0.00}{##1}}}
\expandafter\def\csname PY@tok@m\endcsname{\def\PY@tc##1{\textcolor[rgb]{0.40,0.40,0.40}{##1}}}
\expandafter\def\csname PY@tok@gh\endcsname{\let\PY@bf=\textbf\def\PY@tc##1{\textcolor[rgb]{0.00,0.00,0.50}{##1}}}
\expandafter\def\csname PY@tok@gu\endcsname{\let\PY@bf=\textbf\def\PY@tc##1{\textcolor[rgb]{0.50,0.00,0.50}{##1}}}
\expandafter\def\csname PY@tok@gd\endcsname{\def\PY@tc##1{\textcolor[rgb]{0.63,0.00,0.00}{##1}}}
\expandafter\def\csname PY@tok@gi\endcsname{\def\PY@tc##1{\textcolor[rgb]{0.00,0.63,0.00}{##1}}}
\expandafter\def\csname PY@tok@gr\endcsname{\def\PY@tc##1{\textcolor[rgb]{1.00,0.00,0.00}{##1}}}
\expandafter\def\csname PY@tok@ge\endcsname{\let\PY@it=\textit}
\expandafter\def\csname PY@tok@gs\endcsname{\let\PY@bf=\textbf}
\expandafter\def\csname PY@tok@gp\endcsname{\let\PY@bf=\textbf\def\PY@tc##1{\textcolor[rgb]{0.00,0.00,0.50}{##1}}}
\expandafter\def\csname PY@tok@go\endcsname{\def\PY@tc##1{\textcolor[rgb]{0.53,0.53,0.53}{##1}}}
\expandafter\def\csname PY@tok@gt\endcsname{\def\PY@tc##1{\textcolor[rgb]{0.00,0.27,0.87}{##1}}}
\expandafter\def\csname PY@tok@err\endcsname{\def\PY@bc##1{\setlength{\fboxsep}{0pt}\fcolorbox[rgb]{1.00,0.00,0.00}{1,1,1}{\strut ##1}}}
\expandafter\def\csname PY@tok@kc\endcsname{\let\PY@bf=\textbf\def\PY@tc##1{\textcolor[rgb]{0.00,0.50,0.00}{##1}}}
\expandafter\def\csname PY@tok@kd\endcsname{\let\PY@bf=\textbf\def\PY@tc##1{\textcolor[rgb]{0.00,0.50,0.00}{##1}}}
\expandafter\def\csname PY@tok@kn\endcsname{\let\PY@bf=\textbf\def\PY@tc##1{\textcolor[rgb]{0.00,0.50,0.00}{##1}}}
\expandafter\def\csname PY@tok@kr\endcsname{\let\PY@bf=\textbf\def\PY@tc##1{\textcolor[rgb]{0.00,0.50,0.00}{##1}}}
\expandafter\def\csname PY@tok@bp\endcsname{\def\PY@tc##1{\textcolor[rgb]{0.00,0.50,0.00}{##1}}}
\expandafter\def\csname PY@tok@fm\endcsname{\def\PY@tc##1{\textcolor[rgb]{0.00,0.00,1.00}{##1}}}
\expandafter\def\csname PY@tok@vc\endcsname{\def\PY@tc##1{\textcolor[rgb]{0.10,0.09,0.49}{##1}}}
\expandafter\def\csname PY@tok@vg\endcsname{\def\PY@tc##1{\textcolor[rgb]{0.10,0.09,0.49}{##1}}}
\expandafter\def\csname PY@tok@vi\endcsname{\def\PY@tc##1{\textcolor[rgb]{0.10,0.09,0.49}{##1}}}
\expandafter\def\csname PY@tok@vm\endcsname{\def\PY@tc##1{\textcolor[rgb]{0.10,0.09,0.49}{##1}}}
\expandafter\def\csname PY@tok@sa\endcsname{\def\PY@tc##1{\textcolor[rgb]{0.73,0.13,0.13}{##1}}}
\expandafter\def\csname PY@tok@sb\endcsname{\def\PY@tc##1{\textcolor[rgb]{0.73,0.13,0.13}{##1}}}
\expandafter\def\csname PY@tok@sc\endcsname{\def\PY@tc##1{\textcolor[rgb]{0.73,0.13,0.13}{##1}}}
\expandafter\def\csname PY@tok@dl\endcsname{\def\PY@tc##1{\textcolor[rgb]{0.73,0.13,0.13}{##1}}}
\expandafter\def\csname PY@tok@s2\endcsname{\def\PY@tc##1{\textcolor[rgb]{0.73,0.13,0.13}{##1}}}
\expandafter\def\csname PY@tok@sh\endcsname{\def\PY@tc##1{\textcolor[rgb]{0.73,0.13,0.13}{##1}}}
\expandafter\def\csname PY@tok@s1\endcsname{\def\PY@tc##1{\textcolor[rgb]{0.73,0.13,0.13}{##1}}}
\expandafter\def\csname PY@tok@mb\endcsname{\def\PY@tc##1{\textcolor[rgb]{0.40,0.40,0.40}{##1}}}
\expandafter\def\csname PY@tok@mf\endcsname{\def\PY@tc##1{\textcolor[rgb]{0.40,0.40,0.40}{##1}}}
\expandafter\def\csname PY@tok@mh\endcsname{\def\PY@tc##1{\textcolor[rgb]{0.40,0.40,0.40}{##1}}}
\expandafter\def\csname PY@tok@mi\endcsname{\def\PY@tc##1{\textcolor[rgb]{0.40,0.40,0.40}{##1}}}
\expandafter\def\csname PY@tok@il\endcsname{\def\PY@tc##1{\textcolor[rgb]{0.40,0.40,0.40}{##1}}}
\expandafter\def\csname PY@tok@mo\endcsname{\def\PY@tc##1{\textcolor[rgb]{0.40,0.40,0.40}{##1}}}
\expandafter\def\csname PY@tok@ch\endcsname{\let\PY@it=\textit\def\PY@tc##1{\textcolor[rgb]{0.25,0.50,0.50}{##1}}}
\expandafter\def\csname PY@tok@cm\endcsname{\let\PY@it=\textit\def\PY@tc##1{\textcolor[rgb]{0.25,0.50,0.50}{##1}}}
\expandafter\def\csname PY@tok@cpf\endcsname{\let\PY@it=\textit\def\PY@tc##1{\textcolor[rgb]{0.25,0.50,0.50}{##1}}}
\expandafter\def\csname PY@tok@c1\endcsname{\let\PY@it=\textit\def\PY@tc##1{\textcolor[rgb]{0.25,0.50,0.50}{##1}}}
\expandafter\def\csname PY@tok@cs\endcsname{\let\PY@it=\textit\def\PY@tc##1{\textcolor[rgb]{0.25,0.50,0.50}{##1}}}

\def\PYZbs{\char`\\}
\def\PYZus{\char`\_}
\def\PYZob{\char`\{}
\def\PYZcb{\char`\}}
\def\PYZca{\char`\^}
\def\PYZam{\char`\&}
\def\PYZlt{\char`\<}
\def\PYZgt{\char`\>}
\def\PYZsh{\char`\#}
\def\PYZpc{\char`\%}
\def\PYZdl{\char`\$}
\def\PYZhy{\char`\-}
\def\PYZsq{\char`\'}
\def\PYZdq{\char`\"}
\def\PYZti{\char`\~}
% for compatibility with earlier versions
\def\PYZat{@}
\def\PYZlb{[}
\def\PYZrb{]}
\makeatother


    % Exact colors from NB
    \definecolor{incolor}{rgb}{0.0, 0.0, 0.5}
    \definecolor{outcolor}{rgb}{0.545, 0.0, 0.0}



    
    % Prevent overflowing lines due to hard-to-break entities
    \sloppy 
    % Setup hyperref package
    \hypersetup{
      breaklinks=true,  % so long urls are correctly broken across lines
      colorlinks=true,
      urlcolor=urlcolor,
      linkcolor=linkcolor,
      citecolor=citecolor,
      }
    % Slightly bigger margins than the latex defaults
    
    \geometry{verbose,tmargin=1in,bmargin=1in,lmargin=1in,rmargin=1in}
    
    

    \begin{document}
    
    
    \maketitle
    
    

    
    \subsection{Python Built-in functions playground - 75 in
total}\label{python-built-in-functions-playground---75-in-total}

    \begin{Verbatim}[commandchars=\\\{\}]
{\color{incolor}In [{\color{incolor}1}]:} \PY{c+c1}{\PYZsh{} all(iterable) return True if all elements of iterable are True}
        
        \PY{n}{my\PYZus{}list} \PY{o}{=} \PY{p}{[}\PY{l+s+s1}{\PYZsq{}}\PY{l+s+s1}{o}\PY{l+s+s1}{\PYZsq{}}\PY{p}{,} \PY{l+m+mi}{1}\PY{p}{,} \PY{l+m+mi}{2}\PY{p}{,} \PY{l+s+s1}{\PYZsq{}}\PY{l+s+s1}{Text}\PY{l+s+s1}{\PYZsq{}}\PY{p}{,} \PY{k+kc}{True}\PY{p}{,} \PY{l+s+s1}{\PYZsq{}}\PY{l+s+s1}{abc}\PY{l+s+s1}{\PYZsq{}}\PY{p}{]} \PY{c+c1}{\PYZsh{} list elements can be different types}
        \PY{n+nb}{print}\PY{p}{(}\PY{n}{f}\PY{l+s+s2}{\PYZdq{}}\PY{l+s+s2}{My list is: }\PY{l+s+se}{\PYZbs{}n}\PY{l+s+s2}{ }\PY{l+s+si}{\PYZob{}my\PYZus{}list\PYZcb{}}\PY{l+s+s2}{, }\PY{l+s+se}{\PYZbs{}n}\PY{l+s+s2}{ the result of all() is: }\PY{l+s+s2}{\PYZob{}}\PY{l+s+s2}{all(my\PYZus{}list)\PYZcb{}}\PY{l+s+s2}{\PYZdq{}}\PY{p}{)}
        \PY{n+nb}{print}\PY{p}{(}\PY{l+s+s1}{\PYZsq{}}\PY{l+s+s1}{*}\PY{l+s+s1}{\PYZsq{}} \PY{o}{*} \PY{l+m+mi}{15}\PY{p}{)}
        \PY{n}{my\PYZus{}list}\PY{o}{.}\PY{n}{append}\PY{p}{(}\PY{l+s+s1}{\PYZsq{}}\PY{l+s+s1}{\PYZsq{}}\PY{p}{)} \PY{c+c1}{\PYZsh{} \PYZsq{}\PYZsq{} 空字符 被当做False处理。}
        \PY{n+nb}{print}\PY{p}{(}\PY{n}{f}\PY{l+s+s2}{\PYZdq{}}\PY{l+s+s2}{My list is updated as: }\PY{l+s+se}{\PYZbs{}n}\PY{l+s+s2}{ }\PY{l+s+si}{\PYZob{}my\PYZus{}list\PYZcb{}}\PY{l+s+s2}{, }\PY{l+s+se}{\PYZbs{}n}\PY{l+s+s2}{ the result of all() is now: }\PY{l+s+s2}{\PYZob{}}\PY{l+s+s2}{all(my\PYZus{}list)\PYZcb{}}\PY{l+s+s2}{\PYZdq{}}\PY{p}{)}
\end{Verbatim}


    \begin{Verbatim}[commandchars=\\\{\}]
My list is: 
 ['o', 1, 2, 'Text', True, 'abc'], 
 the result of all() is: True
***************
My list is updated as: 
 ['o', 1, 2, 'Text', True, 'abc', ''], 
 the result of all() is now: False

    \end{Verbatim}

    \begin{Verbatim}[commandchars=\\\{\}]
{\color{incolor}In [{\color{incolor}2}]:} \PY{c+c1}{\PYZsh{} any() similar to all()}
        
        \PY{n+nb}{print}\PY{p}{(}\PY{n+nb}{any}\PY{p}{(}\PY{n}{my\PYZus{}list}\PY{p}{)}\PY{p}{)}
\end{Verbatim}


    \begin{Verbatim}[commandchars=\\\{\}]
True

    \end{Verbatim}

    \begin{Verbatim}[commandchars=\\\{\}]
{\color{incolor}In [{\color{incolor}3}]:} \PY{c+c1}{\PYZsh{} ascii() returns the ascii string of a string. 我理解为ascii的参数是想要输出的内容,ascii的结果为对应的输入。相当于给定输出求输入??}
        
        \PY{n+nb}{print}\PY{p}{(}\PY{n}{ascii}\PY{p}{(}\PY{l+s+s1}{\PYZsq{}}\PY{l+s+se}{\PYZbs{}\PYZbs{}}\PY{l+s+s1}{\PYZsq{}}\PY{p}{)} \PY{o}{==} \PY{l+s+s2}{\PYZdq{}}\PY{l+s+s2}{\PYZsq{}}\PY{l+s+se}{\PYZbs{}\PYZbs{}}\PY{l+s+se}{\PYZbs{}\PYZbs{}}\PY{l+s+s2}{\PYZsq{}}\PY{l+s+s2}{\PYZdq{}}\PY{p}{)}
        \PY{n+nb}{print}\PY{p}{(}\PY{n}{ascii}\PY{p}{(}\PY{l+s+s1}{\PYZsq{}}\PY{l+s+se}{\PYZbs{}n}\PY{l+s+s1}{\PYZsq{}}\PY{p}{)} \PY{o}{==} \PY{l+s+s2}{\PYZdq{}}\PY{l+s+s2}{\PYZsq{}}\PY{l+s+se}{\PYZbs{}\PYZbs{}}\PY{l+s+s2}{n}\PY{l+s+s2}{\PYZsq{}}\PY{l+s+s2}{\PYZdq{}}\PY{p}{)}
        \PY{n+nb}{print}\PY{p}{(}\PY{n}{ascii}\PY{p}{(}\PY{l+s+s1}{\PYZsq{}}\PY{l+s+s1}{0b10}\PY{l+s+s1}{\PYZsq{}}\PY{p}{)} \PY{o}{==} \PY{l+s+s2}{\PYZdq{}}\PY{l+s+s2}{\PYZsq{}}\PY{l+s+s2}{2}\PY{l+s+s2}{\PYZsq{}}\PY{l+s+s2}{\PYZdq{}}\PY{p}{)}
        \PY{n+nb}{print}\PY{p}{(}\PY{n}{ascii}\PY{p}{(}\PY{l+s+s1}{\PYZsq{}}\PY{l+s+s1}{0b10}\PY{l+s+s1}{\PYZsq{}}\PY{p}{)}\PY{p}{)}
\end{Verbatim}


    \begin{Verbatim}[commandchars=\\\{\}]
True
True
False
'0b10'

    \end{Verbatim}

    \begin{Verbatim}[commandchars=\\\{\}]
{\color{incolor}In [{\color{incolor}4}]:} \PY{c+c1}{\PYZsh{} bin() Convert an integer number to a binary string prefixed with “0b”.}
        
        \PY{n+nb}{bin}\PY{p}{(}\PY{l+m+mi}{2}\PY{p}{)}
\end{Verbatim}


\begin{Verbatim}[commandchars=\\\{\}]
{\color{outcolor}Out[{\color{outcolor}4}]:} '0b10'
\end{Verbatim}
            
    \begin{Verbatim}[commandchars=\\\{\}]
{\color{incolor}In [{\color{incolor}5}]:} \PY{n+nb}{format}\PY{p}{(}\PY{l+m+mi}{14}\PY{p}{,} \PY{l+s+s1}{\PYZsq{}}\PY{l+s+s1}{\PYZsh{}b}\PY{l+s+s1}{\PYZsq{}}\PY{p}{)}\PY{p}{,} \PY{n+nb}{format}\PY{p}{(}\PY{l+m+mi}{14}\PY{p}{,} \PY{l+s+s1}{\PYZsq{}}\PY{l+s+s1}{b}\PY{l+s+s1}{\PYZsq{}}\PY{p}{)} \PY{c+c1}{\PYZsh{} 不太懂 要学习一下format函数}
\end{Verbatim}


\begin{Verbatim}[commandchars=\\\{\}]
{\color{outcolor}Out[{\color{outcolor}5}]:} ('0b1110', '1110')
\end{Verbatim}
            
    \begin{Verbatim}[commandchars=\\\{\}]
{\color{incolor}In [{\color{incolor}6}]:} \PY{c+c1}{\PYZsh{} class bool([x]) return a Boolean value, i.e. one of True or False. class部分要学习一下,回头再看这个。}
        
        \PY{c+c1}{\PYZsh{}  x is now a positional\PYZhy{}only parameter??? 这个的含义也不懂。}
\end{Verbatim}


    \begin{Verbatim}[commandchars=\\\{\}]
{\color{incolor}In [{\color{incolor}7}]:} \PY{c+c1}{\PYZsh{} breakpoint(*args, **kws)  debug function.暂时不懂。}
\end{Verbatim}


    \begin{Verbatim}[commandchars=\\\{\}]
{\color{incolor}In [{\color{incolor}8}]:} \PY{c+c1}{\PYZsh{} class bytearray([source[, encoding[, errors]]]) }
\end{Verbatim}


    zip(*iterables)

Returns an iterator of tuples, where the \emph{i}-th tuple contains the
\emph{i}-th element from each of the argument sequences or iterables. It
stops when the shortest input iterable is exhausted.

For example, it can do transpose on 2-d matrix:

    \begin{Verbatim}[commandchars=\\\{\}]
{\color{incolor}In [{\color{incolor}9}]:} \PY{n}{matrix} \PY{o}{=} \PY{p}{[}\PY{p}{[}\PY{l+m+mi}{1}\PY{p}{,} \PY{l+m+mi}{2}\PY{p}{,} \PY{l+m+mi}{3}\PY{p}{]}\PY{p}{,} \PY{p}{[}\PY{l+m+mi}{4}\PY{p}{,} \PY{l+m+mi}{5}\PY{p}{,} \PY{l+m+mi}{6}\PY{p}{]}\PY{p}{,} \PY{p}{[}\PY{l+m+mi}{7}\PY{p}{,} \PY{l+m+mi}{8}\PY{p}{,} \PY{l+m+mi}{9}\PY{p}{]}\PY{p}{,} \PY{p}{[}\PY{l+m+mi}{10}\PY{p}{,} \PY{l+m+mi}{11}\PY{p}{,} \PY{l+m+mi}{12}\PY{p}{]}\PY{p}{]}
        \PY{n}{matrix}
\end{Verbatim}


\begin{Verbatim}[commandchars=\\\{\}]
{\color{outcolor}Out[{\color{outcolor}9}]:} [[1, 2, 3], [4, 5, 6], [7, 8, 9], [10, 11, 12]]
\end{Verbatim}
            
    \begin{Verbatim}[commandchars=\\\{\}]
{\color{incolor}In [{\color{incolor}10}]:} \PY{n+nb}{list}\PY{p}{(}\PY{n+nb}{zip}\PY{p}{(}\PY{o}{*}\PY{n}{matrix}\PY{p}{)}\PY{p}{)}
\end{Verbatim}


\begin{Verbatim}[commandchars=\\\{\}]
{\color{outcolor}Out[{\color{outcolor}10}]:} [(1, 4, 7, 10), (2, 5, 8, 11), (3, 6, 9, 12)]
\end{Verbatim}
            
    \subsection{\texorpdfstring{Sequence Types -\/- \textbf{list},
\textbf{tuple},
\textbf{range}}{Sequence Types -\/- list, tuple, range}}\label{sequence-types----list-tuple-range}

    \subsubsection{Common Sequence
Operations}\label{common-sequence-operations}

    The operations in the following table are supported by most sequence
typs, both mutable and immutable.

In this table \textbf{s} and \textbf{t} are sequences of the same type,
\textbf{n}, \textbf{i}, \textbf{j} and \textbf{k} are integers and
\textbf{x} is an arbitrary object that meets any type and value
restrictions imposed by \textbf{s}.

The \textbf{in} and \textbf{not in} have the same priorities as the
comparison operations. The \textbf{+} (concatenation) and \textbf{*}
(repetition) have the same priority as the corresponding numeric
operations.

\begin{longtable}[]{@{}lll@{}}
\toprule
\begin{minipage}[b]{0.14\columnwidth}\raggedright\strut
Operation\strut
\end{minipage} & \begin{minipage}[b]{0.14\columnwidth}\raggedright\strut
Result\strut
\end{minipage} & \begin{minipage}[b]{0.14\columnwidth}\raggedright\strut
Notes\strut
\end{minipage}\tabularnewline
\midrule
\endhead
\begin{minipage}[t]{0.14\columnwidth}\raggedright\strut
x in s\strut
\end{minipage} & \begin{minipage}[t]{0.14\columnwidth}\raggedright\strut
\textbf{True} if an item of s is equal to x, else \textbf{False}\strut
\end{minipage} & \begin{minipage}[t]{0.14\columnwidth}\raggedright\strut
1\strut
\end{minipage}\tabularnewline
\begin{minipage}[t]{0.14\columnwidth}\raggedright\strut
x not in s\strut
\end{minipage} & \begin{minipage}[t]{0.14\columnwidth}\raggedright\strut
\textbf{False} if an item of s is euqal to x, else \textbf{False}\strut
\end{minipage} & \begin{minipage}[t]{0.14\columnwidth}\raggedright\strut
1\strut
\end{minipage}\tabularnewline
\begin{minipage}[t]{0.14\columnwidth}\raggedright\strut
s + t\strut
\end{minipage} & \begin{minipage}[t]{0.14\columnwidth}\raggedright\strut
the concatenation of s and t\strut
\end{minipage} & \begin{minipage}[t]{0.14\columnwidth}\raggedright\strut
6 7\strut
\end{minipage}\tabularnewline
\begin{minipage}[t]{0.14\columnwidth}\raggedright\strut
s * n or n * s\strut
\end{minipage} & \begin{minipage}[t]{0.14\columnwidth}\raggedright\strut
equivalent to adding s to itself n times\strut
\end{minipage} & \begin{minipage}[t]{0.14\columnwidth}\raggedright\strut
2 7\strut
\end{minipage}\tabularnewline
\begin{minipage}[t]{0.14\columnwidth}\raggedright\strut
s{[}i{]}\strut
\end{minipage} & \begin{minipage}[t]{0.14\columnwidth}\raggedright\strut
\emph{i}th item of s, origin 0\strut
\end{minipage} & \begin{minipage}[t]{0.14\columnwidth}\raggedright\strut
3\strut
\end{minipage}\tabularnewline
\begin{minipage}[t]{0.14\columnwidth}\raggedright\strut
s{[}i:j{]}\strut
\end{minipage} & \begin{minipage}[t]{0.14\columnwidth}\raggedright\strut
slice of s from i(inc) to j(exc)\strut
\end{minipage} & \begin{minipage}[t]{0.14\columnwidth}\raggedright\strut
3 4\strut
\end{minipage}\tabularnewline
\begin{minipage}[t]{0.14\columnwidth}\raggedright\strut
s{[}i:j:k{]}\strut
\end{minipage} & \begin{minipage}[t]{0.14\columnwidth}\raggedright\strut
slice of s from i to j with step k\strut
\end{minipage} & \begin{minipage}[t]{0.14\columnwidth}\raggedright\strut
3 5\strut
\end{minipage}\tabularnewline
\begin{minipage}[t]{0.14\columnwidth}\raggedright\strut
len(s)\strut
\end{minipage} & \begin{minipage}[t]{0.14\columnwidth}\raggedright\strut
length of s\strut
\end{minipage} & \begin{minipage}[t]{0.14\columnwidth}\raggedright\strut
\strut
\end{minipage}\tabularnewline
\begin{minipage}[t]{0.14\columnwidth}\raggedright\strut
min(s)\strut
\end{minipage} & \begin{minipage}[t]{0.14\columnwidth}\raggedright\strut
smallest item of s\strut
\end{minipage} & \begin{minipage}[t]{0.14\columnwidth}\raggedright\strut
\strut
\end{minipage}\tabularnewline
\begin{minipage}[t]{0.14\columnwidth}\raggedright\strut
max(s)\strut
\end{minipage} & \begin{minipage}[t]{0.14\columnwidth}\raggedright\strut
largest item of s\strut
\end{minipage} & \begin{minipage}[t]{0.14\columnwidth}\raggedright\strut
\strut
\end{minipage}\tabularnewline
\begin{minipage}[t]{0.14\columnwidth}\raggedright\strut
s.index(x{[}, i{[}, j{]}{]})\strut
\end{minipage} & \begin{minipage}[t]{0.14\columnwidth}\raggedright\strut
index of the first occurrence of x in s (at or after index i and before
index j)\strut
\end{minipage} & \begin{minipage}[t]{0.14\columnwidth}\raggedright\strut
8\strut
\end{minipage}\tabularnewline
\begin{minipage}[t]{0.14\columnwidth}\raggedright\strut
s.count(x)\strut
\end{minipage} & \begin{minipage}[t]{0.14\columnwidth}\raggedright\strut
total number of occurrences of x in s\strut
\end{minipage} & \begin{minipage}[t]{0.14\columnwidth}\raggedright\strut
\strut
\end{minipage}\tabularnewline
\bottomrule
\end{longtable}

\begin{enumerate}
\def\labelenumi{\arabic{enumi}.}
\tightlist
\item
  While the \textbf{in} and \textbf{not in} operations are used only for
  simple containment testing in the general case, some specialised
  sequences (such as \textbf{str}, \textbf{bytes}, \textbf{bytearray})
  also use them for subsequence testing:
\end{enumerate}

    \begin{Verbatim}[commandchars=\\\{\}]
{\color{incolor}In [{\color{incolor}11}]:} \PY{l+s+s2}{\PYZdq{}}\PY{l+s+s2}{gg}\PY{l+s+s2}{\PYZdq{}} \PY{o+ow}{in} \PY{l+s+s2}{\PYZdq{}}\PY{l+s+s2}{eggs}\PY{l+s+s2}{\PYZdq{}}
\end{Verbatim}


\begin{Verbatim}[commandchars=\\\{\}]
{\color{outcolor}Out[{\color{outcolor}11}]:} True
\end{Verbatim}
            
    \begin{enumerate}
\def\labelenumi{\arabic{enumi}.}
\setcounter{enumi}{1}
\tightlist
\item
  Values of \emph{n} less than \textbf{0} are treated as \textbf{0}
  (which yields an empty sequence of the same type as s). Items in the
  sequence \textbf{s} are not copied; they are referenced multiple
  times: \sout{(这个例子不是太理解,后边要继续看append()
  回头看是不是可以理解这里了。)}
\end{enumerate}

    \begin{Verbatim}[commandchars=\\\{\}]
{\color{incolor}In [{\color{incolor}12}]:} \PY{l+s+sd}{\PYZsq{}\PYZsq{}\PYZsq{}Fist of all,we are talking about Sqeuence types here, so the following examle isn\PYZsq{}t part of this topic, it is here only for claritying }
         \PY{l+s+sd}{It is not the same case when multiply non\PYZhy{}list.\PYZsq{}\PYZsq{}\PYZsq{}}
         \PY{c+c1}{\PYZsh{} have to know THIS is wrong to multiply a one\PYZhy{}dimentional list:}
         \PY{n}{lists4} \PY{o}{=} \PY{p}{[}\PY{p}{]} \PY{o}{*} \PY{l+m+mi}{5} \PY{c+c1}{\PYZsh{} This is an empty list, nothing to multiply.}
         \PY{n}{lists4}
\end{Verbatim}


\begin{Verbatim}[commandchars=\\\{\}]
{\color{outcolor}Out[{\color{outcolor}12}]:} []
\end{Verbatim}
            
    \begin{Verbatim}[commandchars=\\\{\}]
{\color{incolor}In [{\color{incolor}13}]:} \PY{c+c1}{\PYZsh{} You have to put something in the [] in order to multiply a one\PYZhy{}dimentional list:}
         
         \PY{n}{list5} \PY{o}{=} \PY{p}{[}\PY{l+m+mi}{2}\PY{p}{]} \PY{o}{*} \PY{l+m+mi}{5}
         \PY{n}{list5}
\end{Verbatim}


\begin{Verbatim}[commandchars=\\\{\}]
{\color{outcolor}Out[{\color{outcolor}13}]:} [2, 2, 2, 2, 2]
\end{Verbatim}
            
    \begin{Verbatim}[commandchars=\\\{\}]
{\color{incolor}In [{\color{incolor}14}]:} \PY{c+c1}{\PYZsh{} Now try to change some item: }
         \PY{c+c1}{\PYZsh{} This example is different from the below, integers are multiplied here, not lists. Thus these items are independent from each other.}
         \PY{c+c1}{\PYZsh{} 不确定这样解释对不对,再看看吧. 这样解释说得通,被此处被复制的是int 2,而不是list;所以互相独立。}
         \PY{n}{list5}\PY{p}{[}\PY{l+m+mi}{0}\PY{p}{]} \PY{o}{=} \PY{l+m+mi}{1}
         \PY{n}{list5}
\end{Verbatim}


\begin{Verbatim}[commandchars=\\\{\}]
{\color{outcolor}Out[{\color{outcolor}14}]:} [1, 2, 2, 2, 2]
\end{Verbatim}
            
    Now we go back to 2.

    \begin{Verbatim}[commandchars=\\\{\}]
{\color{incolor}In [{\color{incolor}15}]:} \PY{n}{lists} \PY{o}{=} \PY{p}{[}\PY{p}{[}\PY{k+kc}{None}\PY{p}{]}\PY{p}{]} \PY{o}{*} \PY{l+m+mi}{5}    \PY{c+c1}{\PYZsh{}注意此处乘号在两层中括号的外侧,所以[None]被复制后集体在外层中括号的内部;\PYZbs{}}
         \PY{c+c1}{\PYZsh{} 被复制内容集体存在最外层中括号减1的位置。读不懂了看下下下边的有for loop的例子}
         \PY{n}{lists}
         \PY{n}{lists}\PY{p}{[}\PY{l+m+mi}{0}\PY{p}{]}\PY{o}{.}\PY{n}{append}\PY{p}{(}\PY{l+m+mi}{2}\PY{p}{)}
         \PY{n}{lists}
\end{Verbatim}


\begin{Verbatim}[commandchars=\\\{\}]
{\color{outcolor}Out[{\color{outcolor}15}]:} [[None, 2], [None, 2], [None, 2], [None, 2], [None, 2]]
\end{Verbatim}
            
    \begin{Verbatim}[commandchars=\\\{\}]
{\color{incolor}In [{\color{incolor}16}]:} \PY{n}{lists}\PY{p}{[}\PY{l+m+mi}{1}\PY{p}{]}\PY{o}{.}\PY{n}{append}\PY{p}{(}\PY{l+m+mi}{3}\PY{p}{)}
         \PY{n}{lists}
\end{Verbatim}


\begin{Verbatim}[commandchars=\\\{\}]
{\color{outcolor}Out[{\color{outcolor}16}]:} [[None, 2, 3], [None, 2, 3], [None, 2, 3], [None, 2, 3], [None, 2, 3]]
\end{Verbatim}
            
    What happened is that {[}{[}{]}{]} is a one-element list containing an
empty list, so all three elements of {[}{[}{]}{]} * 3 are references to
this single empty list.Modifying any of the elements of lists modifies
this single list. 就是说用 * 新建的list
指向同一个引用;更改一个的内容,其他的一同改变。\href{https://docs.python.org/3/faq/programming.html\#faq-multidimensional-list}{check
this for further explaination}

Use the following method to create a list of different lists this way:

    \begin{Verbatim}[commandchars=\\\{\}]
{\color{incolor}In [{\color{incolor}17}]:} \PY{n}{lists2} \PY{o}{=} \PY{p}{[}\PY{p}{[}\PY{p}{]} \PY{k}{for} \PY{n}{i} \PY{o+ow}{in} \PY{n+nb}{range}\PY{p}{(}\PY{l+m+mi}{3}\PY{p}{)}\PY{p}{]}
         \PY{n}{lists2}\PY{p}{[}\PY{l+m+mi}{0}\PY{p}{]}\PY{o}{.}\PY{n}{append}\PY{p}{(}\PY{l+m+mi}{2}\PY{p}{)}
         \PY{n}{lists2}\PY{p}{[}\PY{l+m+mi}{1}\PY{p}{]}\PY{o}{.}\PY{n}{append}\PY{p}{(}\PY{l+m+mi}{3}\PY{p}{)}
         \PY{n}{lists2}\PY{p}{[}\PY{l+m+mi}{2}\PY{p}{]}\PY{o}{.}\PY{n}{append}\PY{p}{(}\PY{l+m+mi}{5}\PY{p}{)}
         \PY{n}{lists2}
\end{Verbatim}


\begin{Verbatim}[commandchars=\\\{\}]
{\color{outcolor}Out[{\color{outcolor}17}]:} [[2], [3], [5]]
\end{Verbatim}
            
    \begin{Verbatim}[commandchars=\\\{\}]
{\color{incolor}In [{\color{incolor}18}]:} \PY{n}{lists3} \PY{o}{=} \PY{p}{[}\PY{p}{[}\PY{k+kc}{None}\PY{p}{]} \PY{o}{*} \PY{l+m+mi}{2} \PY{k}{for} \PY{n}{i} \PY{o+ow}{in} \PY{n+nb}{range}\PY{p}{(}\PY{l+m+mi}{3}\PY{p}{)}\PY{p}{]}    \PY{c+c1}{\PYZsh{}此处乘号在内外层中括号之间,所以multipy结果如下,此处注意六个None是互相独立的。}
         \PY{n+nb}{print}\PY{p}{(}\PY{n}{lists3}\PY{p}{)}
         \PY{n}{lists3}\PY{p}{[}\PY{l+m+mi}{0}\PY{p}{]}\PY{o}{.}\PY{n}{append}\PY{p}{(}\PY{l+m+mi}{2}\PY{p}{)}
         \PY{n}{lists3}\PY{p}{[}\PY{l+m+mi}{1}\PY{p}{]}\PY{o}{.}\PY{n}{append}\PY{p}{(}\PY{l+m+mi}{3}\PY{p}{)}
         \PY{n}{lists3}\PY{p}{[}\PY{l+m+mi}{2}\PY{p}{]}\PY{o}{.}\PY{n}{append}\PY{p}{(}\PY{l+m+mi}{5}\PY{p}{)}
         \PY{n}{lists3} \PY{c+c1}{\PYZsh{} append之后结果如下}
\end{Verbatim}


    \begin{Verbatim}[commandchars=\\\{\}]
[[None, None], [None, None], [None, None]]

    \end{Verbatim}

\begin{Verbatim}[commandchars=\\\{\}]
{\color{outcolor}Out[{\color{outcolor}18}]:} [[None, None, 2], [None, None, 3], [None, None, 5]]
\end{Verbatim}
            
    \begin{Verbatim}[commandchars=\\\{\}]
{\color{incolor}In [{\color{incolor}19}]:} \PY{n}{lists3}\PY{p}{[}\PY{l+m+mi}{0}\PY{p}{]}\PY{p}{[}\PY{l+m+mi}{0}\PY{p}{]} \PY{o}{=} \PY{l+m+mi}{1}    \PY{c+c1}{\PYZsh{} 再次证明六个None之间是互相独立的。}
         \PY{n}{lists3}
\end{Verbatim}


\begin{Verbatim}[commandchars=\\\{\}]
{\color{outcolor}Out[{\color{outcolor}19}]:} [[1, None, 2], [None, None, 3], [None, None, 5]]
\end{Verbatim}
            
    \begin{Verbatim}[commandchars=\\\{\}]
{\color{incolor}In [{\color{incolor}20}]:} \PY{n}{list4} \PY{o}{=} \PY{p}{[}\PY{p}{[}\PY{k+kc}{None}\PY{p}{]} \PY{o}{*} \PY{l+m+mi}{2}\PY{p}{]} \PY{o}{*} \PY{l+m+mi}{3} \PY{c+c1}{\PYZsh{} 此例子跟上边做对比, 互相之间还是关联的。但上边例子中 for loop之前的乘2并没有关联!!!!!}
         \PY{n+nb}{print}\PY{p}{(}\PY{n}{list4}\PY{p}{)}
         \PY{n}{list4}\PY{p}{[}\PY{l+m+mi}{0}\PY{p}{]}\PY{p}{[}\PY{l+m+mi}{0}\PY{p}{]} \PY{o}{=} \PY{l+m+mi}{1}
         \PY{n+nb}{print}\PY{p}{(}\PY{n}{list4}\PY{p}{)}
         \PY{n}{list4}\PY{p}{[}\PY{l+m+mi}{0}\PY{p}{]}\PY{o}{.}\PY{n}{append}\PY{p}{(}\PY{l+m+mi}{2}\PY{p}{)}
         \PY{n+nb}{print}\PY{p}{(}\PY{n}{list4}\PY{p}{)}
\end{Verbatim}


    \begin{Verbatim}[commandchars=\\\{\}]
[[None, None], [None, None], [None, None]]
[[1, None], [1, None], [1, None]]
[[1, None, 2], [1, None, 2], [1, None, 2]]

    \end{Verbatim}

    \begin{Verbatim}[commandchars=\\\{\}]
{\color{incolor}In [{\color{incolor}21}]:} \PY{c+c1}{\PYZsh{} items are independent from each other, correct way to initialize a two\PYZhy{}dimentional list: (because [None] * w multiplies a single value, not a list.?????)}
         \PY{n}{w}\PY{p}{,} \PY{n}{h} \PY{o}{=} \PY{l+m+mi}{2}\PY{p}{,} \PY{l+m+mi}{3}
         \PY{n}{A} \PY{o}{=} \PY{p}{[}\PY{p}{[}\PY{k+kc}{None}\PY{p}{]} \PY{o}{*} \PY{n}{w} \PY{k}{for} \PY{n}{i} \PY{o+ow}{in} \PY{n+nb}{range}\PY{p}{(}\PY{n}{h}\PY{p}{)}\PY{p}{]}
         \PY{n}{A}\PY{p}{[}\PY{l+m+mi}{0}\PY{p}{]}\PY{p}{[}\PY{l+m+mi}{0}\PY{p}{]} \PY{o}{=} \PY{l+m+mi}{1}
         \PY{n}{A}
\end{Verbatim}


\begin{Verbatim}[commandchars=\\\{\}]
{\color{outcolor}Out[{\color{outcolor}21}]:} [[1, None], [None, None], [None, None]]
\end{Verbatim}
            
    \begin{enumerate}
\def\labelenumi{\arabic{enumi}.}
\setcounter{enumi}{2}
\tightlist
\item
  If \emph{i} or \emph{j} is negative, the index is relative to the end
  of sequence s: \textbf{len(s) + i} or \textbf{len(s) + j} is
  substituted. But note that -0 is still 0.
\end{enumerate}

    \begin{enumerate}
\def\labelenumi{\arabic{enumi}.}
\setcounter{enumi}{3}
\tightlist
\item
  \textbf{i} is inclusive, \textbf{j} is exclusive. If \textbf{i} or
  \textbf{j} is greater than \textbf{len(s)}, use \textbf{len(s)}. If
  \textbf{i} is omitted or \textbf{None}, use \textbf{0}. If \textbf{j}
  is omitted or \textbf{None}, use \textbf{len(s)}. If \textbf{i} is
  greater than or equal to \textbf{j}, the slice is empty.
\end{enumerate}

    \begin{enumerate}
\def\labelenumi{\arabic{enumi}.}
\setcounter{enumi}{4}
\tightlist
\item
  s{[}i:j:k{]}, the indices are \textbf{i}, \textbf{i+k}, \textbf{i+2k},
  and so on, stopping when \textbf{j} is reached (but
  exclusive).还有解释,不写了,去网上看。。。
\end{enumerate}

    \begin{enumerate}
\def\labelenumi{\arabic{enumi}.}
\setcounter{enumi}{5}
\tightlist
\item
  Concatenating immutable swquences always results in a new object,
  which means runtime cost is much higher!!! 4 ways to save the runtime.
  看网站去。
\end{enumerate}

    \begin{enumerate}
\def\labelenumi{\arabic{enumi}.}
\setcounter{enumi}{6}
\item
\end{enumerate}

    \begin{enumerate}
\def\labelenumi{\arabic{enumi}.}
\setcounter{enumi}{7}
\item
\end{enumerate}

    \subsection{String}\label{string}

    \subsubsection{string basic}\label{string-basic}

    If you don't want characters prefaced by ~to be interpreted as special
characters, you can use raw strings by adding an r before the first
quote:

    \begin{Verbatim}[commandchars=\\\{\}]
{\color{incolor}In [{\color{incolor}22}]:} \PY{n+nb}{print}\PY{p}{(}\PY{l+s+sa}{r}\PY{l+s+s1}{\PYZsq{}}\PY{l+s+s1}{C:}\PY{l+s+s1}{\PYZbs{}}\PY{l+s+s1}{some}\PY{l+s+s1}{\PYZbs{}}\PY{l+s+s1}{name}\PY{l+s+s1}{\PYZsq{}}\PY{p}{)}  \PY{c+c1}{\PYZsh{} note the r before the quote}
\end{Verbatim}


    \begin{Verbatim}[commandchars=\\\{\}]
C:\textbackslash{}some\textbackslash{}name

    \end{Verbatim}

    Two or more string literals (i.e. the ones enclosed between quotes) next
to each other are automatically concatenated.

    \begin{Verbatim}[commandchars=\\\{\}]
{\color{incolor}In [{\color{incolor}23}]:} \PY{l+s+s1}{\PYZsq{}}\PY{l+s+s1}{Py}\PY{l+s+s1}{\PYZsq{}} \PY{l+s+s1}{\PYZsq{}}\PY{l+s+s1}{thon}\PY{l+s+s1}{\PYZsq{}}
\end{Verbatim}


\begin{Verbatim}[commandchars=\\\{\}]
{\color{outcolor}Out[{\color{outcolor}23}]:} 'Python'
\end{Verbatim}
            
    \begin{Verbatim}[commandchars=\\\{\}]
{\color{incolor}In [{\color{incolor}24}]:} \PY{n}{text} \PY{o}{=} \PY{p}{(}\PY{l+s+s1}{\PYZsq{}}\PY{l+s+s1}{Put several strings within parentheses }\PY{l+s+s1}{\PYZsq{}} \PY{l+s+s1}{\PYZsq{}}\PY{l+s+s1}{to have them joined together.}\PY{l+s+s1}{\PYZsq{}}\PY{p}{)}
         \PY{n}{text}
\end{Verbatim}


\begin{Verbatim}[commandchars=\\\{\}]
{\color{outcolor}Out[{\color{outcolor}24}]:} 'Put several strings within parentheses to have them joined together.'
\end{Verbatim}
            
    \begin{Verbatim}[commandchars=\\\{\}]
{\color{incolor}In [{\color{incolor}25}]:} \PY{n+nb}{type}\PY{p}{(}\PY{n}{text}\PY{p}{)}
\end{Verbatim}


\begin{Verbatim}[commandchars=\\\{\}]
{\color{outcolor}Out[{\color{outcolor}25}]:} str
\end{Verbatim}
            
    Python strings cannot be changed --- they are immutable.

    \subsubsection{String method}\label{string-method}

    \begin{Verbatim}[commandchars=\\\{\}]
{\color{incolor}In [{\color{incolor}26}]:} \PY{c+c1}{\PYZsh{} str.strip([char])}
         \PY{l+s+sd}{\PYZsq{}\PYZsq{}\PYZsq{}The OUTERMOST leading and trailing chars argument values are stripped from the string. \PYZbs{}}
         \PY{l+s+sd}{Unil reaching a string character that is not contained in the set of char, [char].\PYZbs{}}
         \PY{l+s+sd}{The chars argument is not a prefix or suffix; rather, every single character \PYZbs{}}
         \PY{l+s+sd}{in it is stripped. The default value is white\PYZhy{}space.\PYZbs{}}
         \PY{l+s+sd}{\PYZsq{}\PYZsq{}\PYZsq{}}
         \PY{c+c1}{\PYZsh{} example:}
\end{Verbatim}


\begin{Verbatim}[commandchars=\\\{\}]
{\color{outcolor}Out[{\color{outcolor}26}]:} 'The OUTERMOST leading and trailing chars argument values are stripped from the string. Unil reaching a string character that is not contained in the set of char, [char].The chars argument is not a prefix or suffix; rather, every single character in it is stripped. The default value is white-space.'
\end{Verbatim}
            
    \begin{Verbatim}[commandchars=\\\{\}]
{\color{incolor}In [{\color{incolor}27}]:} \PY{l+s+s1}{\PYZsq{}}\PY{l+s+s1}{     spacious     }\PY{l+s+s1}{\PYZsq{}}\PY{o}{.}\PY{n}{strip}\PY{p}{(}\PY{p}{)}
\end{Verbatim}


\begin{Verbatim}[commandchars=\\\{\}]
{\color{outcolor}Out[{\color{outcolor}27}]:} 'spacious'
\end{Verbatim}
            
    \begin{Verbatim}[commandchars=\\\{\}]
{\color{incolor}In [{\color{incolor}28}]:} \PY{l+s+s1}{\PYZsq{}}\PY{l+s+s1}{www.example.com}\PY{l+s+s1}{\PYZsq{}}\PY{o}{.}\PY{n}{strip}\PY{p}{(}\PY{l+s+s1}{\PYZsq{}}\PY{l+s+s1}{comwa}\PY{l+s+s1}{\PYZsq{}}\PY{p}{)} \PY{c+c1}{\PYZsh{} pay attention here, a is not stripped, since it is in the inner side.}
\end{Verbatim}


\begin{Verbatim}[commandchars=\\\{\}]
{\color{outcolor}Out[{\color{outcolor}28}]:} '.example.'
\end{Verbatim}
            
    \subsection{Lists}\label{lists}

    Lists are mutable, able to contain different types of item.

    Nest lists:

    \begin{Verbatim}[commandchars=\\\{\}]
{\color{incolor}In [{\color{incolor}29}]:} \PY{n}{a} \PY{o}{=} \PY{p}{[}\PY{l+s+s1}{\PYZsq{}}\PY{l+s+s1}{a}\PY{l+s+s1}{\PYZsq{}}\PY{p}{,} \PY{l+s+s1}{\PYZsq{}}\PY{l+s+s1}{b}\PY{l+s+s1}{\PYZsq{}}\PY{p}{,} \PY{l+s+s1}{\PYZsq{}}\PY{l+s+s1}{c}\PY{l+s+s1}{\PYZsq{}}\PY{p}{]}
         \PY{n}{b} \PY{o}{=} \PY{p}{[}\PY{l+m+mi}{1}\PY{p}{,} \PY{l+m+mi}{2}\PY{p}{,} \PY{l+m+mi}{3}\PY{p}{]}
         \PY{n}{x} \PY{o}{=} \PY{p}{[}\PY{n}{a}\PY{p}{,} \PY{n}{b}\PY{p}{]}
         \PY{n}{x}\PY{p}{[}\PY{l+m+mi}{0}\PY{p}{]}
\end{Verbatim}


\begin{Verbatim}[commandchars=\\\{\}]
{\color{outcolor}Out[{\color{outcolor}29}]:} ['a', 'b', 'c']
\end{Verbatim}
            
    \begin{Verbatim}[commandchars=\\\{\}]
{\color{incolor}In [{\color{incolor}30}]:} \PY{n}{x}\PY{p}{[}\PY{l+m+mi}{0}\PY{p}{]}\PY{p}{[}\PY{l+m+mi}{1}\PY{p}{]}
\end{Verbatim}


\begin{Verbatim}[commandchars=\\\{\}]
{\color{outcolor}Out[{\color{outcolor}30}]:} 'b'
\end{Verbatim}
            
    生成连续数列:

    \begin{Verbatim}[commandchars=\\\{\}]
{\color{incolor}In [{\color{incolor}31}]:} \PY{n+nb}{list}\PY{p}{(}\PY{n+nb}{range}\PY{p}{(}\PY{l+m+mi}{3}\PY{p}{,} \PY{l+m+mi}{9}\PY{p}{)}\PY{p}{)}
\end{Verbatim}


\begin{Verbatim}[commandchars=\\\{\}]
{\color{outcolor}Out[{\color{outcolor}31}]:} [3, 4, 5, 6, 7, 8]
\end{Verbatim}
            
    \subsection{Control Flow Tools}\label{control-flow-tools}

    \subsubsection{if-elif-elif-else}\label{if-elif-elif-else}

    \subsubsection{For Loops}\label{for-loops}

    \begin{itemize}
\tightlist
\item
  Python's for statement iterates over the items of any sequence (a list
  or a string), in the order that they appear in the sequence.
\end{itemize}

    \begin{Verbatim}[commandchars=\\\{\}]
{\color{incolor}In [{\color{incolor}32}]:} \PY{n}{words} \PY{o}{=} \PY{p}{[}\PY{l+s+s1}{\PYZsq{}}\PY{l+s+s1}{cat}\PY{l+s+s1}{\PYZsq{}}\PY{p}{,} \PY{l+s+s1}{\PYZsq{}}\PY{l+s+s1}{window}\PY{l+s+s1}{\PYZsq{}}\PY{p}{,} \PY{l+s+s1}{\PYZsq{}}\PY{l+s+s1}{defenestrate}\PY{l+s+s1}{\PYZsq{}}\PY{p}{]}
         \PY{k}{for} \PY{n}{w} \PY{o+ow}{in} \PY{n}{words}\PY{p}{:}
             \PY{n+nb}{print}\PY{p}{(}\PY{n}{w}\PY{p}{,} \PY{n+nb}{len}\PY{p}{(}\PY{n}{w}\PY{p}{)}\PY{p}{)}
\end{Verbatim}


    \begin{Verbatim}[commandchars=\\\{\}]
cat 3
window 6
defenestrate 12

    \end{Verbatim}

    \begin{itemize}
\tightlist
\item
  If you need to modify the sequence you are iterating over while inside
  the loop (for example to duplicate selected items), it is recommended
  that you first make a copy. Iterating over a sequence does not
  implicitly make a copy. The slice notation makes this especially
  convenient:
  作为循环判断条件的list:words在循环过程中被改变了,所以使用words{[}:{]}的方式在循环开始时复制一个words,
  以此初始状态完成循环,循环过程中对words所做修改不再影响for循环的判定。也可以使用words.copy()
\end{itemize}

    \begin{Verbatim}[commandchars=\\\{\}]
{\color{incolor}In [{\color{incolor}33}]:} \PY{n}{words} \PY{o}{=} \PY{p}{[}\PY{l+s+s1}{\PYZsq{}}\PY{l+s+s1}{cat}\PY{l+s+s1}{\PYZsq{}}\PY{p}{,} \PY{l+s+s1}{\PYZsq{}}\PY{l+s+s1}{window}\PY{l+s+s1}{\PYZsq{}}\PY{p}{,} \PY{l+s+s1}{\PYZsq{}}\PY{l+s+s1}{defenestrate}\PY{l+s+s1}{\PYZsq{}}\PY{p}{]}
         \PY{k}{for} \PY{n}{w} \PY{o+ow}{in} \PY{n}{words}\PY{p}{[}\PY{p}{:}\PY{p}{]}\PY{p}{:}    \PY{c+c1}{\PYZsh{}此处用的words[:]}
             \PY{k}{if} \PY{n+nb}{len}\PY{p}{(}\PY{n}{w}\PY{p}{)} \PY{o}{\PYZgt{}} \PY{l+m+mi}{6}\PY{p}{:}
                 \PY{n}{words}\PY{o}{.}\PY{n}{insert}\PY{p}{(}\PY{l+m+mi}{0}\PY{p}{,} \PY{n}{w}\PY{p}{)} \PY{c+c1}{\PYZsh{} insert w to the 0 position in words}
         \PY{n}{words}
\end{Verbatim}


\begin{Verbatim}[commandchars=\\\{\}]
{\color{outcolor}Out[{\color{outcolor}33}]:} ['defenestrate', 'cat', 'window', 'defenestrate']
\end{Verbatim}
            
    \begin{Verbatim}[commandchars=\\\{\}]
{\color{incolor}In [{\color{incolor}34}]:} \PY{n}{words} \PY{o}{=} \PY{p}{[}\PY{l+s+s1}{\PYZsq{}}\PY{l+s+s1}{cat}\PY{l+s+s1}{\PYZsq{}}\PY{p}{,} \PY{l+s+s1}{\PYZsq{}}\PY{l+s+s1}{window}\PY{l+s+s1}{\PYZsq{}}\PY{p}{,} \PY{l+s+s1}{\PYZsq{}}\PY{l+s+s1}{defenestrate}\PY{l+s+s1}{\PYZsq{}}\PY{p}{]}
         \PY{k}{for} \PY{n}{w} \PY{o+ow}{in} \PY{n}{words}\PY{o}{.}\PY{n}{copy}\PY{p}{(}\PY{p}{)}\PY{p}{:}    \PY{n}{此处用的words}\PY{o}{.}\PY{n}{copy}\PY{p}{(}\PY{p}{)}
             \PY{k}{if} \PY{n+nb}{len}\PY{p}{(}\PY{n}{w}\PY{p}{)} \PY{o}{\PYZgt{}} \PY{l+m+mi}{6}\PY{p}{:}
                 \PY{n}{words}\PY{o}{.}\PY{n}{insert}\PY{p}{(}\PY{l+m+mi}{0}\PY{p}{,} \PY{n}{w}\PY{p}{)} \PY{c+c1}{\PYZsh{} insert w to the 0 position in words}
         \PY{n}{words}
\end{Verbatim}


    \begin{Verbatim}[commandchars=\\\{\}]

          File "<ipython-input-34-804612cdad38>", line 3
        if len(w) > 6:
        \^{}
    IndentationError: unexpected indent


    \end{Verbatim}

    \subsubsection{range() function}\label{range-function}

    \begin{Verbatim}[commandchars=\\\{\}]
{\color{incolor}In [{\color{incolor} }]:} \PY{k}{for} \PY{n}{i} \PY{o+ow}{in} \PY{n+nb}{range}\PY{p}{(}\PY{l+m+mi}{5}\PY{p}{)}\PY{p}{:}
            \PY{n+nb}{print}\PY{p}{(}\PY{n}{i}\PY{p}{,} \PY{n}{end}\PY{o}{=}\PY{l+s+s1}{\PYZsq{}}\PY{l+s+s1}{, }\PY{l+s+s1}{\PYZsq{}}\PY{p}{)}
\end{Verbatim}


    \begin{Verbatim}[commandchars=\\\{\}]
{\color{incolor}In [{\color{incolor} }]:} \PY{k}{for} \PY{n}{i} \PY{o+ow}{in} \PY{n+nb}{range}\PY{p}{(}\PY{l+m+mi}{3}\PY{p}{,} \PY{l+m+mi}{5}\PY{p}{)}\PY{p}{:}
            \PY{n+nb}{print}\PY{p}{(}\PY{n}{i}\PY{p}{,} \PY{n}{end}\PY{o}{=}\PY{l+s+s1}{\PYZsq{}}\PY{l+s+s1}{, }\PY{l+s+s1}{\PYZsq{}}\PY{p}{)}
\end{Verbatim}


    \begin{Verbatim}[commandchars=\\\{\}]
{\color{incolor}In [{\color{incolor} }]:} \PY{k}{for} \PY{n}{i} \PY{o+ow}{in} \PY{n+nb}{range}\PY{p}{(}\PY{l+m+mi}{1}\PY{p}{,} \PY{l+m+mi}{20}\PY{p}{,} \PY{l+m+mi}{3}\PY{p}{)}\PY{p}{:}
            \PY{n+nb}{print}\PY{p}{(}\PY{n}{i}\PY{p}{,} \PY{n}{end}\PY{o}{=}\PY{l+s+s1}{\PYZsq{}}\PY{l+s+s1}{, }\PY{l+s+s1}{\PYZsq{}}\PY{p}{)}
\end{Verbatim}


    \subsubsection{break and continue Statements, and else Clauses on
Loops}\label{break-and-continue-statements-and-else-clauses-on-loops}

    \begin{enumerate}
\def\labelenumi{\arabic{enumi}.}
\tightlist
\item
  break, breaks out of the innermost enclosing \emph{for} or
  \emph{while} loop
\end{enumerate}

    \begin{enumerate}
\def\labelenumi{\arabic{enumi}.}
\setcounter{enumi}{1}
\tightlist
\item
  else clause: is executed when the loop terminates through exhaustion
  of the list (with \emph{for} loop) or when the condition becomes false
  (with \emph{while} loop), but not when the loop is terminated by a
  \emph{break} statement,下面举一个找质数(prime number)的例子:
\end{enumerate}

    \begin{Verbatim}[commandchars=\\\{\}]
{\color{incolor}In [{\color{incolor} }]:} \PY{n}{m} \PY{o}{=} \PY{l+m+mi}{10} \PY{c+c1}{\PYZsh{} searching for all the prime numbers from zero to m.}
        \PY{k}{for} \PY{n}{n} \PY{o+ow}{in} \PY{n+nb}{range}\PY{p}{(}\PY{l+m+mi}{2}\PY{p}{,} \PY{n}{m}\PY{o}{+}\PY{l+m+mi}{1}\PY{p}{)}\PY{p}{:}
            \PY{n+nb}{print}\PY{p}{(}\PY{n}{f}\PY{l+s+s1}{\PYZsq{}}\PY{l+s+s1}{n is }\PY{l+s+si}{\PYZob{}n\PYZcb{}}\PY{l+s+s1}{\PYZsq{}}\PY{p}{)}
            \PY{k}{for} \PY{n}{x} \PY{o+ow}{in} \PY{n+nb}{range}\PY{p}{(}\PY{l+m+mi}{2}\PY{p}{,} \PY{n}{n}\PY{p}{)}\PY{p}{:}
                \PY{n+nb}{print}\PY{p}{(}\PY{n}{f}\PY{l+s+s1}{\PYZsq{}}\PY{l+s+s1}{x is }\PY{l+s+si}{\PYZob{}x\PYZcb{}}\PY{l+s+s1}{\PYZsq{}}\PY{p}{)}
                \PY{k}{if} \PY{n}{n} \PY{o}{\PYZpc{}} \PY{n}{x} \PY{o}{==} \PY{l+m+mi}{0}\PY{p}{:}
                    \PY{n+nb}{print}\PY{p}{(}\PY{n}{n}\PY{p}{,} \PY{l+s+s1}{\PYZsq{}}\PY{l+s+s1}{equals}\PY{l+s+s1}{\PYZsq{}}\PY{p}{,} \PY{n}{x}\PY{p}{,} \PY{l+s+s1}{\PYZsq{}}\PY{l+s+s1}{*}\PY{l+s+s1}{\PYZsq{}}\PY{p}{,} \PY{n}{n}\PY{o}{/}\PY{o}{/}\PY{n}{x}\PY{p}{)}
                    \PY{k}{break}
            \PY{k}{else}\PY{p}{:}
                \PY{c+c1}{\PYZsh{} loop fell through without finding a factor}
                \PY{n+nb}{print}\PY{p}{(}\PY{n}{n}\PY{p}{,} \PY{l+s+s1}{\PYZsq{}}\PY{l+s+s1}{is a prime number.}\PY{l+s+s1}{\PYZsq{}}\PY{p}{)}
        \PY{l+s+sd}{\PYZsq{}\PYZsq{}\PYZsq{}\PYZbs{}}
        \PY{l+s+sd}{In the above example, line3 and line7 are a pair }
        \PY{l+s+sd}{(if for loop start from line3 finishes without a break,}
        \PY{l+s+sd}{the else clause will be executed).}
        
        \PY{l+s+sd}{注意n等于2时,x in range(2, 2)循环直接结束,进入else,2直接判定为质数。\PYZbs{}}
        \PY{l+s+sd}{\PYZsq{}\PYZsq{}\PYZsq{}}
\end{Verbatim}


    \begin{enumerate}
\def\labelenumi{\arabic{enumi}.}
\setcounter{enumi}{2}
\tightlist
\item
  The \emph{continue} statement, contnues with the next iteration of the
  loop.
\end{enumerate}

    \begin{Verbatim}[commandchars=\\\{\}]
{\color{incolor}In [{\color{incolor} }]:} \PY{n}{m} \PY{o}{=} \PY{l+m+mi}{10}
        \PY{k}{for} \PY{n}{num} \PY{o+ow}{in} \PY{n+nb}{range}\PY{p}{(}\PY{l+m+mi}{2}\PY{p}{,} \PY{n}{m}\PY{p}{)}\PY{p}{:}
            \PY{k}{if} \PY{n}{num} \PY{o}{\PYZpc{}} \PY{l+m+mi}{2} \PY{o}{==} \PY{l+m+mi}{0}\PY{p}{:}
                \PY{n+nb}{print}\PY{p}{(}\PY{l+s+s2}{\PYZdq{}}\PY{l+s+s2}{Found an even number}\PY{l+s+s2}{\PYZdq{}}\PY{p}{,} \PY{n}{num}\PY{p}{)}
                \PY{k}{continue}
            \PY{n+nb}{print}\PY{p}{(}\PY{l+s+s2}{\PYZdq{}}\PY{l+s+s2}{Found an odd number}\PY{l+s+s2}{\PYZdq{}}\PY{p}{,} \PY{n}{num}\PY{p}{)}
                      
\end{Verbatim}


    \subsubsection{pass Statements}\label{pass-statements}

    pass does nothing. It can be used when a statement is required
syntactically but the program requires no action:
定义了函数,loop等,内部需要有代码,否则会报错。又啥都不想写的时候,直接一个pass搞定。

    \begin{Verbatim}[commandchars=\\\{\}]
{\color{incolor}In [{\color{incolor} }]:} \PY{k}{while} \PY{k+kc}{True}\PY{p}{:}
            \PY{k}{pass} \PY{c+c1}{\PYZsh{} Busy\PYZhy{}wait for keyboard interrupt}
\end{Verbatim}


    \begin{Verbatim}[commandchars=\\\{\}]
{\color{incolor}In [{\color{incolor} }]:} \PY{k}{class} \PY{n+nc}{MyEnptyClass}\PY{p}{:}
            \PY{k}{pass}
\end{Verbatim}


    \begin{Verbatim}[commandchars=\\\{\}]
{\color{incolor}In [{\color{incolor} }]:} \PY{k}{def} \PY{n+nf}{initlog}\PY{p}{(}\PY{o}{*}\PY{n}{args}\PY{p}{)}\PY{p}{:}
            \PY{k}{pass} \PY{c+c1}{\PYZsh{} Remember to implement this!}
\end{Verbatim}


    \subsubsection{Defining Functions}\label{defining-functions}

    \paragraph{Function return:}\label{function-return}

\begin{itemize}
\tightlist
\item
  return without an expression argument returns \textbf{None}
\item
  Falling off the end of a function also returns \textbf{None}
\end{itemize}

    \subsubsection{More on Defining
Functions}\label{more-on-defining-functions}

    \paragraph{Default Argument Values:}\label{default-argument-values}

\begin{itemize}
\tightlist
\item
  Specifying a default value for an argument, makes the argument
  \textbf{optional} when calling the function.
\item
  The default value is only evaluated once, when the function is
  defined. But, if the optional argument is a mutable object, and the
  function is called multiple times without specifying this argument,
  this value could be accumulated in the following example:
\end{itemize}

    \begin{Verbatim}[commandchars=\\\{\}]
{\color{incolor}In [{\color{incolor} }]:} \PY{k}{def} \PY{n+nf}{f}\PY{p}{(}\PY{n}{a}\PY{p}{,} \PY{n}{L}\PY{o}{=}\PY{p}{[}\PY{p}{]}\PY{p}{)}\PY{p}{:}
            \PY{n}{L}\PY{o}{.}\PY{n}{append}\PY{p}{(}\PY{n}{a}\PY{p}{)}
            \PY{k}{return} \PY{n}{L}
        
        \PY{n+nb}{print}\PY{p}{(}\PY{n}{f}\PY{p}{(}\PY{l+m+mi}{1}\PY{p}{)}\PY{p}{)}
        \PY{n+nb}{print}\PY{p}{(}\PY{n}{f}\PY{p}{(}\PY{l+m+mi}{2}\PY{p}{)}\PY{p}{)}
        \PY{n+nb}{print}\PY{p}{(}\PY{n}{f}\PY{p}{(}\PY{l+m+mi}{3}\PY{p}{)}\PY{p}{)}
\end{Verbatim}


    \paragraph{Keyword Arguments}\label{keyword-arguments}

    \begin{enumerate}
\def\labelenumi{\arabic{enumi}.}
\tightlist
\item
  Passing arguments to a function, can use \textbf{positional} or
  \textbf{keyword} arguments. More detailed are in
  \href{docs.python.org/3/tutorial/controlflow.html}{Python Docs}
\end{enumerate}

    \begin{enumerate}
\def\labelenumi{\arabic{enumi}.}
\setcounter{enumi}{1}
\tightlist
\item
  Two more options of passing arguments (these are \textbf{arbitrary
  number} of arguments):

  \begin{itemize}
  \tightlist
  \item
    *name: It recieves a tuple containing the \textbf{positional}
    arguments beyond the formal parameter list.
  \item
    **name: It recevices a dictionary containing all \textbf{keyword}
    arguments except for those corresponding to a formal parameter.
  \item
    *name should always be before **name arguments
  \end{itemize}
\end{enumerate}

    \begin{Verbatim}[commandchars=\\\{\}]
{\color{incolor}In [{\color{incolor} }]:} \PY{k}{def} \PY{n+nf}{cheeseshop}\PY{p}{(}\PY{n}{kind}\PY{p}{,} \PY{o}{*}\PY{n}{arguments}\PY{p}{,} \PY{o}{*}\PY{o}{*}\PY{n}{keywords}\PY{p}{)}\PY{p}{:}
            \PY{n+nb}{print}\PY{p}{(}\PY{l+s+s2}{\PYZdq{}}\PY{l+s+s2}{\PYZhy{}\PYZhy{} Do you have any}\PY{l+s+s2}{\PYZdq{}}\PY{p}{,} \PY{n}{kind}\PY{p}{,} \PY{l+s+s2}{\PYZdq{}}\PY{l+s+s2}{?}\PY{l+s+s2}{\PYZdq{}}\PY{p}{)}
            \PY{n+nb}{print}\PY{p}{(}\PY{n}{f}\PY{l+s+s2}{\PYZdq{}}\PY{l+s+s2}{\PYZhy{}\PYZhy{} I}\PY{l+s+s2}{\PYZsq{}}\PY{l+s+s2}{m sorry, we}\PY{l+s+s2}{\PYZsq{}}\PY{l+s+s2}{re all out of }\PY{l+s+si}{\PYZob{}kind\PYZcb{}}\PY{l+s+s2}{.}\PY{l+s+s2}{\PYZdq{}}\PY{p}{)}
            \PY{k}{for} \PY{n}{arg} \PY{o+ow}{in} \PY{n}{arguments}\PY{p}{:}
                \PY{n+nb}{print}\PY{p}{(}\PY{n}{arg}\PY{p}{)}
            \PY{n+nb}{print}\PY{p}{(}\PY{l+s+s1}{\PYZsq{}}\PY{l+s+s1}{*}\PY{l+s+s1}{\PYZsq{}} \PY{o}{*} \PY{l+m+mi}{40}\PY{p}{)}
            \PY{k}{for} \PY{n}{key} \PY{o+ow}{in} \PY{n}{keywords}\PY{p}{:}
                \PY{n+nb}{print}\PY{p}{(}\PY{n}{key}\PY{p}{,} \PY{l+s+s2}{\PYZdq{}}\PY{l+s+s2}{:}\PY{l+s+s2}{\PYZdq{}}\PY{p}{,} \PY{n}{keywords}\PY{p}{[}\PY{n}{key}\PY{p}{]}\PY{p}{)}
\end{Verbatim}


    The above could be called like this:

    \begin{Verbatim}[commandchars=\\\{\}]
{\color{incolor}In [{\color{incolor} }]:} \PY{n}{cheeseshop}\PY{p}{(}\PY{l+s+s2}{\PYZdq{}}\PY{l+s+s2}{Limburger}\PY{l+s+s2}{\PYZdq{}}\PY{p}{,} \PY{l+s+s2}{\PYZdq{}}\PY{l+s+s2}{It}\PY{l+s+s2}{\PYZsq{}}\PY{l+s+s2}{s very runny, sir.}\PY{l+s+s2}{\PYZdq{}}\PY{p}{,}
                  \PY{l+s+s2}{\PYZdq{}}\PY{l+s+s2}{It}\PY{l+s+s2}{\PYZsq{}}\PY{l+s+s2}{s really very, VERY runny, sir.}\PY{l+s+s2}{\PYZdq{}}\PY{p}{,}
                  \PY{n}{shopkeeper}\PY{o}{=}\PY{l+s+s2}{\PYZdq{}}\PY{l+s+s2}{Michael Palin}\PY{l+s+s2}{\PYZdq{}}\PY{p}{,}
                  \PY{n}{client}\PY{o}{=}\PY{l+s+s2}{\PYZdq{}}\PY{l+s+s2}{John Cleese}\PY{l+s+s2}{\PYZdq{}}\PY{p}{,}
                  \PY{n}{sketch}\PY{o}{=}\PY{l+s+s2}{\PYZdq{}}\PY{l+s+s2}{Cheese Shop Sketch}\PY{l+s+s2}{\PYZdq{}}\PY{p}{)}
\end{Verbatim}


    \paragraph{Unpacking Argument Lists}\label{unpacking-argument-lists}

    \begin{Verbatim}[commandchars=\\\{\}]
{\color{incolor}In [{\color{incolor} }]:} \PY{n+nb}{list}\PY{p}{(}\PY{n+nb}{range}\PY{p}{(}\PY{l+m+mi}{3}\PY{p}{,} \PY{l+m+mi}{6}\PY{p}{)}\PY{p}{)}    \PY{c+c1}{\PYZsh{} Normal call with separate arguments}
\end{Verbatim}


    \begin{Verbatim}[commandchars=\\\{\}]
{\color{incolor}In [{\color{incolor} }]:} \PY{n}{args} \PY{o}{=} \PY{p}{[}\PY{l+m+mi}{3}\PY{p}{,} \PY{l+m+mi}{6}\PY{p}{]}
        \PY{n+nb}{list}\PY{p}{(}\PY{n+nb}{range}\PY{p}{(}\PY{o}{*}\PY{n}{args}\PY{p}{)}\PY{p}{)} \PY{c+c1}{\PYZsh{} call with arguments unpacked from a list}
\end{Verbatim}


    \paragraph{Lambda Expressions}\label{lambda-expressions}

    Small anonymous functions can be created with the \textbf{lambda}
keyword.这个例子仔细看,make\_incrementor是个关于n的函数,f是个关于x的函数。

接受lambda function为参数的变量,作为lambdafunction的变量。

    \begin{Verbatim}[commandchars=\\\{\}]
{\color{incolor}In [{\color{incolor} }]:} \PY{k}{def} \PY{n+nf}{make\PYZus{}incrementor}\PY{p}{(}\PY{n}{n}\PY{p}{)}\PY{p}{:}
            \PY{k}{return} \PY{k}{lambda} \PY{n}{x}\PY{p}{:} \PY{n}{x} \PY{o}{+} \PY{n}{n}
        
        \PY{n}{f} \PY{o}{=} \PY{n}{make\PYZus{}incrementor}\PY{p}{(}\PY{l+m+mi}{42}\PY{p}{)}
        \PY{n}{f}\PY{p}{(}\PY{l+m+mi}{1}\PY{p}{)}
\end{Verbatim}


    Another use is to pass a small function as an argument.

    \begin{Verbatim}[commandchars=\\\{\}]
{\color{incolor}In [{\color{incolor} }]:} \PY{n}{pairs} \PY{o}{=} \PY{p}{[}\PY{p}{(}\PY{l+m+mi}{1}\PY{p}{,} \PY{l+s+s1}{\PYZsq{}}\PY{l+s+s1}{one}\PY{l+s+s1}{\PYZsq{}}\PY{p}{)}\PY{p}{,} \PY{p}{(}\PY{l+m+mi}{2}\PY{p}{,} \PY{l+s+s1}{\PYZsq{}}\PY{l+s+s1}{two}\PY{l+s+s1}{\PYZsq{}}\PY{p}{)}\PY{p}{,} \PY{p}{(}\PY{l+m+mi}{3}\PY{p}{,} \PY{l+s+s1}{\PYZsq{}}\PY{l+s+s1}{three}\PY{l+s+s1}{\PYZsq{}}\PY{p}{)}\PY{p}{,} \PY{p}{(}\PY{l+m+mi}{4}\PY{p}{,} \PY{l+s+s1}{\PYZsq{}}\PY{l+s+s1}{four}\PY{l+s+s1}{\PYZsq{}}\PY{p}{)}\PY{p}{]}
        \PY{n}{pairs}\PY{o}{.}\PY{n}{sort}\PY{p}{(}\PY{n}{key}\PY{o}{=}\PY{k}{lambda} \PY{n}{pair}\PY{p}{:} \PY{n}{pair}\PY{p}{[}\PY{l+m+mi}{1}\PY{p}{]}\PY{p}{)}    \PY{c+c1}{\PYZsh{}按里层list第二个元素排序 也就是 按照one two three four的首字母排序 }
        \PY{c+c1}{\PYZsh{} 下边list.sort()有进一步解释}
        \PY{n}{pairs}
\end{Verbatim}


    \paragraph{Function Annotations}\label{function-annotations}

    暂时用不到,用来说明自定义函数的参数类型,方便阅读代码的。

    \subsubsection{Coding Style}\label{coding-style}

    \begin{enumerate}
\def\labelenumi{\arabic{enumi}.}
\tightlist
\item
  Use 4-space indentation, and no \textbf{tabs}.
\end{enumerate}

    \begin{enumerate}
\def\labelenumi{\arabic{enumi}.}
\setcounter{enumi}{1}
\tightlist
\item
  Wrap lines so that they don't exceed 79 characters.
\end{enumerate}

    \begin{enumerate}
\def\labelenumi{\arabic{enumi}.}
\setcounter{enumi}{2}
\tightlist
\item
  Use blank lines to serarate functions and classes, and larger blocks
  of code inside functions.
\end{enumerate}

    \begin{enumerate}
\def\labelenumi{\arabic{enumi}.}
\setcounter{enumi}{3}
\tightlist
\item
  When possible, put comments on a line of their own.
\end{enumerate}

    \begin{enumerate}
\def\labelenumi{\arabic{enumi}.}
\setcounter{enumi}{4}
\tightlist
\item
  Use docstrings. ?????
\end{enumerate}

    \begin{enumerate}
\def\labelenumi{\arabic{enumi}.}
\setcounter{enumi}{5}
\item
  Use spaces around operators and after commas, but not directly inside
  bracketing constructs: (此处不太懂具体说的什么意思。)

  \begin{itemize}
  \tightlist
  \item
    a = f(1, 3) + g(3, 4)
  \end{itemize}
\end{enumerate}

    \begin{enumerate}
\def\labelenumi{\arabic{enumi}.}
\setcounter{enumi}{6}
\tightlist
\item
  Name your classes and functions consistently; the convention is to use
  \textbf{CamelCase} (每个单词首字母大写,单词间不空格) for classes and
  \textbf{lower\_case\_with\_underscores}(全部小写,单词间用下划线分隔)
  for functions and methods.
\end{enumerate}

    \begin{enumerate}
\def\labelenumi{\arabic{enumi}.}
\setcounter{enumi}{7}
\tightlist
\item
  Encodings: Python's default is UTF-8, or even plain ASCII works best
  in any case.
\end{enumerate}

    \subsection{Data Structures}\label{data-structures}

    \subsubsection{methods of list type}\label{methods-of-list-type}

    \begin{enumerate}
\def\labelenumi{\arabic{enumi}.}
\tightlist
\item
  list.append(x)
\end{enumerate}

Add an item to the end of the list. Equivalent to a{[}len(a):{]} =
{[}x{]} \# 此处x可以是含有多项的list,但此list被append时
是作为一个整体被加入的。

    \begin{enumerate}
\def\labelenumi{\arabic{enumi}.}
\setcounter{enumi}{1}
\tightlist
\item
  list.extend(iterable)
\end{enumerate}

Add an item to the end of the list. Equivalent to a{[}len(a):{]} =
{[}iterable{]}

此处注意 iterable
需要是一个list,或者其他类型,暂时还不知道别的什么可以的,以后看到了再加上哦。

    以上append 跟 extend 的区别在于
append的参数作为一个整体被加到list的最后;extend的参数必须是iterable的(看下边例子
出错的这个),拆分开分别加进list。

    \begin{Verbatim}[commandchars=\\\{\}]
{\color{incolor}In [{\color{incolor} }]:} \PY{n}{list\PYZus{}a} \PY{o}{=} \PY{p}{[}\PY{l+m+mi}{1}\PY{p}{,} \PY{l+m+mi}{2}\PY{p}{,} \PY{l+m+mi}{3}\PY{p}{,} \PY{l+m+mi}{4}\PY{p}{,} \PY{l+m+mi}{5}\PY{p}{]}
        \PY{n}{list\PYZus{}a}\PY{o}{.}\PY{n}{append}\PY{p}{(}\PY{l+m+mi}{6}\PY{p}{)}
        \PY{n}{list\PYZus{}a}
\end{Verbatim}


    \begin{Verbatim}[commandchars=\\\{\}]
{\color{incolor}In [{\color{incolor} }]:} \PY{n}{list\PYZus{}c} \PY{o}{=} \PY{p}{[}\PY{l+m+mi}{1}\PY{p}{,} \PY{l+m+mi}{2}\PY{p}{,} \PY{l+m+mi}{3}\PY{p}{,} \PY{l+m+mi}{4}\PY{p}{,} \PY{l+m+mi}{5}\PY{p}{]}
        \PY{n}{list\PYZus{}c}\PY{o}{.}\PY{n}{append}\PY{p}{(}\PY{p}{[}\PY{l+m+mi}{6}\PY{p}{,} \PY{l+m+mi}{7}\PY{p}{,} \PY{l+m+mi}{8}\PY{p}{]}\PY{p}{)}
        \PY{n}{list\PYZus{}c}
\end{Verbatim}


    \begin{Verbatim}[commandchars=\\\{\}]
{\color{incolor}In [{\color{incolor} }]:} \PY{n}{list\PYZus{}b} \PY{o}{=} \PY{p}{[}\PY{l+m+mi}{1}\PY{p}{,} \PY{l+m+mi}{2}\PY{p}{,} \PY{l+m+mi}{3}\PY{p}{,} \PY{l+m+mi}{4}\PY{p}{,} \PY{l+m+mi}{5}\PY{p}{]}
        \PY{n}{list\PYZus{}b}\PY{o}{.}\PY{n}{extend}\PY{p}{(}\PY{p}{[}\PY{l+m+mi}{6}\PY{p}{,} \PY{l+m+mi}{7}\PY{p}{,} \PY{l+m+mi}{8}\PY{p}{]}\PY{p}{)}
        \PY{n}{list\PYZus{}b}
\end{Verbatim}


    \begin{Verbatim}[commandchars=\\\{\}]
{\color{incolor}In [{\color{incolor} }]:} \PY{n}{list\PYZus{}d} \PY{o}{=} \PY{p}{[}\PY{l+m+mi}{1}\PY{p}{,} \PY{l+m+mi}{2}\PY{p}{,} \PY{l+m+mi}{3}\PY{p}{,} \PY{l+m+mi}{4}\PY{p}{,} \PY{l+m+mi}{5}\PY{p}{]}
        \PY{n}{list\PYZus{}d}\PY{o}{.}\PY{n}{extend}\PY{p}{(}\PY{l+m+mi}{6}\PY{p}{)}    \PY{c+c1}{\PYZsh{} 此处故意写出错误代码 以示区别}
        \PY{n}{list\PYZus{}d}
\end{Verbatim}


    \begin{enumerate}
\def\labelenumi{\arabic{enumi}.}
\setcounter{enumi}{2}
\tightlist
\item
  list.insert(i, x)
\end{enumerate}

Insert an item at a given position. * i is the index of the element
before which to insert, so \textbf{a.insert(0, x)} inserts at the front
of the list. * \textbf{a.insert(len(a), x)} is equivalent to
\textbf{a.append(x)}

    \begin{enumerate}
\def\labelenumi{\arabic{enumi}.}
\setcounter{enumi}{3}
\item
  list.remove(x)

  \begin{itemize}
  \tightlist
  \item
    Remove the first item from the list whose value is equal to x.
  \item
    It raises a \textbf{ValueError} if there is no such item.
  \end{itemize}
\end{enumerate}

    \begin{enumerate}
\def\labelenumi{\arabic{enumi}.}
\setcounter{enumi}{4}
\tightlist
\item
  list.pop({[}i{]})
\end{enumerate}

{[}{]}代表可选参数,参数默认值为-1.

Remove the item at the given position in the list, and return it. If no
index is specified, it removes and returns the last item in the list.

    \begin{enumerate}
\def\labelenumi{\arabic{enumi}.}
\setcounter{enumi}{5}
\tightlist
\item
  list.clear()
\end{enumerate}

Remove all items from the list. Equivalent to \textbf{del list{[}:{]}}

    \begin{enumerate}
\def\labelenumi{\arabic{enumi}.}
\setcounter{enumi}{6}
\item
  list.index(x{[}, start{[}, end{]}{]})

  \begin{itemize}
  \item
    Return zero-based index in the list of the first item whose value is
    equal to x. Raises a \textbf{ValueError} if there is no such item.
  \item
    The optional arguments are the slice notation. The returned index is
    still computed relative to the beginning of the full list.
  \end{itemize}
\end{enumerate}

    \begin{enumerate}
\def\labelenumi{\arabic{enumi}.}
\setcounter{enumi}{7}
\tightlist
\item
  list.count(x)
\end{enumerate}

Return the number of times x appears in the list.

    \begin{enumerate}
\def\labelenumi{\arabic{enumi}.}
\setcounter{enumi}{8}
\tightlist
\item
  list.sort(key=None, reverse=False)
\end{enumerate}

Sort the items of the list in place. key specifies a function of one
argument that is used to extract a comparison key from each list element
(key 必须是个function(lambda function)???)

    \begin{Verbatim}[commandchars=\\\{\}]
{\color{incolor}In [{\color{incolor} }]:} \PY{n}{pairs} \PY{o}{=} \PY{p}{[}\PY{p}{(}\PY{l+m+mi}{1}\PY{p}{,} \PY{l+s+s1}{\PYZsq{}}\PY{l+s+s1}{one}\PY{l+s+s1}{\PYZsq{}}\PY{p}{)}\PY{p}{,} \PY{p}{(}\PY{l+m+mi}{2}\PY{p}{,} \PY{l+s+s1}{\PYZsq{}}\PY{l+s+s1}{two}\PY{l+s+s1}{\PYZsq{}}\PY{p}{)}\PY{p}{,} \PY{p}{(}\PY{l+m+mi}{3}\PY{p}{,} \PY{l+s+s1}{\PYZsq{}}\PY{l+s+s1}{three}\PY{l+s+s1}{\PYZsq{}}\PY{p}{)}\PY{p}{,} \PY{p}{(}\PY{l+m+mi}{4}\PY{p}{,} \PY{l+s+s1}{\PYZsq{}}\PY{l+s+s1}{four}\PY{l+s+s1}{\PYZsq{}}\PY{p}{)}\PY{p}{]}
        \PY{n}{pairs}\PY{o}{.}\PY{n}{sort}\PY{p}{(}\PY{n}{key}\PY{o}{=}\PY{k}{lambda} \PY{n}{pair}\PY{p}{:} \PY{n}{pair}\PY{p}{[}\PY{l+m+mi}{1}\PY{p}{]}\PY{p}{)}    \PY{c+c1}{\PYZsh{}按 里层 list第二个元素排序 也就是 按照one two three four的首字母排序}
        \PY{n}{pairs}
\end{Verbatim}


    \begin{Verbatim}[commandchars=\\\{\}]
{\color{incolor}In [{\color{incolor} }]:} \PY{n}{pairs} \PY{o}{=} \PY{p}{[}\PY{p}{(}\PY{l+m+mi}{1}\PY{p}{,} \PY{l+s+s1}{\PYZsq{}}\PY{l+s+s1}{one}\PY{l+s+s1}{\PYZsq{}}\PY{p}{)}\PY{p}{,} \PY{p}{(}\PY{l+m+mi}{2}\PY{p}{,} \PY{l+s+s1}{\PYZsq{}}\PY{l+s+s1}{two}\PY{l+s+s1}{\PYZsq{}}\PY{p}{)}\PY{p}{,} \PY{p}{(}\PY{l+m+mi}{3}\PY{p}{,} \PY{l+s+s1}{\PYZsq{}}\PY{l+s+s1}{three}\PY{l+s+s1}{\PYZsq{}}\PY{p}{)}\PY{p}{,} \PY{p}{(}\PY{l+m+mi}{4}\PY{p}{,} \PY{l+s+s1}{\PYZsq{}}\PY{l+s+s1}{four}\PY{l+s+s1}{\PYZsq{}}\PY{p}{)}\PY{p}{]}
        \PY{n}{pairs}\PY{o}{.}\PY{n}{sort}\PY{p}{(}\PY{n}{key}\PY{o}{=}\PY{n}{pairs}\PY{p}{[}\PY{l+m+mi}{0}\PY{p}{]}\PY{p}{[}\PY{l+m+mi}{1}\PY{p}{]}\PY{p}{)}    \PY{c+c1}{\PYZsh{}此处故意把key改成非lambda函数,以示区别。}
\end{Verbatim}


    \begin{enumerate}
\def\labelenumi{\arabic{enumi}.}
\setcounter{enumi}{9}
\tightlist
\item
  list.reverse()
\end{enumerate}

Reverse the elements of the list in place.

    \begin{enumerate}
\def\labelenumi{\arabic{enumi}.}
\setcounter{enumi}{10}
\tightlist
\item
  list.copy()
\end{enumerate}

Return a shallow copy of the list. Equivalent to \textbf{a{[}:{]}}.

    Examples that uses most of the list methods:

    \begin{Verbatim}[commandchars=\\\{\}]
{\color{incolor}In [{\color{incolor} }]:} \PY{n}{fruits} \PY{o}{=} \PY{p}{[}\PY{l+s+s1}{\PYZsq{}}\PY{l+s+s1}{orrange}\PY{l+s+s1}{\PYZsq{}}\PY{p}{,} \PY{l+s+s1}{\PYZsq{}}\PY{l+s+s1}{apple}\PY{l+s+s1}{\PYZsq{}}\PY{p}{,} \PY{l+s+s1}{\PYZsq{}}\PY{l+s+s1}{pear}\PY{l+s+s1}{\PYZsq{}}\PY{p}{,} \PY{l+s+s1}{\PYZsq{}}\PY{l+s+s1}{banana}\PY{l+s+s1}{\PYZsq{}}\PY{p}{,} \PY{l+s+s1}{\PYZsq{}}\PY{l+s+s1}{kiwi}\PY{l+s+s1}{\PYZsq{}}\PY{p}{,} \PY{l+s+s1}{\PYZsq{}}\PY{l+s+s1}{apple}\PY{l+s+s1}{\PYZsq{}}\PY{p}{,} \PY{l+s+s1}{\PYZsq{}}\PY{l+s+s1}{banana}\PY{l+s+s1}{\PYZsq{}}\PY{p}{]}
        \PY{n}{fruits}\PY{o}{.}\PY{n}{count}\PY{p}{(}\PY{l+s+s1}{\PYZsq{}}\PY{l+s+s1}{apple}\PY{l+s+s1}{\PYZsq{}}\PY{p}{)}
\end{Verbatim}


    \begin{Verbatim}[commandchars=\\\{\}]
{\color{incolor}In [{\color{incolor} }]:} \PY{n}{fruits}\PY{o}{.}\PY{n}{count}\PY{p}{(}\PY{l+s+s1}{\PYZsq{}}\PY{l+s+s1}{tangerine}\PY{l+s+s1}{\PYZsq{}}\PY{p}{)}
\end{Verbatim}


    \begin{Verbatim}[commandchars=\\\{\}]
{\color{incolor}In [{\color{incolor} }]:} \PY{n}{fruits}\PY{o}{.}\PY{n}{index}\PY{p}{(}\PY{l+s+s1}{\PYZsq{}}\PY{l+s+s1}{banana}\PY{l+s+s1}{\PYZsq{}}\PY{p}{)}
\end{Verbatim}


    \begin{Verbatim}[commandchars=\\\{\}]
{\color{incolor}In [{\color{incolor} }]:} \PY{n}{fruits}\PY{o}{.}\PY{n}{index}\PY{p}{(}\PY{l+s+s1}{\PYZsq{}}\PY{l+s+s1}{banana}\PY{l+s+s1}{\PYZsq{}}\PY{p}{,} \PY{l+m+mi}{4}\PY{p}{)}  \PY{c+c1}{\PYZsh{} Find next banana starting at position 4}
\end{Verbatim}


    \begin{Verbatim}[commandchars=\\\{\}]
{\color{incolor}In [{\color{incolor} }]:} \PY{n}{fruits}\PY{o}{.}\PY{n}{reverse}\PY{p}{(}\PY{p}{)}
\end{Verbatim}


    \begin{Verbatim}[commandchars=\\\{\}]
{\color{incolor}In [{\color{incolor} }]:} \PY{n}{fruits}
\end{Verbatim}


    \begin{Verbatim}[commandchars=\\\{\}]
{\color{incolor}In [{\color{incolor} }]:} \PY{n}{fruits}\PY{o}{.}\PY{n}{append}\PY{p}{(}\PY{l+s+s1}{\PYZsq{}}\PY{l+s+s1}{grape}\PY{l+s+s1}{\PYZsq{}}\PY{p}{)}
\end{Verbatim}


    \begin{Verbatim}[commandchars=\\\{\}]
{\color{incolor}In [{\color{incolor} }]:} \PY{n}{fruits}
\end{Verbatim}


    \begin{Verbatim}[commandchars=\\\{\}]
{\color{incolor}In [{\color{incolor} }]:} \PY{n}{fruits}\PY{o}{.}\PY{n}{sort}\PY{p}{(}\PY{p}{)}
\end{Verbatim}


    \begin{Verbatim}[commandchars=\\\{\}]
{\color{incolor}In [{\color{incolor} }]:} \PY{n}{fruits}
\end{Verbatim}


    \begin{Verbatim}[commandchars=\\\{\}]
{\color{incolor}In [{\color{incolor} }]:} \PY{n}{fruits}\PY{o}{.}\PY{n}{pop}\PY{p}{(}\PY{p}{)}
\end{Verbatim}


    \paragraph{Using Lists as Stacks}\label{using-lists-as-stacks}

    \begin{itemize}
\tightlist
\item
  The list methods make it very easy to use a list as a stack, where the
  last element added is the first element retrieved (last-in,
  first-out).
\item
  To add an item to the top of the stack, use \textbf{append()}. To
  retrieve an item from the top of the stack, use \textbf{pop()} without
  an explicit index. For example:
\end{itemize}

    \begin{Verbatim}[commandchars=\\\{\}]
{\color{incolor}In [{\color{incolor} }]:} \PY{n}{stack} \PY{o}{=} \PY{p}{[}\PY{l+m+mi}{3}\PY{p}{,} \PY{l+m+mi}{4}\PY{p}{,} \PY{l+m+mi}{5}\PY{p}{]}
        \PY{n}{stack}\PY{o}{.}\PY{n}{append}\PY{p}{(}\PY{l+m+mi}{6}\PY{p}{)}
        \PY{n}{stack}\PY{o}{.}\PY{n}{append}\PY{p}{(}\PY{l+m+mi}{7}\PY{p}{)}
        \PY{n}{stack}
\end{Verbatim}


    \begin{Verbatim}[commandchars=\\\{\}]
{\color{incolor}In [{\color{incolor} }]:} \PY{n}{stack}\PY{o}{.}\PY{n}{pop}\PY{p}{(}\PY{p}{)}
\end{Verbatim}


    \begin{Verbatim}[commandchars=\\\{\}]
{\color{incolor}In [{\color{incolor} }]:} \PY{n}{stack}
\end{Verbatim}


    \begin{Verbatim}[commandchars=\\\{\}]
{\color{incolor}In [{\color{incolor} }]:} \PY{n}{stack}\PY{o}{.}\PY{n}{pop}\PY{p}{(}\PY{p}{)}
\end{Verbatim}


    \begin{Verbatim}[commandchars=\\\{\}]
{\color{incolor}In [{\color{incolor} }]:} \PY{n}{stack}
\end{Verbatim}


    \begin{Verbatim}[commandchars=\\\{\}]
{\color{incolor}In [{\color{incolor} }]:} \PY{n}{stack}\PY{o}{.}\PY{n}{pop}\PY{p}{(}\PY{p}{)}
\end{Verbatim}


    \begin{Verbatim}[commandchars=\\\{\}]
{\color{incolor}In [{\color{incolor} }]:} \PY{n}{stack}
\end{Verbatim}


    \paragraph{Using Lists as Queues}\label{using-lists-as-queues}

    As queues, the first element added is the first element retrieved
(first-in, first-out); however, \textbf{lists are not efficient for this
purpose}.Because all of the other elements have to be shifted by one.To
implement a queue, use a \textbf{collections.deque} which was designed
to have fast appends and pops from both ends. For example:

    \begin{Verbatim}[commandchars=\\\{\}]
{\color{incolor}In [{\color{incolor} }]:} \PY{k+kn}{from} \PY{n+nn}{collections} \PY{k}{import} \PY{n}{deque}
        \PY{n}{queue} \PY{o}{=} \PY{n}{deque}\PY{p}{(}\PY{p}{[}\PY{l+s+s2}{\PYZdq{}}\PY{l+s+s2}{Eric}\PY{l+s+s2}{\PYZdq{}}\PY{p}{,} \PY{l+s+s2}{\PYZdq{}}\PY{l+s+s2}{John}\PY{l+s+s2}{\PYZdq{}}\PY{p}{,} \PY{l+s+s2}{\PYZdq{}}\PY{l+s+s2}{Michael}\PY{l+s+s2}{\PYZdq{}}\PY{p}{]}\PY{p}{)}
        \PY{n}{queue}\PY{o}{.}\PY{n}{append}\PY{p}{(}\PY{l+s+s2}{\PYZdq{}}\PY{l+s+s2}{Terry}\PY{l+s+s2}{\PYZdq{}}\PY{p}{)}
        \PY{n}{queue}\PY{o}{.}\PY{n}{append}\PY{p}{(}\PY{l+s+s2}{\PYZdq{}}\PY{l+s+s2}{Graham}\PY{l+s+s2}{\PYZdq{}}\PY{p}{)}
        \PY{n}{queue}
\end{Verbatim}


    \begin{Verbatim}[commandchars=\\\{\}]
{\color{incolor}In [{\color{incolor} }]:} \PY{n}{queue}\PY{o}{.}\PY{n}{popleft}\PY{p}{(}\PY{p}{)}    \PY{c+c1}{\PYZsh{} no popright() function}
\end{Verbatim}


    \begin{Verbatim}[commandchars=\\\{\}]
{\color{incolor}In [{\color{incolor} }]:} \PY{n}{queue}\PY{o}{.}\PY{n}{popleft}\PY{p}{(}\PY{p}{)}
\end{Verbatim}


    \begin{Verbatim}[commandchars=\\\{\}]
{\color{incolor}In [{\color{incolor} }]:} \PY{n}{queue}
\end{Verbatim}


    \paragraph{List Comprehensions}\label{list-comprehensions}

    List comprehensions provide a concise way to create lists. Example:

    \begin{Verbatim}[commandchars=\\\{\}]
{\color{incolor}In [{\color{incolor} }]:} \PY{n}{squares} \PY{o}{=} \PY{p}{[}\PY{p}{]}
        \PY{k}{for} \PY{n}{x} \PY{o+ow}{in} \PY{n+nb}{range}\PY{p}{(}\PY{l+m+mi}{10}\PY{p}{)}\PY{p}{:}
            \PY{n}{squares}\PY{o}{.}\PY{n}{append}\PY{p}{(}\PY{n}{x}\PY{o}{*}\PY{o}{*}\PY{l+m+mi}{2}\PY{p}{)}
        \PY{n}{squares}
\end{Verbatim}


    下面是更简单的两种办法,算同样的squares

    map(func, squence) function: 返回在squence上使用func函数得到的结果。

    \begin{Verbatim}[commandchars=\\\{\}]
{\color{incolor}In [{\color{incolor} }]:} \PY{n}{squares} \PY{o}{=} \PY{n+nb}{list}\PY{p}{(}\PY{n+nb}{map}\PY{p}{(}\PY{k}{lambda} \PY{n}{x}\PY{p}{:} \PY{n}{x}\PY{o}{*}\PY{o}{*}\PY{l+m+mi}{2}\PY{p}{,} \PY{n+nb}{range}\PY{p}{(}\PY{l+m+mi}{10}\PY{p}{)}\PY{p}{)}\PY{p}{)}
        \PY{n}{squares}
\end{Verbatim}


    \begin{Verbatim}[commandchars=\\\{\}]
{\color{incolor}In [{\color{incolor} }]:} \PY{n}{squares} \PY{o}{=} \PY{p}{[}\PY{n}{x}\PY{o}{*}\PY{o}{*}\PY{l+m+mi}{2} \PY{k}{for} \PY{n}{x} \PY{o+ow}{in} \PY{n+nb}{range}\PY{p}{(}\PY{l+m+mi}{10}\PY{p}{)}\PY{p}{]} \PY{c+c1}{\PYZsh{}this is more concise and readable}
        \PY{n}{squares}
\end{Verbatim}


    A list comprehension consists of brackets containing an expression
followed by a \emph{for} clause, then zero or more \emph{for} or
\emph{if} clauses. The result will be a new list resulting form
evluating the expression in the context of the \emph{for} and \emph{if}
clauses which follow it. Example:

Combines the elements of two lists if they are not equal.

    \begin{Verbatim}[commandchars=\\\{\}]
{\color{incolor}In [{\color{incolor} }]:} \PY{p}{[}\PY{p}{(}\PY{n}{x}\PY{p}{,} \PY{n}{y}\PY{p}{)} \PY{k}{for} \PY{n}{x} \PY{o+ow}{in} \PY{p}{[}\PY{l+m+mi}{1}\PY{p}{,}\PY{l+m+mi}{2}\PY{p}{,}\PY{l+m+mi}{3}\PY{p}{]} \PY{k}{for} \PY{n}{y} \PY{o+ow}{in} \PY{p}{[}\PY{l+m+mi}{1}\PY{p}{,}\PY{l+m+mi}{2}\PY{p}{,}\PY{l+m+mi}{4}\PY{p}{,}\PY{l+m+mi}{5}\PY{p}{]} \PY{k}{if} \PY{n}{x} \PY{o}{!=} \PY{n}{y}\PY{p}{]}
\end{Verbatim}


    The above is equavelent to:

    \begin{Verbatim}[commandchars=\\\{\}]
{\color{incolor}In [{\color{incolor} }]:} \PY{n}{combs} \PY{o}{=} \PY{p}{[}\PY{p}{]}
        \PY{k}{for} \PY{n}{x} \PY{o+ow}{in} \PY{p}{[}\PY{l+m+mi}{1}\PY{p}{,}\PY{l+m+mi}{2}\PY{p}{,}\PY{l+m+mi}{3}\PY{p}{]}\PY{p}{:}
            \PY{k}{for} \PY{n}{y} \PY{o+ow}{in} \PY{p}{[}\PY{l+m+mi}{1}\PY{p}{,}\PY{l+m+mi}{2}\PY{p}{,}\PY{l+m+mi}{4}\PY{p}{,}\PY{l+m+mi}{5}\PY{p}{]}\PY{p}{:}
                \PY{k}{if} \PY{n}{x} \PY{o}{!=} \PY{n}{y}\PY{p}{:}
                    \PY{n}{combs}\PY{o}{.}\PY{n}{append}\PY{p}{(}\PY{p}{(}\PY{n}{x}\PY{p}{,} \PY{n}{y}\PY{p}{)}\PY{p}{)}
        \PY{n}{combs}
\end{Verbatim}


    If the expression is a tuple, it must be parenthesized:

    \begin{Verbatim}[commandchars=\\\{\}]
{\color{incolor}In [{\color{incolor} }]:} \PY{n}{vec} \PY{o}{=} \PY{p}{[}\PY{o}{\PYZhy{}}\PY{l+m+mi}{4}\PY{p}{,} \PY{o}{\PYZhy{}}\PY{l+m+mi}{2}\PY{p}{,} \PY{l+m+mi}{0}\PY{p}{,} \PY{l+m+mi}{2}\PY{p}{,} \PY{l+m+mi}{4}\PY{p}{]}
        \PY{p}{[}\PY{n}{x}\PY{o}{*}\PY{l+m+mi}{2} \PY{k}{for} \PY{n}{x} \PY{o+ow}{in} \PY{n}{vec}\PY{p}{]}
\end{Verbatim}


    filter the list with an \emph{if} clause:

    \begin{Verbatim}[commandchars=\\\{\}]
{\color{incolor}In [{\color{incolor} }]:} \PY{p}{[}\PY{n}{x} \PY{k}{for} \PY{n}{x} \PY{o+ow}{in} \PY{n}{vec} \PY{k}{if} \PY{n}{x} \PY{o}{\PYZgt{}}\PY{o}{=}\PY{l+m+mi}{0}\PY{p}{]}
\end{Verbatim}


    apply a function to all elements:

    \begin{Verbatim}[commandchars=\\\{\}]
{\color{incolor}In [{\color{incolor} }]:} \PY{p}{[}\PY{n+nb}{abs}\PY{p}{(}\PY{n}{x}\PY{p}{)} \PY{k}{for} \PY{n}{x} \PY{o+ow}{in} \PY{n}{vec}\PY{p}{]}
\end{Verbatim}


    call a method on each element:

    \begin{Verbatim}[commandchars=\\\{\}]
{\color{incolor}In [{\color{incolor} }]:} \PY{n}{freshfruit} \PY{o}{=} \PY{p}{[}\PY{l+s+s1}{\PYZsq{}}\PY{l+s+s1}{   banana}\PY{l+s+s1}{\PYZsq{}}\PY{p}{,} \PY{l+s+s1}{\PYZsq{}}\PY{l+s+s1}{   loganberry  }\PY{l+s+s1}{\PYZsq{}}\PY{p}{,} \PY{l+s+s1}{\PYZsq{}}\PY{l+s+s1}{passion fruit  }\PY{l+s+s1}{\PYZsq{}}\PY{p}{]}
        \PY{p}{[}\PY{n}{weapon}\PY{o}{.}\PY{n}{strip}\PY{p}{(}\PY{p}{)} \PY{k}{for} \PY{n}{weapon} \PY{o+ow}{in} \PY{n}{freshfruit}\PY{p}{]}
\end{Verbatim}


    List comprehensions can contain complex expressions and nested
functions:

    \begin{Verbatim}[commandchars=\\\{\}]
{\color{incolor}In [{\color{incolor} }]:} \PY{k+kn}{from} \PY{n+nn}{math} \PY{k}{import} \PY{n}{pi}
        \PY{p}{[}\PY{n+nb}{str}\PY{p}{(}\PY{n+nb}{round}\PY{p}{(}\PY{n}{pi}\PY{p}{,} \PY{n}{i}\PY{p}{)}\PY{p}{)} \PY{k}{for} \PY{n}{i} \PY{o+ow}{in} \PY{n+nb}{range}\PY{p}{(}\PY{l+m+mi}{1}\PY{p}{,} \PY{l+m+mi}{6}\PY{p}{)}\PY{p}{]}
\end{Verbatim}


    \paragraph{Nested List Comprehensions}\label{nested-list-comprehensions}

    The initial expression in a list comprehension can be any arbitrary
expression, including another list comprehension.

    \begin{Verbatim}[commandchars=\\\{\}]
{\color{incolor}In [{\color{incolor} }]:} \PY{n}{matrix} \PY{o}{=} \PY{p}{[}
            \PY{p}{[}\PY{l+m+mi}{1}\PY{p}{,} \PY{l+m+mi}{2}\PY{p}{,} \PY{l+m+mi}{3}\PY{p}{,} \PY{l+m+mi}{4}\PY{p}{]}\PY{p}{,}
            \PY{p}{[}\PY{l+m+mi}{5}\PY{p}{,} \PY{l+m+mi}{6}\PY{p}{,} \PY{l+m+mi}{7}\PY{p}{,} \PY{l+m+mi}{8}\PY{p}{]}\PY{p}{,}
            \PY{p}{[}\PY{l+m+mi}{9}\PY{p}{,} \PY{l+m+mi}{10}\PY{p}{,} \PY{l+m+mi}{11}\PY{p}{,} \PY{l+m+mi}{12}\PY{p}{]}\PY{p}{,}
        \PY{p}{]}
        \PY{n}{matrix}
\end{Verbatim}


    The following list comprehension will transpose rows and columns,
easiest way to do transpose is by the built-in function zip(), shown
below:

    \begin{Verbatim}[commandchars=\\\{\}]
{\color{incolor}In [{\color{incolor} }]:} \PY{p}{[}\PY{p}{[}\PY{n}{row}\PY{p}{[}\PY{n}{i}\PY{p}{]} \PY{k}{for} \PY{n}{row} \PY{o+ow}{in} \PY{n}{matrix}\PY{p}{]} \PY{k}{for} \PY{n}{i} \PY{o+ow}{in} \PY{n+nb}{range}\PY{p}{(}\PY{l+m+mi}{4}\PY{p}{)}\PY{p}{]}
\end{Verbatim}


    The above example is equal to:

    \begin{Verbatim}[commandchars=\\\{\}]
{\color{incolor}In [{\color{incolor} }]:} \PY{n}{transposed} \PY{o}{=} \PY{p}{[}\PY{p}{]}
        \PY{k}{for} \PY{n}{i} \PY{o+ow}{in} \PY{n+nb}{range}\PY{p}{(}\PY{l+m+mi}{4}\PY{p}{)}\PY{p}{:}
            \PY{n}{transposed}\PY{o}{.}\PY{n}{append}\PY{p}{(}\PY{p}{[}\PY{n}{row}\PY{p}{[}\PY{n}{i}\PY{p}{]} \PY{k}{for} \PY{n}{row} \PY{o+ow}{in} \PY{n}{matrix}\PY{p}{]}\PY{p}{)}
        \PY{n}{transposed}
\end{Verbatim}


    \begin{Verbatim}[commandchars=\\\{\}]
{\color{incolor}In [{\color{incolor} }]:} \PY{n+nb}{list}\PY{p}{(}\PY{n+nb}{zip}\PY{p}{(}\PY{o}{*}\PY{n}{matrix}\PY{p}{)}\PY{p}{)}
\end{Verbatim}


    \subsubsection{The del statement}\label{the-del-statement}

    It is used to remove an item from a list given its index instead of its
value.

    \begin{Verbatim}[commandchars=\\\{\}]
{\color{incolor}In [{\color{incolor} }]:} \PY{n}{a} \PY{o}{=} \PY{p}{[}\PY{o}{\PYZhy{}}\PY{l+m+mi}{1}\PY{p}{,} \PY{l+m+mi}{1}\PY{p}{,} \PY{l+m+mi}{653}\PY{p}{,} \PY{l+m+mi}{435}\PY{p}{,} \PY{l+m+mf}{234.23}\PY{p}{,} \PY{l+m+mi}{99}\PY{p}{]}
        \PY{k}{del} \PY{n}{a}\PY{p}{[}\PY{l+m+mi}{0}\PY{p}{]}
        \PY{n}{a}
\end{Verbatim}


    \begin{Verbatim}[commandchars=\\\{\}]
{\color{incolor}In [{\color{incolor} }]:} \PY{k}{del} \PY{n}{a}\PY{p}{[}\PY{p}{:}\PY{l+m+mi}{2}\PY{p}{]}
        \PY{n}{a}
\end{Verbatim}


    It can also be used to delete the entire variable:

    \begin{Verbatim}[commandchars=\\\{\}]
{\color{incolor}In [{\color{incolor} }]:} \PY{k}{del} \PY{n}{a} 
        \PY{n}{a}    \PY{c+c1}{\PYZsh{} Referencing the name hereafter will show an error.}
\end{Verbatim}


    \subsubsection{Tuples and Sequences}\label{tuples-and-sequences}

    \paragraph{Tuple Basics}\label{tuple-basics}

    This is an example of tuple packing: (we'll see unpacking later)

    \begin{Verbatim}[commandchars=\\\{\}]
{\color{incolor}In [{\color{incolor} }]:} \PY{n}{t} \PY{o}{=} \PY{l+m+mi}{12345}\PY{p}{,} \PY{l+m+mi}{54321}\PY{p}{,} \PY{l+s+s1}{\PYZsq{}}\PY{l+s+s1}{hello}\PY{l+s+s1}{\PYZsq{}}
        \PY{n}{t}\PY{p}{[}\PY{l+m+mi}{0}\PY{p}{]}
\end{Verbatim}


    \begin{Verbatim}[commandchars=\\\{\}]
{\color{incolor}In [{\color{incolor} }]:} \PY{n}{t}
\end{Verbatim}


    \begin{Verbatim}[commandchars=\\\{\}]
{\color{incolor}In [{\color{incolor} }]:} \PY{n}{u} \PY{o}{=} \PY{n}{t}\PY{p}{,} \PY{p}{(}\PY{l+m+mi}{1}\PY{p}{,} \PY{l+m+mi}{2}\PY{p}{,} \PY{l+m+mi}{3}\PY{p}{,} \PY{l+m+mi}{4}\PY{p}{,} \PY{l+m+mi}{5}\PY{p}{)}
        \PY{n}{u}
\end{Verbatim}


    Tuples are immutable!!!!

    \begin{Verbatim}[commandchars=\\\{\}]
{\color{incolor}In [{\color{incolor} }]:} \PY{n}{t}\PY{p}{[}\PY{l+m+mi}{0}\PY{p}{]} \PY{o}{=} \PY{l+m+mi}{88888}
\end{Verbatim}


    Special consideration for constructing tuples containing 0 or 1 items:

\begin{verbatim}
* Empty tuples are constructed by an empty pair of parentheses

* Tuples with 1 item is contructed by following a value with a comma (it is not sufficient to enclose a single value in parentheses). For example:
\end{verbatim}

    \begin{Verbatim}[commandchars=\\\{\}]
{\color{incolor}In [{\color{incolor} }]:} \PY{n}{empty} \PY{o}{=} \PY{p}{(}\PY{p}{)}
        \PY{n}{singleton} \PY{o}{=} \PY{l+s+s1}{\PYZsq{}}\PY{l+s+s1}{hello}\PY{l+s+s1}{\PYZsq{}}\PY{p}{,}
        \PY{n+nb}{len}\PY{p}{(}\PY{n}{empty}\PY{p}{)}
\end{Verbatim}


    \begin{Verbatim}[commandchars=\\\{\}]
{\color{incolor}In [{\color{incolor} }]:} \PY{n+nb}{len}\PY{p}{(}\PY{n}{singleton}\PY{p}{)}
\end{Verbatim}


    \begin{Verbatim}[commandchars=\\\{\}]
{\color{incolor}In [{\color{incolor} }]:} \PY{n}{singleton}
\end{Verbatim}


    tuple unpacking:

    \begin{Verbatim}[commandchars=\\\{\}]
{\color{incolor}In [{\color{incolor} }]:} \PY{n}{x}\PY{p}{,} \PY{n}{y}\PY{p}{,} \PY{n}{z}\PY{p}{,} \PY{o}{=} \PY{n}{t}
        \PY{n}{x}
\end{Verbatim}


    \begin{Verbatim}[commandchars=\\\{\}]
{\color{incolor}In [{\color{incolor} }]:} \PY{n}{y}
\end{Verbatim}


    \begin{Verbatim}[commandchars=\\\{\}]
{\color{incolor}In [{\color{incolor} }]:} \PY{n}{z}
\end{Verbatim}


    \subsubsection{Sets}\label{sets}

    A set is an \textbf{unordered} collection with \textbf{no duplicate}
elements.

Basic uses: * membership testing ??????? * eliminating duplicate entries

Also support mathematical operations like: * union * intersection *
difference * symmetric difference ?????

How to create sets: * Curly braces \{\} (not for enpty sets, this will
create an empty dictionary) * set() function (have to use this to create
empty sets) * The above two ways are not exactly the same, see the below
demo.

here's a brief demonstration:

    \begin{Verbatim}[commandchars=\\\{\}]
{\color{incolor}In [{\color{incolor} }]:} \PY{n}{basket} \PY{o}{=} \PY{p}{\PYZob{}}\PY{l+s+s1}{\PYZsq{}}\PY{l+s+s1}{apple}\PY{l+s+s1}{\PYZsq{}}\PY{p}{,} \PY{l+s+s1}{\PYZsq{}}\PY{l+s+s1}{orange}\PY{l+s+s1}{\PYZsq{}}\PY{p}{,} \PY{l+s+s1}{\PYZsq{}}\PY{l+s+s1}{apple}\PY{l+s+s1}{\PYZsq{}}\PY{p}{,} \PY{l+s+s1}{\PYZsq{}}\PY{l+s+s1}{pear}\PY{l+s+s1}{\PYZsq{}}\PY{p}{,} \PY{l+s+s1}{\PYZsq{}}\PY{l+s+s1}{orange}\PY{l+s+s1}{\PYZsq{}}\PY{p}{,} \PY{l+s+s1}{\PYZsq{}}\PY{l+s+s1}{banana}\PY{l+s+s1}{\PYZsq{}}\PY{p}{\PYZcb{}}
        \PY{n+nb}{print}\PY{p}{(}\PY{n}{basket}\PY{p}{)} \PY{c+c1}{\PYZsh{} duplicates have been removed}
        \PY{l+s+s1}{\PYZsq{}}\PY{l+s+s1}{orange}\PY{l+s+s1}{\PYZsq{}} \PY{o+ow}{in} \PY{n}{basket} \PY{c+c1}{\PYZsh{} fast membership testing}
\end{Verbatim}


    \begin{Verbatim}[commandchars=\\\{\}]
{\color{incolor}In [{\color{incolor} }]:} \PY{l+s+s1}{\PYZsq{}}\PY{l+s+s1}{crabgrass}\PY{l+s+s1}{\PYZsq{}} \PY{o+ow}{in} \PY{n}{basket}
\end{Verbatim}


    set operations on unique letters from two words

    \begin{Verbatim}[commandchars=\\\{\}]
{\color{incolor}In [{\color{incolor} }]:} \PY{n}{a} \PY{o}{=} \PY{n+nb}{set}\PY{p}{(}\PY{l+s+s1}{\PYZsq{}}\PY{l+s+s1}{abracadabra}\PY{l+s+s1}{\PYZsq{}}\PY{p}{)} \PY{c+c1}{\PYZsh{} use \PYZob{}\PYZcb{} if you want to create a set == \PYZsq{}abracadabra\PYZsq{}}
        \PY{n}{b} \PY{o}{=} \PY{n+nb}{set}\PY{p}{(}\PY{l+s+s1}{\PYZsq{}}\PY{l+s+s1}{alacazam}\PY{l+s+s1}{\PYZsq{}}\PY{p}{)}
        \PY{n}{a}    \PY{c+c1}{\PYZsh{} unique letters in a}
\end{Verbatim}


    \begin{Verbatim}[commandchars=\\\{\}]
{\color{incolor}In [{\color{incolor} }]:} \PY{n}{aa} \PY{o}{=} \PY{p}{\PYZob{}}\PY{l+s+s1}{\PYZsq{}}\PY{l+s+s1}{abracadabra}\PY{l+s+s1}{\PYZsq{}}\PY{p}{\PYZcb{}} \PY{c+c1}{\PYZsh{} same string, different sets as the above}
        \PY{n}{aa}
\end{Verbatim}


    \begin{Verbatim}[commandchars=\\\{\}]
{\color{incolor}In [{\color{incolor} }]:} \PY{n}{a} \PY{o}{\PYZhy{}} \PY{n}{b}    \PY{c+c1}{\PYZsh{} letters in a but not in b}
\end{Verbatim}


    \begin{Verbatim}[commandchars=\\\{\}]
{\color{incolor}In [{\color{incolor} }]:} \PY{n}{a} \PY{o}{|} \PY{n}{b}    \PY{c+c1}{\PYZsh{} letters in a or b}
\end{Verbatim}


    \begin{Verbatim}[commandchars=\\\{\}]
{\color{incolor}In [{\color{incolor} }]:} \PY{n}{a} \PY{o}{\PYZam{}} \PY{n}{b}    \PY{c+c1}{\PYZsh{} letters in a and b}
\end{Verbatim}


    \begin{Verbatim}[commandchars=\\\{\}]
{\color{incolor}In [{\color{incolor} }]:} \PY{n}{a} \PY{o}{\PYZca{}} \PY{n}{b}    \PY{c+c1}{\PYZsh{} letters in a or b but not both}
\end{Verbatim}


    \subsubsection{Dictionaries}\label{dictionaries}

    Unlike sequences, which are indexed by a range of numbers, dictionaries
are indexed by \emph{keys}, which can be any immutable type: * strings
and numbers can always be \emph{keys}. * \emph{Tuples} can be used as
keys if they contain only strings, numbers, or tuples; * \emph{tuples}
contains any mutable object either directly or indirectly, it cannot be
used as a key. * lists can not be used as keys.

It is best to think of a dictionary as a set of \emph{key: value} pairs,
with the requirement that the key are unique.

use \emph{del} to remove a key:value pair in a dictionary

create umpty dictionary: use \{\}

\begin{verbatim}
    * Performing list(dict) on a dictionary returns a list of all the keys used in the dict, in insertion order (插入顺序,dict的创建顺序). If you need it sorted, just use sorted(dict). 
    * To check whether a single key is in the dictionary, use the in keyword.
\end{verbatim}

Here is an example:

    \begin{Verbatim}[commandchars=\\\{\}]
{\color{incolor}In [{\color{incolor}35}]:} \PY{n}{tel} \PY{o}{=} \PY{p}{\PYZob{}}\PY{l+s+s1}{\PYZsq{}}\PY{l+s+s1}{jack}\PY{l+s+s1}{\PYZsq{}}\PY{p}{:} \PY{l+m+mi}{4098}\PY{p}{,} \PY{l+s+s1}{\PYZsq{}}\PY{l+s+s1}{sape}\PY{l+s+s1}{\PYZsq{}}\PY{p}{:} \PY{l+m+mi}{4139}\PY{p}{\PYZcb{}}
         \PY{n}{tel}\PY{p}{[}\PY{l+s+s1}{\PYZsq{}}\PY{l+s+s1}{guido}\PY{l+s+s1}{\PYZsq{}}\PY{p}{]} \PY{o}{=} \PY{l+m+mi}{4127}
         \PY{n}{tel}
\end{Verbatim}


\begin{Verbatim}[commandchars=\\\{\}]
{\color{outcolor}Out[{\color{outcolor}35}]:} \{'jack': 4098, 'sape': 4139, 'guido': 4127\}
\end{Verbatim}
            
    \begin{Verbatim}[commandchars=\\\{\}]
{\color{incolor}In [{\color{incolor}36}]:} \PY{k}{del} \PY{n}{tel}\PY{p}{[}\PY{l+s+s1}{\PYZsq{}}\PY{l+s+s1}{sape}\PY{l+s+s1}{\PYZsq{}}\PY{p}{]}
         \PY{n}{tel}\PY{p}{[}\PY{l+s+s1}{\PYZsq{}}\PY{l+s+s1}{irv}\PY{l+s+s1}{\PYZsq{}}\PY{p}{]} \PY{o}{=} \PY{l+m+mi}{4127}
         \PY{n}{tel}
\end{Verbatim}


\begin{Verbatim}[commandchars=\\\{\}]
{\color{outcolor}Out[{\color{outcolor}36}]:} \{'jack': 4098, 'guido': 4127, 'irv': 4127\}
\end{Verbatim}
            
    \begin{Verbatim}[commandchars=\\\{\}]
{\color{incolor}In [{\color{incolor}37}]:} \PY{n+nb}{sorted}\PY{p}{(}\PY{n}{tel}\PY{p}{)}
\end{Verbatim}


\begin{Verbatim}[commandchars=\\\{\}]
{\color{outcolor}Out[{\color{outcolor}37}]:} ['guido', 'irv', 'jack']
\end{Verbatim}
            
    \begin{Verbatim}[commandchars=\\\{\}]
{\color{incolor}In [{\color{incolor}38}]:} \PY{l+s+s1}{\PYZsq{}}\PY{l+s+s1}{guido}\PY{l+s+s1}{\PYZsq{}} \PY{o+ow}{in} \PY{n}{tel}
\end{Verbatim}


\begin{Verbatim}[commandchars=\\\{\}]
{\color{outcolor}Out[{\color{outcolor}38}]:} True
\end{Verbatim}
            
    \begin{Verbatim}[commandchars=\\\{\}]
{\color{incolor}In [{\color{incolor}39}]:} \PY{l+s+s1}{\PYZsq{}}\PY{l+s+s1}{jack}\PY{l+s+s1}{\PYZsq{}} \PY{o+ow}{not} \PY{o+ow}{in} \PY{n}{tel}
\end{Verbatim}


\begin{Verbatim}[commandchars=\\\{\}]
{\color{outcolor}Out[{\color{outcolor}39}]:} False
\end{Verbatim}
            
    \paragraph{dict() function construct builds dictionaries directly from
sequences of key-value
pairs:}\label{dict-function-construct-builds-dictionaries-directly-from-sequences-of-key-value-pairs}

    \begin{Verbatim}[commandchars=\\\{\}]
{\color{incolor}In [{\color{incolor}40}]:} \PY{n+nb}{dict}\PY{p}{(}\PY{p}{[}\PY{p}{(}\PY{l+s+s1}{\PYZsq{}}\PY{l+s+s1}{sape}\PY{l+s+s1}{\PYZsq{}}\PY{p}{,} \PY{l+m+mi}{4139}\PY{p}{)}\PY{p}{,} \PY{p}{(}\PY{l+s+s1}{\PYZsq{}}\PY{l+s+s1}{guido}\PY{l+s+s1}{\PYZsq{}}\PY{p}{,} \PY{l+m+mi}{4127}\PY{p}{)}\PY{p}{,} \PY{p}{(}\PY{l+s+s1}{\PYZsq{}}\PY{l+s+s1}{jack}\PY{l+s+s1}{\PYZsq{}}\PY{p}{,} \PY{l+m+mi}{4098}\PY{p}{)}\PY{p}{]}\PY{p}{)}    \PY{c+c1}{\PYZsh{} the pairs not necessarilly have to have a \PYZsq{}:\PYZsq{}}
\end{Verbatim}


\begin{Verbatim}[commandchars=\\\{\}]
{\color{outcolor}Out[{\color{outcolor}40}]:} \{'sape': 4139, 'guido': 4127, 'jack': 4098\}
\end{Verbatim}
            
    When the keys are simple strings, it is sometimes easier to specify
pairs using keyword arguments:

    \begin{Verbatim}[commandchars=\\\{\}]
{\color{incolor}In [{\color{incolor}42}]:} \PY{n+nb}{dict}\PY{p}{(}\PY{n}{sape}\PY{o}{=}\PY{l+m+mi}{4139}\PY{p}{,} \PY{n}{guido}\PY{o}{=}\PY{l+m+mi}{4127}\PY{p}{,} \PY{n}{jack}\PY{o}{=}\PY{l+m+mi}{4098}\PY{p}{)}
\end{Verbatim}


\begin{Verbatim}[commandchars=\\\{\}]
{\color{outcolor}Out[{\color{outcolor}42}]:} \{'sape': 4139, 'guido': 4127, 'jack': 4098\}
\end{Verbatim}
            
    \paragraph{dict comprehensions can be used to create dictionaries from
arbitrary key and value
expresssions:}\label{dict-comprehensions-can-be-used-to-create-dictionaries-from-arbitrary-key-and-value-expresssions}

    \begin{Verbatim}[commandchars=\\\{\}]
{\color{incolor}In [{\color{incolor}41}]:} \PY{p}{\PYZob{}}\PY{n}{x}\PY{p}{:} \PY{n}{x}\PY{o}{*}\PY{o}{*}\PY{l+m+mi}{2} \PY{k}{for} \PY{n}{x} \PY{o+ow}{in} \PY{p}{(}\PY{l+m+mi}{2}\PY{p}{,} \PY{l+m+mi}{3}\PY{p}{,} \PY{l+m+mi}{4}\PY{p}{,} \PY{l+m+mi}{5}\PY{p}{)}\PY{p}{\PYZcb{}}        \PY{c+c1}{\PYZsh{} 这里需要冒号,冒号前是key,冒号后是value。跟前边list是类似的。}
\end{Verbatim}


\begin{Verbatim}[commandchars=\\\{\}]
{\color{outcolor}Out[{\color{outcolor}41}]:} \{2: 4, 3: 9, 4: 16, 5: 25\}
\end{Verbatim}
            
    \subsubsection{Looping Techniques}\label{looping-techniques}

    \begin{enumerate}
\def\labelenumi{\arabic{enumi}.}
\tightlist
\item
  When looping through dictionaries, the key and corresponding value can
  be retrieved at the same time using the items() method.
\end{enumerate}

    \begin{Verbatim}[commandchars=\\\{\}]
{\color{incolor}In [{\color{incolor}43}]:} \PY{n}{knights} \PY{o}{=} \PY{p}{\PYZob{}}\PY{l+s+s1}{\PYZsq{}}\PY{l+s+s1}{gallahad}\PY{l+s+s1}{\PYZsq{}}\PY{p}{:} \PY{l+s+s1}{\PYZsq{}}\PY{l+s+s1}{the pure}\PY{l+s+s1}{\PYZsq{}}\PY{p}{,} \PY{l+s+s1}{\PYZsq{}}\PY{l+s+s1}{robin}\PY{l+s+s1}{\PYZsq{}}\PY{p}{:} \PY{l+s+s1}{\PYZsq{}}\PY{l+s+s1}{the brave}\PY{l+s+s1}{\PYZsq{}}\PY{p}{\PYZcb{}}
         \PY{k}{for} \PY{n}{k}\PY{p}{,} \PY{n}{v} \PY{o+ow}{in} \PY{n}{knights}\PY{o}{.}\PY{n}{items}\PY{p}{(}\PY{p}{)}\PY{p}{:}
             \PY{n+nb}{print}\PY{p}{(}\PY{n}{f}\PY{l+s+s1}{\PYZsq{}}\PY{l+s+si}{\PYZob{}k\PYZcb{}}\PY{l+s+s1}{: }\PY{l+s+si}{\PYZob{}v\PYZcb{}}\PY{l+s+s1}{\PYZsq{}}\PY{p}{)}
\end{Verbatim}


    \begin{Verbatim}[commandchars=\\\{\}]
gallahad: the pure
robin: the brave

    \end{Verbatim}

    \begin{enumerate}
\def\labelenumi{\arabic{enumi}.}
\setcounter{enumi}{1}
\tightlist
\item
  When looping through a sequence, the position index and corresponding
  value can be retrived at the same time using the enumerate() function.
\end{enumerate}

    \begin{Verbatim}[commandchars=\\\{\}]
{\color{incolor}In [{\color{incolor}1}]:} \PY{k}{for} \PY{n}{i}\PY{p}{,} \PY{n}{v} \PY{o+ow}{in} \PY{n+nb}{enumerate}\PY{p}{(}\PY{p}{[}\PY{l+s+s1}{\PYZsq{}}\PY{l+s+s1}{tic}\PY{l+s+s1}{\PYZsq{}}\PY{p}{,} \PY{l+s+s1}{\PYZsq{}}\PY{l+s+s1}{tac}\PY{l+s+s1}{\PYZsq{}}\PY{p}{,} \PY{l+s+s1}{\PYZsq{}}\PY{l+s+s1}{toe}\PY{l+s+s1}{\PYZsq{}}\PY{p}{]}\PY{p}{)}\PY{p}{:}
            \PY{n+nb}{print}\PY{p}{(}\PY{n}{i}\PY{p}{,} \PY{n}{v}\PY{p}{)}
\end{Verbatim}


    \begin{Verbatim}[commandchars=\\\{\}]
0 tic
1 tac
2 toe

    \end{Verbatim}

    \begin{enumerate}
\def\labelenumi{\arabic{enumi}.}
\setcounter{enumi}{2}
\tightlist
\item
  To loop over two or more sequences at the same time, the entries can
  be paired with the zip() function.
\end{enumerate}

    \begin{Verbatim}[commandchars=\\\{\}]
{\color{incolor}In [{\color{incolor}3}]:} \PY{n}{questions} \PY{o}{=} \PY{p}{[}\PY{l+s+s1}{\PYZsq{}}\PY{l+s+s1}{name}\PY{l+s+s1}{\PYZsq{}}\PY{p}{,} \PY{l+s+s1}{\PYZsq{}}\PY{l+s+s1}{quest}\PY{l+s+s1}{\PYZsq{}}\PY{p}{,} \PY{l+s+s1}{\PYZsq{}}\PY{l+s+s1}{favorite color}\PY{l+s+s1}{\PYZsq{}}\PY{p}{]}
        \PY{n}{answers} \PY{o}{=} \PY{p}{[}\PY{l+s+s1}{\PYZsq{}}\PY{l+s+s1}{lancelot}\PY{l+s+s1}{\PYZsq{}}\PY{p}{,} \PY{l+s+s1}{\PYZsq{}}\PY{l+s+s1}{the holy grail}\PY{l+s+s1}{\PYZsq{}}\PY{p}{,} \PY{l+s+s1}{\PYZsq{}}\PY{l+s+s1}{blue}\PY{l+s+s1}{\PYZsq{}}\PY{p}{]}
        \PY{k}{for} \PY{n}{q}\PY{p}{,} \PY{n}{a} \PY{o+ow}{in} \PY{n+nb}{zip}\PY{p}{(}\PY{n}{questions}\PY{p}{,} \PY{n}{answers}\PY{p}{)}\PY{p}{:}
            \PY{n+nb}{print}\PY{p}{(}\PY{l+s+s1}{\PYZsq{}}\PY{l+s+s1}{What is your }\PY{l+s+si}{\PYZob{}\PYZcb{}}\PY{l+s+s1}{? It is }\PY{l+s+si}{\PYZob{}\PYZcb{}}\PY{l+s+s1}{.}\PY{l+s+s1}{\PYZsq{}}\PY{o}{.}\PY{n}{format}\PY{p}{(}\PY{n}{q}\PY{p}{,} \PY{n}{a}\PY{p}{)}\PY{p}{)}
\end{Verbatim}


    \begin{Verbatim}[commandchars=\\\{\}]
What is your name? It is lancelot.
What is your quest? It is the holy grail.
What is your favorite color? It is blue.

    \end{Verbatim}

    \begin{enumerate}
\def\labelenumi{\arabic{enumi}.}
\setcounter{enumi}{3}
\tightlist
\item
  To loop over a sequence in reverse, first specify in a forword
  direction and then call the reversed() function.
\end{enumerate}

    \begin{Verbatim}[commandchars=\\\{\}]
{\color{incolor}In [{\color{incolor}4}]:} \PY{k}{for} \PY{n}{i} \PY{o+ow}{in} \PY{n+nb}{reversed}\PY{p}{(}\PY{n+nb}{range}\PY{p}{(}\PY{l+m+mi}{1}\PY{p}{,} \PY{l+m+mi}{10}\PY{p}{,} \PY{l+m+mi}{2}\PY{p}{)}\PY{p}{)}\PY{p}{:}
            \PY{n+nb}{print}\PY{p}{(}\PY{n}{i}\PY{p}{)}
\end{Verbatim}


    \begin{Verbatim}[commandchars=\\\{\}]
9
7
5
3
1

    \end{Verbatim}

    \begin{enumerate}
\def\labelenumi{\arabic{enumi}.}
\setcounter{enumi}{4}
\tightlist
\item
  To loop over a sequence in sorted order, use the sorted() function
  which returns a new sorted list while leaving the source unaltered.
\end{enumerate}

    \begin{Verbatim}[commandchars=\\\{\}]
{\color{incolor}In [{\color{incolor}5}]:} \PY{n}{basket} \PY{o}{=} \PY{p}{[}\PY{l+s+s1}{\PYZsq{}}\PY{l+s+s1}{apple}\PY{l+s+s1}{\PYZsq{}}\PY{p}{,} \PY{l+s+s1}{\PYZsq{}}\PY{l+s+s1}{orange}\PY{l+s+s1}{\PYZsq{}}\PY{p}{,} \PY{l+s+s1}{\PYZsq{}}\PY{l+s+s1}{apple}\PY{l+s+s1}{\PYZsq{}}\PY{p}{,} \PY{l+s+s1}{\PYZsq{}}\PY{l+s+s1}{pear}\PY{l+s+s1}{\PYZsq{}}\PY{p}{,} \PY{l+s+s1}{\PYZsq{}}\PY{l+s+s1}{orange}\PY{l+s+s1}{\PYZsq{}}\PY{p}{,} \PY{l+s+s1}{\PYZsq{}}\PY{l+s+s1}{banana}\PY{l+s+s1}{\PYZsq{}}\PY{p}{]}
        \PY{k}{for} \PY{n}{f} \PY{o+ow}{in} \PY{n+nb}{sorted}\PY{p}{(}\PY{n+nb}{set}\PY{p}{(}\PY{n}{basket}\PY{p}{)}\PY{p}{)}\PY{p}{:}    \PY{c+c1}{\PYZsh{} duplicate items are omitted.}
            \PY{n+nb}{print}\PY{p}{(}\PY{n}{f}\PY{p}{)}
\end{Verbatim}


    \begin{Verbatim}[commandchars=\\\{\}]
apple
banana
orange
pear

    \end{Verbatim}

    \begin{enumerate}
\def\labelenumi{\arabic{enumi}.}
\setcounter{enumi}{5}
\tightlist
\item
  It is sometimes tempting to change a list while you are looping over
  it; however, it is often simpler and safer to create a new list
  instead.
\end{enumerate}

    \begin{Verbatim}[commandchars=\\\{\}]
{\color{incolor}In [{\color{incolor}6}]:} \PY{k+kn}{import} \PY{n+nn}{math}
        \PY{n}{raw\PYZus{}data} \PY{o}{=} \PY{p}{[}\PY{l+m+mf}{56.2}\PY{p}{,} \PY{n+nb}{float}\PY{p}{(}\PY{l+s+s1}{\PYZsq{}}\PY{l+s+s1}{NaN}\PY{l+s+s1}{\PYZsq{}}\PY{p}{)}\PY{p}{,} \PY{l+m+mf}{51.7}\PY{p}{,} \PY{l+m+mf}{55.3}\PY{p}{,} \PY{l+m+mf}{52.5}\PY{p}{,} \PY{n+nb}{float}\PY{p}{(}\PY{l+s+s1}{\PYZsq{}}\PY{l+s+s1}{NaN}\PY{l+s+s1}{\PYZsq{}}\PY{p}{)}\PY{p}{,} \PY{l+m+mf}{47.8}\PY{p}{]}
        \PY{n}{filtered\PYZus{}data} \PY{o}{=} \PY{p}{[}\PY{p}{]}
        \PY{k}{for} \PY{n}{value} \PY{o+ow}{in} \PY{n}{raw\PYZus{}data}\PY{p}{:}
            \PY{k}{if} \PY{o+ow}{not} \PY{n}{math}\PY{o}{.}\PY{n}{isnan}\PY{p}{(}\PY{n}{value}\PY{p}{)}\PY{p}{:}
                \PY{n}{filtered\PYZus{}data}\PY{o}{.}\PY{n}{append}\PY{p}{(}\PY{n}{value}\PY{p}{)}
                
        \PY{n}{filtered\PYZus{}data}
\end{Verbatim}


\begin{Verbatim}[commandchars=\\\{\}]
{\color{outcolor}Out[{\color{outcolor}6}]:} [56.2, 51.7, 55.3, 52.5, 47.8]
\end{Verbatim}
            
    尝试写一写更简单的代码达到上边的目的:

    \begin{Verbatim}[commandchars=\\\{\}]
{\color{incolor}In [{\color{incolor}8}]:} \PY{k+kn}{import} \PY{n+nn}{math}
        \PY{n}{raw\PYZus{}data} \PY{o}{=} \PY{p}{[}\PY{l+m+mf}{56.2}\PY{p}{,} \PY{n+nb}{float}\PY{p}{(}\PY{l+s+s1}{\PYZsq{}}\PY{l+s+s1}{NaN}\PY{l+s+s1}{\PYZsq{}}\PY{p}{)}\PY{p}{,} \PY{l+m+mf}{51.7}\PY{p}{,} \PY{l+m+mf}{55.3}\PY{p}{,} \PY{l+m+mf}{52.5}\PY{p}{,} \PY{n+nb}{float}\PY{p}{(}\PY{l+s+s1}{\PYZsq{}}\PY{l+s+s1}{NaN}\PY{l+s+s1}{\PYZsq{}}\PY{p}{)}\PY{p}{,} \PY{l+m+mf}{47.8}\PY{p}{]}
        \PY{p}{[}\PY{n}{value} \PY{k}{for} \PY{n}{value} \PY{o+ow}{in} \PY{n}{raw\PYZus{}data} \PY{k}{if} \PY{o+ow}{not} \PY{n}{math}\PY{o}{.}\PY{n}{isnan}\PY{p}{(}\PY{n}{value}\PY{p}{)}\PY{p}{]}
\end{Verbatim}


\begin{Verbatim}[commandchars=\\\{\}]
{\color{outcolor}Out[{\color{outcolor}8}]:} [56.2, 51.7, 55.3, 52.5, 47.8]
\end{Verbatim}
            
    \subsubsection{More on conditions}\label{more-on-conditions}

    \begin{enumerate}
\def\labelenumi{\arabic{enumi}.}
\item
  The conditions used in \emph{while} and \emph{if} statements can
  contain any operators, not just comparisons.

  \begin{itemize}
  \tightlist
  \item
    comparison operators \emph{in} and \emph{not in} check whether a
    value occurs in a sequence.
  \item
    comparison operators \emph{is} and \emph{is not} compare whether two
    objects are really the same object; this only matters for mutable
    objects like lists.
  \end{itemize}
\end{enumerate}

    \subsection{Modules}\label{modules}

    Module is basically a file contains functions and definitions. It can be
imported to other modules or the main module to be used. The file name
is the module name.

    \begin{Verbatim}[commandchars=\\\{\}]
{\color{incolor}In [{\color{incolor}9}]:} \PY{c+c1}{\PYZsh{} Fibonacci numbers module}
        
        \PY{k}{def} \PY{n+nf}{fib}\PY{p}{(}\PY{n}{n}\PY{p}{)}\PY{p}{:}    \PY{c+c1}{\PYZsh{} Write Fibonacci series up to n}
            \PY{n}{a}\PY{p}{,} \PY{n}{b} \PY{o}{=} \PY{l+m+mi}{0}\PY{p}{,} \PY{l+m+mi}{1}
            \PY{k}{while} \PY{n}{a} \PY{o}{\PYZlt{}} \PY{n}{n}\PY{p}{:}
                \PY{n+nb}{print}\PY{p}{(}\PY{n}{a}\PY{p}{,} \PY{n}{end}\PY{o}{=}\PY{l+s+s1}{\PYZsq{}}\PY{l+s+s1}{, }\PY{l+s+s1}{\PYZsq{}}\PY{p}{)}
                \PY{n}{a}\PY{p}{,} \PY{n}{b} \PY{o}{=} \PY{n}{b}\PY{p}{,} \PY{n}{a}\PY{o}{+}\PY{n}{b}
            \PY{n+nb}{print}\PY{p}{(}\PY{p}{)}
        
        \PY{k}{def} \PY{n+nf}{fib2}\PY{p}{(}\PY{n}{n}\PY{p}{)}\PY{p}{:}    \PY{c+c1}{\PYZsh{} return Fibonacci series up to \PYZsh{} n}
            \PY{n}{a}\PY{p}{,} \PY{n}{b} \PY{o}{=} \PY{l+m+mi}{0}\PY{p}{,} \PY{l+m+mi}{1}
            \PY{n}{result} \PY{o}{=} \PY{p}{[}\PY{p}{]}
            \PY{k}{while} \PY{n}{a} \PY{o}{\PYZlt{}} \PY{n}{n}\PY{p}{:}
                \PY{n}{result}\PY{o}{.}\PY{n}{append}\PY{p}{(}\PY{n}{a}\PY{p}{)}
                \PY{n}{a}\PY{p}{,} \PY{n}{b} \PY{o}{=} \PY{n}{b}\PY{p}{,} \PY{n}{a}\PY{o}{+}\PY{n}{b}
            \PY{k}{return} \PY{n}{result}
\end{Verbatim}


    Way to import a module:

    \begin{Verbatim}[commandchars=\\\{\}]
{\color{incolor}In [{\color{incolor}10}]:} \PY{k+kn}{import} \PY{n+nn}{fibo}    \PY{c+c1}{\PYZsh{} the module name will be available in the symbol table.}
\end{Verbatim}


    Way to use a function from an imported module

    \begin{Verbatim}[commandchars=\\\{\}]
{\color{incolor}In [{\color{incolor}11}]:} \PY{n}{fibo}\PY{o}{.}\PY{n}{fib}\PY{p}{(}\PY{l+m+mi}{100}\PY{p}{)}
\end{Verbatim}


    \begin{Verbatim}[commandchars=\\\{\}]
0, 1, 1, 2, 3, 5, 8, 13, 21, 34, 55, 89, 

    \end{Verbatim}

    The module name can be accessed in the following way:

    \begin{Verbatim}[commandchars=\\\{\}]
{\color{incolor}In [{\color{incolor}12}]:} \PY{n}{fibo}\PY{o}{.}\PY{n+nv+vm}{\PYZus{}\PYZus{}name\PYZus{}\PYZus{}}
\end{Verbatim}


\begin{Verbatim}[commandchars=\\\{\}]
{\color{outcolor}Out[{\color{outcolor}12}]:} 'fibo'
\end{Verbatim}
            
    import functinos from a module directly: so no need to write the module
name when calling the function.

    \begin{Verbatim}[commandchars=\\\{\}]
{\color{incolor}In [{\color{incolor}13}]:} \PY{k+kn}{from} \PY{n+nn}{fibo} \PY{k}{import} \PY{n}{fib}\PY{p}{,} \PY{n}{fib2}    \PY{c+c1}{\PYZsh{} here fib, fib2 are functions, we\PYZsq{}ll talk about packages and modules later.}
         \PY{n}{fib2}\PY{p}{(}\PY{l+m+mi}{100}\PY{p}{)}
\end{Verbatim}


\begin{Verbatim}[commandchars=\\\{\}]
{\color{outcolor}Out[{\color{outcolor}13}]:} [0, 1, 1, 2, 3, 5, 8, 13, 21, 34, 55, 89]
\end{Verbatim}
            
    Import all functions from a module:

    \begin{Verbatim}[commandchars=\\\{\}]
{\color{incolor}In [{\color{incolor}15}]:} \PY{k+kn}{from} \PY{n+nn}{fibo} \PY{k}{import} \PY{o}{*}
\end{Verbatim}


    import a module and change its name:

    \begin{Verbatim}[commandchars=\\\{\}]
{\color{incolor}In [{\color{incolor}16}]:} \PY{k+kn}{import} \PY{n+nn}{fibo} \PY{k}{as} \PY{n+nn}{fib}
         \PY{n}{fib}\PY{o}{.}\PY{n}{fib2}\PY{p}{(}\PY{l+m+mi}{100}\PY{p}{)}
\end{Verbatim}


\begin{Verbatim}[commandchars=\\\{\}]
{\color{outcolor}Out[{\color{outcolor}16}]:} [0, 1, 1, 2, 3, 5, 8, 13, 21, 34, 55, 89]
\end{Verbatim}
            
    import a functino and change its name:

    \begin{Verbatim}[commandchars=\\\{\}]
{\color{incolor}In [{\color{incolor}17}]:} \PY{k+kn}{from} \PY{n+nn}{fibo} \PY{k}{import} \PY{n}{fib} \PY{k}{as} \PY{n}{fibonacci}
         \PY{n}{fibonacci}\PY{p}{(}\PY{l+m+mi}{200}\PY{p}{)}
\end{Verbatim}


    \begin{Verbatim}[commandchars=\\\{\}]
0, 1, 1, 2, 3, 5, 8, 13, 21, 34, 55, 89, 144, 

    \end{Verbatim}

    \paragraph{Executing modules as
scripts}\label{executing-modules-as-scripts}

    Add the following lines at the end of the module file:

    \begin{Verbatim}[commandchars=\\\{\}]
{\color{incolor}In [{\color{incolor}18}]:} \PY{k}{if} \PY{n+nv+vm}{\PYZus{}\PYZus{}name\PYZus{}\PYZus{}} \PY{o}{==} \PY{l+s+s2}{\PYZdq{}}\PY{l+s+s2}{\PYZus{}\PYZus{}main\PYZus{}\PYZus{}}\PY{l+s+s2}{\PYZdq{}}\PY{p}{:}
             \PY{k+kn}{import} \PY{n+nn}{sys}
             \PY{n}{fib}\PY{p}{(}\PY{n+nb}{int}\PY{p}{(}\PY{n}{sys}\PY{o}{.}\PY{n}{argv}\PY{p}{[}\PY{l+m+mi}{1}\PY{p}{]}\PY{p}{)}
\end{Verbatim}


    \begin{Verbatim}[commandchars=\\\{\}]

          File "<ipython-input-18-c20df94c9865>", line 3
        fib(int(sys.argv[1])
                            \^{}
    SyntaxError: unexpected EOF while parsing


    \end{Verbatim}

    This module can be used as a scirpt (in the method shown below) as well
as an importable module.

    python fibo.py 

    Running module as a script is useful for testing.

    The followings are module search path, complied python files, would need
to know them for now.

    \subsubsection{Standard Modules}\label{standard-modules}

    Python comes with a library of standart modules, description is in the
Python library reference.

Some modules are built into the interpreter; these provide access to
operations that are not part of the core of the language but are
nevertheless built in.

Some modules also depend on the platform, i.e. windows/linux.

One particular deserves some attention: sys, which is built into every
Python interpreter. The variables sys.ps1 and sys.ps2 define the strings
used as primary and secondary prompts, it can be changed accordingly:

This two variables are only defined if the interpreter is in interactive
mode.

    \begin{Verbatim}[commandchars=\\\{\}]
{\color{incolor}In [{\color{incolor}19}]:} \PY{k+kn}{import} \PY{n+nn}{sys}
         \PY{n}{sys}\PY{o}{.}\PY{n}{ps1}
\end{Verbatim}


\begin{Verbatim}[commandchars=\\\{\}]
{\color{outcolor}Out[{\color{outcolor}19}]:} 'In : '
\end{Verbatim}
            
    \begin{Verbatim}[commandchars=\\\{\}]
{\color{incolor}In [{\color{incolor}20}]:} \PY{n}{sys}\PY{o}{.}\PY{n}{ps2}
\end{Verbatim}


\begin{Verbatim}[commandchars=\\\{\}]
{\color{outcolor}Out[{\color{outcolor}20}]:} '{\ldots}: '
\end{Verbatim}
            
    The variable sys.path is a list of strings that determines the
interpreter's search path for modules. It is initialized to a default
path taken from the environment variable PYTHONPATH, or from a built-in
default if PYTHONPATH is not set. It can be modified using standard list
operations:

    \begin{Verbatim}[commandchars=\\\{\}]
{\color{incolor}In [{\color{incolor}21}]:} \PY{k+kn}{import} \PY{n+nn}{sys}
         \PY{n}{sys}\PY{o}{.}\PY{n}{path}\PY{o}{.}\PY{n}{append}\PY{p}{(}\PY{l+s+s1}{\PYZsq{}}\PY{l+s+s1}{\PYZti{}\PYZti{}\PYZti{}\PYZti{}\PYZti{}\PYZti{}}\PY{l+s+s1}{\PYZsq{}}\PY{p}{)}
\end{Verbatim}


    \subsubsection{The dir() function}\label{the-dir-function}

    The built-in function dir() is used to find out which names a module
defines. It returns a sorted list of strings:

    \begin{Verbatim}[commandchars=\\\{\}]
{\color{incolor}In [{\color{incolor}22}]:} \PY{k+kn}{import} \PY{n+nn}{sys}
         \PY{n+nb}{dir}\PY{p}{(}\PY{n}{sys}\PY{p}{)}
\end{Verbatim}


\begin{Verbatim}[commandchars=\\\{\}]
{\color{outcolor}Out[{\color{outcolor}22}]:} ['\_\_displayhook\_\_',
          '\_\_doc\_\_',
          '\_\_excepthook\_\_',
          '\_\_interactivehook\_\_',
          '\_\_loader\_\_',
          '\_\_name\_\_',
          '\_\_package\_\_',
          '\_\_spec\_\_',
          '\_\_stderr\_\_',
          '\_\_stdin\_\_',
          '\_\_stdout\_\_',
          '\_clear\_type\_cache',
          '\_current\_frames',
          '\_debugmallocstats',
          '\_getframe',
          '\_git',
          '\_home',
          '\_xoptions',
          'abiflags',
          'api\_version',
          'argv',
          'base\_exec\_prefix',
          'base\_prefix',
          'builtin\_module\_names',
          'byteorder',
          'call\_tracing',
          'callstats',
          'copyright',
          'displayhook',
          'dont\_write\_bytecode',
          'exc\_info',
          'excepthook',
          'exec\_prefix',
          'executable',
          'exit',
          'flags',
          'float\_info',
          'float\_repr\_style',
          'get\_asyncgen\_hooks',
          'get\_coroutine\_wrapper',
          'getallocatedblocks',
          'getcheckinterval',
          'getdefaultencoding',
          'getdlopenflags',
          'getfilesystemencodeerrors',
          'getfilesystemencoding',
          'getprofile',
          'getrecursionlimit',
          'getrefcount',
          'getsizeof',
          'getswitchinterval',
          'gettrace',
          'hash\_info',
          'hexversion',
          'implementation',
          'int\_info',
          'intern',
          'is\_finalizing',
          'last\_traceback',
          'last\_type',
          'last\_value',
          'maxsize',
          'maxunicode',
          'meta\_path',
          'modules',
          'path',
          'path\_hooks',
          'path\_importer\_cache',
          'platform',
          'prefix',
          'ps1',
          'ps2',
          'ps3',
          'set\_asyncgen\_hooks',
          'set\_coroutine\_wrapper',
          'setcheckinterval',
          'setdlopenflags',
          'setprofile',
          'setrecursionlimit',
          'setswitchinterval',
          'settrace',
          'stderr',
          'stdin',
          'stdout',
          'thread\_info',
          'version',
          'version\_info',
          'warnoptions']
\end{Verbatim}
            
    Without arguments, dir() lists the names you have defined currently:

It lists all types of names, modules, functions; excluding names of
built-in functions and variables, they are defined in the standard
module builtins.

    \begin{Verbatim}[commandchars=\\\{\}]
{\color{incolor}In [{\color{incolor}23}]:} \PY{n+nb}{dir}\PY{p}{(}\PY{p}{)}
\end{Verbatim}


\begin{Verbatim}[commandchars=\\\{\}]
{\color{outcolor}Out[{\color{outcolor}23}]:} ['In',
          'Out',
          '\_',
          '\_12',
          '\_13',
          '\_16',
          '\_19',
          '\_20',
          '\_22',
          '\_6',
          '\_8',
          '\_\_',
          '\_\_\_',
          '\_\_builtin\_\_',
          '\_\_builtins\_\_',
          '\_\_doc\_\_',
          '\_\_loader\_\_',
          '\_\_name\_\_',
          '\_\_package\_\_',
          '\_\_spec\_\_',
          '\_dh',
          '\_i',
          '\_i1',
          '\_i10',
          '\_i11',
          '\_i12',
          '\_i13',
          '\_i14',
          '\_i15',
          '\_i16',
          '\_i17',
          '\_i18',
          '\_i19',
          '\_i2',
          '\_i20',
          '\_i21',
          '\_i22',
          '\_i23',
          '\_i3',
          '\_i4',
          '\_i5',
          '\_i6',
          '\_i7',
          '\_i8',
          '\_i9',
          '\_ih',
          '\_ii',
          '\_iii',
          '\_oh',
          'a',
          'answers',
          'basket',
          'exit',
          'f',
          'fib',
          'fib2',
          'fibo',
          'fibonacci',
          'filtered\_data',
          'get\_ipython',
          'i',
          'math',
          'q',
          'questions',
          'quit',
          'raw\_data',
          'sys',
          'v',
          'value']
\end{Verbatim}
            
    \begin{Verbatim}[commandchars=\\\{\}]
{\color{incolor}In [{\color{incolor}24}]:} \PY{k+kn}{import} \PY{n+nn}{builtins}
         \PY{n+nb}{dir}\PY{p}{(}\PY{n}{builtins}\PY{p}{)}
\end{Verbatim}


\begin{Verbatim}[commandchars=\\\{\}]
{\color{outcolor}Out[{\color{outcolor}24}]:} ['ArithmeticError',
          'AssertionError',
          'AttributeError',
          'BaseException',
          'BlockingIOError',
          'BrokenPipeError',
          'BufferError',
          'BytesWarning',
          'ChildProcessError',
          'ConnectionAbortedError',
          'ConnectionError',
          'ConnectionRefusedError',
          'ConnectionResetError',
          'DeprecationWarning',
          'EOFError',
          'Ellipsis',
          'EnvironmentError',
          'Exception',
          'False',
          'FileExistsError',
          'FileNotFoundError',
          'FloatingPointError',
          'FutureWarning',
          'GeneratorExit',
          'IOError',
          'ImportError',
          'ImportWarning',
          'IndentationError',
          'IndexError',
          'InterruptedError',
          'IsADirectoryError',
          'KeyError',
          'KeyboardInterrupt',
          'LookupError',
          'MemoryError',
          'ModuleNotFoundError',
          'NameError',
          'None',
          'NotADirectoryError',
          'NotImplemented',
          'NotImplementedError',
          'OSError',
          'OverflowError',
          'PendingDeprecationWarning',
          'PermissionError',
          'ProcessLookupError',
          'RecursionError',
          'ReferenceError',
          'ResourceWarning',
          'RuntimeError',
          'RuntimeWarning',
          'StopAsyncIteration',
          'StopIteration',
          'SyntaxError',
          'SyntaxWarning',
          'SystemError',
          'SystemExit',
          'TabError',
          'TimeoutError',
          'True',
          'TypeError',
          'UnboundLocalError',
          'UnicodeDecodeError',
          'UnicodeEncodeError',
          'UnicodeError',
          'UnicodeTranslateError',
          'UnicodeWarning',
          'UserWarning',
          'ValueError',
          'Warning',
          'ZeroDivisionError',
          '\_\_IPYTHON\_\_',
          '\_\_build\_class\_\_',
          '\_\_debug\_\_',
          '\_\_doc\_\_',
          '\_\_import\_\_',
          '\_\_loader\_\_',
          '\_\_name\_\_',
          '\_\_package\_\_',
          '\_\_spec\_\_',
          'abs',
          'all',
          'any',
          'ascii',
          'bin',
          'bool',
          'bytearray',
          'bytes',
          'callable',
          'chr',
          'classmethod',
          'compile',
          'complex',
          'copyright',
          'credits',
          'delattr',
          'dict',
          'dir',
          'display',
          'divmod',
          'enumerate',
          'eval',
          'exec',
          'filter',
          'float',
          'format',
          'frozenset',
          'get\_ipython',
          'getattr',
          'globals',
          'hasattr',
          'hash',
          'help',
          'hex',
          'id',
          'input',
          'int',
          'isinstance',
          'issubclass',
          'iter',
          'len',
          'license',
          'list',
          'locals',
          'map',
          'max',
          'memoryview',
          'min',
          'next',
          'object',
          'oct',
          'open',
          'ord',
          'pow',
          'print',
          'property',
          'range',
          'repr',
          'reversed',
          'round',
          'set',
          'setattr',
          'slice',
          'sorted',
          'staticmethod',
          'str',
          'sum',
          'super',
          'tuple',
          'type',
          'vars',
          'zip']
\end{Verbatim}
            
    \subsubsection{Packages}\label{packages}

    Package is a collection of modules. An example:
sound/                          Top-level package
      __init__.py               Initialize the sound package
      formats/                  Subpackage for file format conversions
              __init__.py
              wavread.py
              wavwrite.py
              aiffread.py
              aiffwrite.py
              auread.py
              auwrite.py
              ...
      effects/                  Subpackage for sound effects
              __init__.py
              echo.py
              surround.py
              reverse.py
              ...
      filters/                  Subpackage for filters
              __init__.py
              equalizer.py
              vocoder.py
              karaoke.py
              ...
    When importing a package, Python searches through the directories on
sys.path looking for the package subdirectory.

The \textbf{init}.py files are required to make Python treat the
directories as containing packages; this is done to prevent directories
with a common name, such as string, from unintentionally hiding valid
modules that occur later on the module search path. In the simplest
case, \textbf{init}.py can just be an empty file, but it can also
execute initialization code for the package or set the \textbf{all}
variable, described later.

    Individual modules can be imported from a package

    \begin{Verbatim}[commandchars=\\\{\}]
{\color{incolor}In [{\color{incolor}25}]:} \PY{k+kn}{import} \PY{n+nn}{sound}\PY{n+nn}{.}\PY{n+nn}{effects}\PY{n+nn}{.}\PY{n+nn}{echo}    \PY{c+c1}{\PYZsh{} just an example, not a real package. This must be referenced with a full name.}
\end{Verbatim}


    \begin{Verbatim}[commandchars=\\\{\}]

        ---------------------------------------------------------------------------

        ModuleNotFoundError                       Traceback (most recent call last)

        <ipython-input-25-fc07412d5a70> in <module>()
    ----> 1 import sound.effects.echo
    

        ModuleNotFoundError: No module named 'sound'

    \end{Verbatim}

    another way to import the echo module: this way we don't need to write
the full name when calling it (talked about functions before similar to
this.)

    \begin{Verbatim}[commandchars=\\\{\}]
{\color{incolor}In [{\color{incolor}26}]:} \PY{k+kn}{from} \PY{n+nn}{sound}\PY{n+nn}{.}\PY{n+nn}{effects} \PY{k}{import} \PY{n}{echo}    \PY{c+c1}{\PYZsh{} here echo is a module. echo is able to be referenced without the full name.}
\end{Verbatim}


    \begin{Verbatim}[commandchars=\\\{\}]

        ---------------------------------------------------------------------------

        ModuleNotFoundError                       Traceback (most recent call last)

        <ipython-input-26-4c42e49b3ae5> in <module>()
    ----> 1 from sound.effects import echo
    

        ModuleNotFoundError: No module named 'sound'

    \end{Verbatim}

    Also possible to import a function from a module in a package directly:

    \begin{Verbatim}[commandchars=\\\{\}]
{\color{incolor}In [{\color{incolor}27}]:} \PY{k+kn}{from} \PY{n+nn}{sound}\PY{n+nn}{.}\PY{n+nn}{effects}\PY{n+nn}{.}\PY{n+nn}{echo} \PY{k}{import} \PY{n}{echofilter}    \PY{c+c1}{\PYZsh{} echofilter is a function, reference it directly without full name.}
\end{Verbatim}


    \begin{Verbatim}[commandchars=\\\{\}]

        ---------------------------------------------------------------------------

        ModuleNotFoundError                       Traceback (most recent call last)

        <ipython-input-27-c0f5183bc1f1> in <module>()
    ----> 1 from sound.effects.echo import echofilter    \# echofilter is a function, reference it directly without full name.
    

        ModuleNotFoundError: No module named 'sound'

    \end{Verbatim}

    Contrarily, when using syntax like \textbf{import
item.subitem.subsubitem}, each item except for the last must be a
package; the last item can be a module or a package but can't be a class
or function or variable defined in the previous item.

    \paragraph{Importing form a package}\label{importing-form-a-package}

好复杂 不想看

    \paragraph{Intra-package References}\label{intra-package-references}

不看不看

    \paragraph{packages in mltiple
directories}\label{packages-in-mltiple-directories}

还是不看。

    \subsection{Input and Output}\label{input-and-output}

    \subsubsection{Fancier Output
Formatting}\label{fancier-output-formatting}

    \begin{enumerate}
\def\labelenumi{\arabic{enumi}.}
\tightlist
\item
  formatted string literals:
\end{enumerate}

\begin{itemize}
\tightlist
\item
  begin a string with f or F before the opeing quotation mark or triple
  quotation mark. Inside the string, wirte a Python expression between
  \{ and \} that can refer to variables or literal values.
\end{itemize}

    \begin{Verbatim}[commandchars=\\\{\}]
{\color{incolor}In [{\color{incolor}28}]:} \PY{n}{year} \PY{o}{=} \PY{l+m+mi}{2016}
         \PY{n}{event} \PY{o}{=} \PY{l+s+s1}{\PYZsq{}}\PY{l+s+s1}{Referendum}\PY{l+s+s1}{\PYZsq{}}
         \PY{n}{f}\PY{l+s+s1}{\PYZsq{}}\PY{l+s+s1}{Results of the }\PY{l+s+si}{\PYZob{}year\PYZcb{}}\PY{l+s+s1}{ }\PY{l+s+si}{\PYZob{}event\PYZcb{}}\PY{l+s+s1}{\PYZsq{}}
\end{Verbatim}


\begin{Verbatim}[commandchars=\\\{\}]
{\color{outcolor}Out[{\color{outcolor}28}]:} 'Results of the 2016 Referendum'
\end{Verbatim}
            
    \begin{itemize}
\tightlist
\item
  str.format() method needs more manual effort. Still use \{ and \} to
  mark where a variable will be substituted and can provide detailed
  formatting directives, but you'll also need to provide the information
  to be formatted.
\end{itemize}

我理解是\{\} 花括号中间的字符是控制输出格式的命令 嗯
肯定是。此处应该看string.format(\emph{args, *}kwargs)

    The string on which this method is called can contain literal text or
replacement fields delimited by braces \{\}. Each replacement field
contains either \textbf{the numeric index} of a positional argument, or
\textbf{the name of a keyword argument}, also check \textbf{Format
String Syntax} for the details.

    \begin{Verbatim}[commandchars=\\\{\}]
{\color{incolor}In [{\color{incolor}29}]:} \PY{n}{yes\PYZus{}votes} \PY{o}{=} \PY{l+m+mi}{42}\PY{n}{\PYZus{}572\PYZus{}654}
         \PY{n}{no\PYZus{}votes} \PY{o}{=} \PY{l+m+mi}{43}\PY{n}{\PYZus{}132\PYZus{}495}
         \PY{n}{percentage} \PY{o}{=} \PY{n}{yes\PYZus{}votes} \PY{o}{/} \PY{p}{(}\PY{n}{yes\PYZus{}votes} \PY{o}{+} \PY{n}{no\PYZus{}votes}\PY{p}{)}
         \PY{l+s+s1}{\PYZsq{}}\PY{l+s+si}{\PYZob{}:\PYZhy{}9\PYZcb{}}\PY{l+s+s1}{ YES votes  }\PY{l+s+si}{\PYZob{}:2.2\PYZpc{}\PYZcb{}}\PY{l+s+s1}{\PYZsq{}}\PY{o}{.}\PY{n}{format}\PY{p}{(}\PY{n}{yes\PYZus{}votes}\PY{p}{,} \PY{n}{percentage}\PY{p}{)}
\end{Verbatim}


\begin{Verbatim}[commandchars=\\\{\}]
{\color{outcolor}Out[{\color{outcolor}29}]:} ' 42572654 YES votes  49.67\%'
\end{Verbatim}
            
    \begin{itemize}
\tightlist
\item
  still can use string slicing and concatenation operations.
\end{itemize}

    For a quick display of some variables for debugging purposes, any value
can be converted to a string with the \emph{repr()} or \emph{str()}
functions.

\begin{itemize}
\tightlist
\item
  str(): human-readable
\item
  repr(): generate representations which can be read by the interpreter.
\end{itemize}

    \begin{Verbatim}[commandchars=\\\{\}]
{\color{incolor}In [{\color{incolor}30}]:} \PY{n}{s} \PY{o}{=} \PY{l+s+s1}{\PYZsq{}}\PY{l+s+s1}{Hello, world.}\PY{l+s+s1}{\PYZsq{}}
         \PY{n+nb}{str}\PY{p}{(}\PY{n}{s}\PY{p}{)}
\end{Verbatim}


\begin{Verbatim}[commandchars=\\\{\}]
{\color{outcolor}Out[{\color{outcolor}30}]:} 'Hello, world.'
\end{Verbatim}
            
    \begin{Verbatim}[commandchars=\\\{\}]
{\color{incolor}In [{\color{incolor}31}]:} \PY{n+nb}{repr}\PY{p}{(}\PY{n}{s}\PY{p}{)}    \PY{c+c1}{\PYZsh{} 注意跟上边的区别,此处多了一个双引号,因为repr()的输出是给编译器看的。}
\end{Verbatim}


\begin{Verbatim}[commandchars=\\\{\}]
{\color{outcolor}Out[{\color{outcolor}31}]:} "'Hello, world.'"
\end{Verbatim}
            
    \begin{Verbatim}[commandchars=\\\{\}]
{\color{incolor}In [{\color{incolor}32}]:} \PY{n+nb}{str}\PY{p}{(}\PY{l+m+mi}{1}\PY{o}{/}\PY{l+m+mi}{7}\PY{p}{)}
\end{Verbatim}


\begin{Verbatim}[commandchars=\\\{\}]
{\color{outcolor}Out[{\color{outcolor}32}]:} '0.14285714285714285'
\end{Verbatim}
            
    \begin{Verbatim}[commandchars=\\\{\}]
{\color{incolor}In [{\color{incolor}33}]:} \PY{n+nb}{repr}\PY{p}{(}\PY{l+m+mi}{1}\PY{o}{/}\PY{l+m+mi}{7}\PY{p}{)}
\end{Verbatim}


\begin{Verbatim}[commandchars=\\\{\}]
{\color{outcolor}Out[{\color{outcolor}33}]:} '0.14285714285714285'
\end{Verbatim}
            
    \begin{Verbatim}[commandchars=\\\{\}]
{\color{incolor}In [{\color{incolor}35}]:} \PY{n}{x} \PY{o}{=} \PY{l+m+mi}{10} \PY{o}{*} \PY{l+m+mf}{3.25}
         \PY{n}{y} \PY{o}{=} \PY{l+m+mi}{200} \PY{o}{*} \PY{l+m+mi}{200}
         \PY{n}{s} \PY{o}{=} \PY{l+s+s1}{\PYZsq{}}\PY{l+s+s1}{The value of x is }\PY{l+s+s1}{\PYZsq{}} \PY{o}{+} \PY{n+nb}{repr}\PY{p}{(}\PY{n}{x}\PY{p}{)} \PY{o}{+} \PY{l+s+s1}{\PYZsq{}}\PY{l+s+s1}{, and y is }\PY{l+s+s1}{\PYZsq{}} \PY{o}{+} \PY{n+nb}{repr}\PY{p}{(}\PY{n}{y}\PY{p}{)} \PY{o}{+} \PY{l+s+s1}{\PYZsq{}}\PY{l+s+s1}{...}\PY{l+s+s1}{\PYZsq{}}
         \PY{n+nb}{print}\PY{p}{(}\PY{n}{s}\PY{p}{)}
\end{Verbatim}


    \begin{Verbatim}[commandchars=\\\{\}]
The value of x is 32.5, and y is 40000{\ldots}

    \end{Verbatim}

    \begin{Verbatim}[commandchars=\\\{\}]
{\color{incolor}In [{\color{incolor}36}]:} \PY{n}{ss} \PY{o}{=} \PY{l+s+s1}{\PYZsq{}}\PY{l+s+s1}{The value of x is }\PY{l+s+s1}{\PYZsq{}} \PY{o}{+} \PY{n+nb}{str}\PY{p}{(}\PY{n}{x}\PY{p}{)} \PY{o}{+} \PY{l+s+s1}{\PYZsq{}}\PY{l+s+s1}{, and y is }\PY{l+s+s1}{\PYZsq{}} \PY{o}{+} \PY{n+nb}{str}\PY{p}{(}\PY{n}{y}\PY{p}{)} \PY{o}{+} \PY{l+s+s1}{\PYZsq{}}\PY{l+s+s1}{...}\PY{l+s+s1}{\PYZsq{}}
         \PY{n+nb}{print}\PY{p}{(}\PY{n}{ss}\PY{p}{)}
\end{Verbatim}


    \begin{Verbatim}[commandchars=\\\{\}]
The value of x is 32.5, and y is 40000{\ldots}

    \end{Verbatim}

    \begin{Verbatim}[commandchars=\\\{\}]
{\color{incolor}In [{\color{incolor}38}]:} \PY{n}{hello} \PY{o}{=} \PY{l+s+s1}{\PYZsq{}}\PY{l+s+s1}{hello world}\PY{l+s+se}{\PYZbs{}n}\PY{l+s+s1}{\PYZsq{}}
         \PY{n+nb}{print}\PY{p}{(}\PY{n}{hello}\PY{p}{)}
         \PY{n+nb}{print}\PY{p}{(}\PY{n+nb}{repr}\PY{p}{(}\PY{n}{hello}\PY{p}{)}\PY{p}{)}
         \PY{n+nb}{print}\PY{p}{(}\PY{n+nb}{str}\PY{p}{(}\PY{n}{hello}\PY{p}{)}\PY{p}{)}    \PY{c+c1}{\PYZsh{}跟直接print(hello)效果一样}
         \PY{n+nb}{print}\PY{p}{(}\PY{l+s+s1}{\PYZsq{}}\PY{l+s+s1}{*}\PY{l+s+s1}{\PYZsq{}} \PY{o}{*} \PY{l+m+mi}{10}\PY{p}{)}
\end{Verbatim}


    \begin{Verbatim}[commandchars=\\\{\}]
hello world

'hello world\textbackslash{}n'
hello world

**********

    \end{Verbatim}

    \begin{Verbatim}[commandchars=\\\{\}]
{\color{incolor}In [{\color{incolor}39}]:} \PY{n+nb}{repr}\PY{p}{(}\PY{p}{(}\PY{n}{x}\PY{p}{,} \PY{n}{y}\PY{p}{,} \PY{p}{(}\PY{l+s+s1}{\PYZsq{}}\PY{l+s+s1}{spam}\PY{l+s+s1}{\PYZsq{}}\PY{p}{,} \PY{l+s+s1}{\PYZsq{}}\PY{l+s+s1}{eggs}\PY{l+s+s1}{\PYZsq{}}\PY{p}{)}\PY{p}{)}\PY{p}{)}    \PY{c+c1}{\PYZsh{} the argument to repr() can be any python object}
\end{Verbatim}


\begin{Verbatim}[commandchars=\\\{\}]
{\color{outcolor}Out[{\color{outcolor}39}]:} "(32.5, 40000, ('spam', 'eggs'))"
\end{Verbatim}
            
    \paragraph{More on Formatted String
Literals}\label{more-on-formatted-string-literals}

    optional format specifier: allows greater control over how the value is
formatted.

The following example rounds pi to three places ater the decimal.

    \begin{Verbatim}[commandchars=\\\{\}]
{\color{incolor}In [{\color{incolor}40}]:} \PY{k+kn}{import} \PY{n+nn}{math}
         \PY{n+nb}{print}\PY{p}{(}\PY{n}{f}\PY{l+s+s1}{\PYZsq{}}\PY{l+s+s1}{The value of pi is approximately }\PY{l+s+si}{\PYZob{}math.pi:.3f\PYZcb{}}\PY{l+s+s1}{\PYZsq{}}\PY{p}{)}
\end{Verbatim}


    \begin{Verbatim}[commandchars=\\\{\}]
The value of pi is approximately 3.142

    \end{Verbatim}

    Passing an integer after the ':' will cause that field to be a minimum
number of characters wide. This is useful for making columns line up.

只有一个数字:字符靠左,空格靠右;

数字d:字符靠右,空格靠左。

    \begin{Verbatim}[commandchars=\\\{\}]
{\color{incolor}In [{\color{incolor}41}]:} \PY{n}{table} \PY{o}{=} \PY{p}{\PYZob{}}\PY{l+s+s1}{\PYZsq{}}\PY{l+s+s1}{Sjoerd}\PY{l+s+s1}{\PYZsq{}}\PY{p}{:} \PY{l+m+mi}{4127}\PY{p}{,} \PY{l+s+s1}{\PYZsq{}}\PY{l+s+s1}{Jack}\PY{l+s+s1}{\PYZsq{}}\PY{p}{:} \PY{l+m+mi}{4098}\PY{p}{,} \PY{l+s+s1}{\PYZsq{}}\PY{l+s+s1}{Dcab}\PY{l+s+s1}{\PYZsq{}}\PY{p}{:} \PY{l+m+mi}{7678}\PY{p}{\PYZcb{}}
         \PY{k}{for} \PY{n}{name}\PY{p}{,} \PY{n}{phone} \PY{o+ow}{in} \PY{n}{table}\PY{o}{.}\PY{n}{items}\PY{p}{(}\PY{p}{)}\PY{p}{:}
             \PY{n+nb}{print}\PY{p}{(}\PY{n}{f}\PY{l+s+s1}{\PYZsq{}}\PY{l+s+si}{\PYZob{}name:10\PYZcb{}}\PY{l+s+s1}{ ==\PYZgt{} }\PY{l+s+si}{\PYZob{}phone:10d\PYZcb{}}\PY{l+s+s1}{\PYZsq{}}\PY{p}{)}    
\end{Verbatim}


    \begin{Verbatim}[commandchars=\\\{\}]
Sjoerd     ==>       4127
Jack       ==>       4098
Dcab       ==>       7678

    \end{Verbatim}

    Other modifiers can be used to convet the value before it is formatted:
'!a' applies ascii(), '!s' applies str(), '!r' applies repr():

    \begin{Verbatim}[commandchars=\\\{\}]
{\color{incolor}In [{\color{incolor}42}]:} \PY{n}{animals} \PY{o}{=} \PY{l+s+s1}{\PYZsq{}}\PY{l+s+s1}{eels}\PY{l+s+s1}{\PYZsq{}}
         \PY{n+nb}{print}\PY{p}{(}\PY{n}{f}\PY{l+s+s1}{\PYZsq{}}\PY{l+s+s1}{My hovercraft is full of }\PY{l+s+si}{\PYZob{}animals\PYZcb{}}\PY{l+s+s1}{.}\PY{l+s+s1}{\PYZsq{}}\PY{p}{)}
         \PY{n+nb}{print}\PY{p}{(}\PY{n}{f}\PY{l+s+s1}{\PYZsq{}}\PY{l+s+s1}{My hovercraft is full of }\PY{l+s+si}{\PYZob{}animals!r\PYZcb{}}\PY{l+s+s1}{.}\PY{l+s+s1}{\PYZsq{}}\PY{p}{)}
\end{Verbatim}


    \begin{Verbatim}[commandchars=\\\{\}]
My hovercraft is full of eels.
My hovercraft is full of 'eels'.

    \end{Verbatim}

    For a reference on these format specifications, see the reference guide
for the
\href{https://docs.python.org/3/tutorial/inputoutput.html}{Format
Specifiction Mini-Language}

    \paragraph{More on The String format()
mathod}\label{more-on-the-string-format-mathod}

    basic use of the str.format() method looks like this:

    \begin{Verbatim}[commandchars=\\\{\}]
{\color{incolor}In [{\color{incolor}43}]:} \PY{n+nb}{print}\PY{p}{(}\PY{l+s+s1}{\PYZsq{}}\PY{l+s+s1}{We are the }\PY{l+s+si}{\PYZob{}\PYZcb{}}\PY{l+s+s1}{ who say }\PY{l+s+s1}{\PYZdq{}}\PY{l+s+si}{\PYZob{}\PYZcb{}}\PY{l+s+s1}{!}\PY{l+s+s1}{\PYZdq{}}\PY{l+s+s1}{\PYZsq{}}\PY{o}{.}\PY{n}{format}\PY{p}{(}\PY{l+s+s1}{\PYZsq{}}\PY{l+s+s1}{knights}\PY{l+s+s1}{\PYZsq{}}\PY{p}{,} \PY{l+s+s1}{\PYZsq{}}\PY{l+s+s1}{Ni}\PY{l+s+s1}{\PYZsq{}}\PY{p}{)}\PY{p}{)}
\end{Verbatim}


    \begin{Verbatim}[commandchars=\\\{\}]
We are the knights who say "Ni!"

    \end{Verbatim}

    A number in the brackets can be used to refer to the position of the
object passed into the str.format() method.

    \begin{Verbatim}[commandchars=\\\{\}]
{\color{incolor}In [{\color{incolor}44}]:} \PY{n+nb}{print}\PY{p}{(}\PY{l+s+s1}{\PYZsq{}}\PY{l+s+si}{\PYZob{}0\PYZcb{}}\PY{l+s+s1}{ and }\PY{l+s+si}{\PYZob{}1\PYZcb{}}\PY{l+s+s1}{\PYZsq{}}\PY{o}{.}\PY{n}{format}\PY{p}{(}\PY{l+s+s1}{\PYZsq{}}\PY{l+s+s1}{spam}\PY{l+s+s1}{\PYZsq{}}\PY{p}{,} \PY{l+s+s1}{\PYZsq{}}\PY{l+s+s1}{eggs}\PY{l+s+s1}{\PYZsq{}}\PY{p}{)}\PY{p}{)}
         \PY{n+nb}{print}\PY{p}{(}\PY{l+s+s1}{\PYZsq{}}\PY{l+s+si}{\PYZob{}1\PYZcb{}}\PY{l+s+s1}{ and }\PY{l+s+si}{\PYZob{}0\PYZcb{}}\PY{l+s+s1}{\PYZsq{}}\PY{o}{.}\PY{n}{format}\PY{p}{(}\PY{l+s+s1}{\PYZsq{}}\PY{l+s+s1}{spam}\PY{l+s+s1}{\PYZsq{}}\PY{p}{,} \PY{l+s+s1}{\PYZsq{}}\PY{l+s+s1}{eggs}\PY{l+s+s1}{\PYZsq{}}\PY{p}{)}\PY{p}{)}
\end{Verbatim}


    \begin{Verbatim}[commandchars=\\\{\}]
spam and eggs
eggs and spam

    \end{Verbatim}

    If keyword arguments are used in the str.format() mathod, their values
are refered to by using the name of the argument.

    \begin{Verbatim}[commandchars=\\\{\}]
{\color{incolor}In [{\color{incolor}45}]:} \PY{n+nb}{print}\PY{p}{(}\PY{l+s+s1}{\PYZsq{}}\PY{l+s+s1}{This }\PY{l+s+si}{\PYZob{}food\PYZcb{}}\PY{l+s+s1}{ is }\PY{l+s+si}{\PYZob{}adjective\PYZcb{}}\PY{l+s+s1}{.}\PY{l+s+s1}{\PYZsq{}}\PY{o}{.}\PY{n}{format}\PY{p}{(}\PY{n}{food}\PY{o}{=}\PY{l+s+s1}{\PYZsq{}}\PY{l+s+s1}{spam}\PY{l+s+s1}{\PYZsq{}}\PY{p}{,} \PY{n}{adjective}\PY{o}{=}\PY{l+s+s1}{\PYZsq{}}\PY{l+s+s1}{absolutely horrible}\PY{l+s+s1}{\PYZsq{}}\PY{p}{)}\PY{p}{)}
\end{Verbatim}


    \begin{Verbatim}[commandchars=\\\{\}]
This spam is absolutely horrible.

    \end{Verbatim}

    Positional and keyword argument can be arbitrarily combined:

BUT, keyword argument have to be referenced by the key!!! Blank does not
work.

Positional arguement have to be all referenced by the index, OR all be
blank!!!

    \begin{Verbatim}[commandchars=\\\{\}]
{\color{incolor}In [{\color{incolor}50}]:} \PY{n+nb}{print}\PY{p}{(}\PY{l+s+s1}{\PYZsq{}}\PY{l+s+s1}{The story of }\PY{l+s+si}{\PYZob{}0\PYZcb{}}\PY{l+s+s1}{, }\PY{l+s+si}{\PYZob{}1\PYZcb{}}\PY{l+s+s1}{, and }\PY{l+s+si}{\PYZob{}other\PYZcb{}}\PY{l+s+s1}{.}\PY{l+s+s1}{\PYZsq{}}\PY{o}{.}\PY{n}{format}\PY{p}{(}\PY{l+s+s1}{\PYZsq{}}\PY{l+s+s1}{Bill}\PY{l+s+s1}{\PYZsq{}}\PY{p}{,} \PY{l+s+s1}{\PYZsq{}}\PY{l+s+s1}{Manfred}\PY{l+s+s1}{\PYZsq{}}\PY{p}{,} \PY{n}{other}\PY{o}{=}\PY{l+s+s1}{\PYZsq{}}\PY{l+s+s1}{Georg}\PY{l+s+s1}{\PYZsq{}}\PY{p}{)}\PY{p}{)}
\end{Verbatim}


    \begin{Verbatim}[commandchars=\\\{\}]
The story of Bill, Manfred, and Georg.

    \end{Verbatim}

    \begin{Verbatim}[commandchars=\\\{\}]
{\color{incolor}In [{\color{incolor}51}]:} \PY{n+nb}{print}\PY{p}{(}\PY{l+s+s1}{\PYZsq{}}\PY{l+s+s1}{The story of }\PY{l+s+si}{\PYZob{}\PYZcb{}}\PY{l+s+s1}{, }\PY{l+s+si}{\PYZob{}\PYZcb{}}\PY{l+s+s1}{, and }\PY{l+s+si}{\PYZob{}other\PYZcb{}}\PY{l+s+s1}{.}\PY{l+s+s1}{\PYZsq{}}\PY{o}{.}\PY{n}{format}\PY{p}{(}\PY{l+s+s1}{\PYZsq{}}\PY{l+s+s1}{Bill}\PY{l+s+s1}{\PYZsq{}}\PY{p}{,} \PY{l+s+s1}{\PYZsq{}}\PY{l+s+s1}{Manfred}\PY{l+s+s1}{\PYZsq{}}\PY{p}{,} \PY{n}{other}\PY{o}{=}\PY{l+s+s1}{\PYZsq{}}\PY{l+s+s1}{Georg}\PY{l+s+s1}{\PYZsq{}}\PY{p}{)}\PY{p}{)}
\end{Verbatim}


    \begin{Verbatim}[commandchars=\\\{\}]
The story of Bill, Manfred, and Georg.

    \end{Verbatim}

    If you have a really long format string that you don't want to split up,
it would be nice if you could reference the variables to be formatted by
name instead of by position. This can be done by simply passing the dict
and using square brackets '{[}{]}' to access the keys

    \begin{Verbatim}[commandchars=\\\{\}]
{\color{incolor}In [{\color{incolor}54}]:} \PY{n}{table} \PY{o}{=} \PY{p}{\PYZob{}}\PY{l+s+s1}{\PYZsq{}}\PY{l+s+s1}{Sjoerd}\PY{l+s+s1}{\PYZsq{}}\PY{p}{:} \PY{l+m+mi}{4127}\PY{p}{,} \PY{l+s+s1}{\PYZsq{}}\PY{l+s+s1}{Jack}\PY{l+s+s1}{\PYZsq{}}\PY{p}{:} \PY{l+m+mi}{4098}\PY{p}{,} \PY{l+s+s1}{\PYZsq{}}\PY{l+s+s1}{Dcab}\PY{l+s+s1}{\PYZsq{}}\PY{p}{:} \PY{l+m+mi}{8637678}\PY{p}{\PYZcb{}}
         \PY{n+nb}{print}\PY{p}{(}\PY{l+s+s1}{\PYZsq{}}\PY{l+s+s1}{Jack: }\PY{l+s+si}{\PYZob{}0[Jack]:d\PYZcb{}}\PY{l+s+s1}{; Sjoerd: }\PY{l+s+si}{\PYZob{}0[Sjoerd]:d\PYZcb{}}\PY{l+s+s1}{; Dcab: }\PY{l+s+si}{\PYZob{}0[Dcab]:d\PYZcb{}}\PY{l+s+s1}{\PYZsq{}}\PY{o}{.}\PY{n}{format}\PY{p}{(}\PY{n}{table}\PY{p}{)}\PY{p}{)}
\end{Verbatim}


    \begin{Verbatim}[commandchars=\\\{\}]
Jack: 4098; Sjoerd: 4127; Dcab: 8637678

    \end{Verbatim}

    This could also be done by passing the table as keyword arguments with
the `**' notation.

    \begin{Verbatim}[commandchars=\\\{\}]
{\color{incolor}In [{\color{incolor}55}]:} \PY{n}{table} \PY{o}{=} \PY{p}{\PYZob{}}\PY{l+s+s1}{\PYZsq{}}\PY{l+s+s1}{Sjoerd}\PY{l+s+s1}{\PYZsq{}}\PY{p}{:} \PY{l+m+mi}{4127}\PY{p}{,} \PY{l+s+s1}{\PYZsq{}}\PY{l+s+s1}{Jack}\PY{l+s+s1}{\PYZsq{}}\PY{p}{:} \PY{l+m+mi}{4098}\PY{p}{,} \PY{l+s+s1}{\PYZsq{}}\PY{l+s+s1}{Dcab}\PY{l+s+s1}{\PYZsq{}}\PY{p}{:} \PY{l+m+mi}{8637678}\PY{p}{\PYZcb{}}
         \PY{n+nb}{print}\PY{p}{(}\PY{l+s+s1}{\PYZsq{}}\PY{l+s+s1}{Jack: }\PY{l+s+si}{\PYZob{}Jack:d\PYZcb{}}\PY{l+s+s1}{; Sjoerd: }\PY{l+s+si}{\PYZob{}Sjoerd:d\PYZcb{}}\PY{l+s+s1}{; Dcab: }\PY{l+s+si}{\PYZob{}Dcab:d\PYZcb{}}\PY{l+s+s1}{\PYZsq{}}\PY{o}{.}\PY{n}{format}\PY{p}{(}\PY{o}{*}\PY{o}{*}\PY{n}{table}\PY{p}{)}\PY{p}{)}
\end{Verbatim}


    \begin{Verbatim}[commandchars=\\\{\}]
Jack: 4098; Sjoerd: 4127; Dcab: 8637678

    \end{Verbatim}

    This is particularly useful in combination with the built-in function
vars(), which returns a dictionary containing all local variables.

As an example, the following lines produce a tidily-aligned set of
columns giving integers and their squares and cubes:

    \begin{Verbatim}[commandchars=\\\{\}]
{\color{incolor}In [{\color{incolor}73}]:} \PY{k}{for} \PY{n}{x} \PY{o+ow}{in} \PY{n+nb}{range}\PY{p}{(}\PY{l+m+mi}{1}\PY{p}{,} \PY{l+m+mi}{11}\PY{p}{)}\PY{p}{:}
             \PY{n+nb}{print}\PY{p}{(}\PY{l+s+s1}{\PYZsq{}}\PY{l+s+si}{\PYZob{}0:2d\PYZcb{}}\PY{l+s+s1}{ }\PY{l+s+si}{\PYZob{}1:3d\PYZcb{}}\PY{l+s+s1}{ }\PY{l+s+si}{\PYZob{}2:4d\PYZcb{}}\PY{l+s+s1}{\PYZsq{}}\PY{o}{.}\PY{n}{format}\PY{p}{(}\PY{n}{x}\PY{p}{,} \PY{n}{x}\PY{o}{*}\PY{n}{x}\PY{p}{,} \PY{n}{x}\PY{o}{*}\PY{n}{x}\PY{o}{*}\PY{n}{x}\PY{p}{)}\PY{p}{)}    \PY{c+c1}{\PYZsh{} 0,1,2 是position,}
\end{Verbatim}


    \begin{Verbatim}[commandchars=\\\{\}]
 1   1    1
 2   4    8
 3   9   27
 4  16   64
 5  25  125
 6  36  216
 7  49  343
 8  64  512
 9  81  729
10 100 1000

    \end{Verbatim}

    \paragraph{Manual String formatting}\label{manual-string-formatting}

    Here's the same table of squares and cubes, formatted manually:

    \begin{Verbatim}[commandchars=\\\{\}]
{\color{incolor}In [{\color{incolor}74}]:} \PY{k}{for} \PY{n}{x} \PY{o+ow}{in} \PY{n+nb}{range} \PY{p}{(}\PY{l+m+mi}{1}\PY{p}{,} \PY{l+m+mi}{11}\PY{p}{)}\PY{p}{:}
             \PY{n+nb}{print}\PY{p}{(}\PY{n+nb}{repr}\PY{p}{(}\PY{n}{x}\PY{p}{)}\PY{o}{.}\PY{n}{rjust}\PY{p}{(}\PY{l+m+mi}{2}\PY{p}{)}\PY{p}{,} \PY{n+nb}{repr}\PY{p}{(}\PY{n}{x}\PY{o}{*}\PY{n}{x}\PY{p}{)}\PY{o}{.}\PY{n}{rjust}\PY{p}{(}\PY{l+m+mi}{3}\PY{p}{)}\PY{p}{,} \PY{n}{end}\PY{o}{=}\PY{l+s+s1}{\PYZsq{}}\PY{l+s+s1}{ }\PY{l+s+s1}{\PYZsq{}}\PY{p}{)}
             \PY{n+nb}{print}\PY{p}{(}\PY{n+nb}{repr}\PY{p}{(}\PY{n}{x}\PY{o}{*}\PY{o}{*}\PY{l+m+mi}{3}\PY{p}{)}\PY{o}{.}\PY{n}{rjust}\PY{p}{(}\PY{l+m+mi}{4}\PY{p}{)}\PY{p}{)}
\end{Verbatim}


    \begin{Verbatim}[commandchars=\\\{\}]
 1   1    1
 2   4    8
 3   9   27
 4  16   64
 5  25  125
 6  36  216
 7  49  343
 8  64  512
 9  81  729
10 100 1000

    \end{Verbatim}

    (Note that the one space between each column was added by the way
print() works: it always adds spaces between its arguments.)

    The str.rjust() method of string objects right-justifies a string in a
field of a given width by padding it with spaces on the left. There are
similar methods str.ljust() and str.center(). These methods do not write
anything, they just return a new string. If the input string is too
long, they don't truncate it, but return it unchanged; this will mess up
your column lay-out but that's usually better than the alternative,
which would be lying about a value. (If you really want truncation you
can always add a slice operation, as in x.ljust(n){[}:n{]}.)

    There is another method, str.zfill(), which pads a numeric string on the
left with zeros. It understands about plus and minus signs:

    \begin{Verbatim}[commandchars=\\\{\}]
{\color{incolor}In [{\color{incolor}75}]:} \PY{l+s+s1}{\PYZsq{}}\PY{l+s+s1}{12}\PY{l+s+s1}{\PYZsq{}}\PY{o}{.}\PY{n}{zfill}\PY{p}{(}\PY{l+m+mi}{5}\PY{p}{)}
\end{Verbatim}


\begin{Verbatim}[commandchars=\\\{\}]
{\color{outcolor}Out[{\color{outcolor}75}]:} '00012'
\end{Verbatim}
            
    \begin{Verbatim}[commandchars=\\\{\}]
{\color{incolor}In [{\color{incolor}76}]:} \PY{l+s+s1}{\PYZsq{}}\PY{l+s+s1}{\PYZhy{}3.14}\PY{l+s+s1}{\PYZsq{}}\PY{o}{.}\PY{n}{zfill}\PY{p}{(}\PY{l+m+mi}{7}\PY{p}{)}
\end{Verbatim}


\begin{Verbatim}[commandchars=\\\{\}]
{\color{outcolor}Out[{\color{outcolor}76}]:} '-003.14'
\end{Verbatim}
            
    \begin{Verbatim}[commandchars=\\\{\}]
{\color{incolor}In [{\color{incolor}77}]:} \PY{l+s+s1}{\PYZsq{}}\PY{l+s+s1}{3.14159265359}\PY{l+s+s1}{\PYZsq{}}\PY{o}{.}\PY{n}{zfill}\PY{p}{(}\PY{l+m+mi}{5}\PY{p}{)}
\end{Verbatim}


\begin{Verbatim}[commandchars=\\\{\}]
{\color{outcolor}Out[{\color{outcolor}77}]:} '3.14159265359'
\end{Verbatim}
            
    \paragraph{Old string formatting:}\label{old-string-formatting}

    \begin{Verbatim}[commandchars=\\\{\}]
{\color{incolor}In [{\color{incolor}78}]:} \PY{k+kn}{import} \PY{n+nn}{math}
         \PY{n+nb}{print}\PY{p}{(}\PY{l+s+s1}{\PYZsq{}}\PY{l+s+s1}{The value of pi is approximately }\PY{l+s+si}{\PYZpc{}5.3f}\PY{l+s+s1}{.}\PY{l+s+s1}{\PYZsq{}} \PY{o}{\PYZpc{}} \PY{n}{math}\PY{o}{.}\PY{n}{pi}\PY{p}{)}
\end{Verbatim}


    \begin{Verbatim}[commandchars=\\\{\}]
The value of pi is approximately 3.142.

    \end{Verbatim}

    \subsubsection{Reading and Writing
Files}\label{reading-and-writing-files}

    open() returns a file object, and is most commonly used with two
arguments: open(file, mode):

    \begin{Verbatim}[commandchars=\\\{\}]
{\color{incolor}In [{\color{incolor}79}]:} \PY{n}{f} \PY{o}{=} \PY{n+nb}{open}\PY{p}{(}\PY{l+s+s1}{\PYZsq{}}\PY{l+s+s1}{filename}\PY{l+s+s1}{\PYZsq{}}\PY{p}{,} \PY{l+s+s1}{\PYZsq{}}\PY{l+s+s1}{w}\PY{l+s+s1}{\PYZsq{}}\PY{p}{)}
\end{Verbatim}


    mode: * r: read only, default mode when mode is omitted. * w: only
write, an EXISTING file with the same name will be erased * a: open the
file for appending, any added data written to the file is automatically
added to the end. * r+: opens the file for both reading and writing

    'b' appended to the mode opens the file in \textbf{binary mode}: this
mode should be used for all files that don't contain text.

In the text mode, the default when \textbf{reading} is to convert
platform-specific line endings (\n on Unix, \r\n on Windows) to just \n.
When \textbf{writing} in text mode, the default is to convert
occurrences of \n back to platform specific line endings. This
background modification to file data is fine for text files, but will
corrupt binary data like that in \emph{JPEG} or \emph{EXE} files.

It is good practice to use the \textbf{with} keyword when dealing with
file objects. The advantage is that the file is properly closed after
its suite finishes, even if an exception is raised at some point. Using
\textbf{with} is also much shorter than writing equivalent
\textbf{try-finally} blocks:

    \begin{Verbatim}[commandchars=\\\{\}]
{\color{incolor}In [{\color{incolor}1}]:} \PY{k}{with} \PY{n+nb}{open}\PY{p}{(}\PY{l+s+s1}{\PYZsq{}}\PY{l+s+s1}{new\PYZus{}file.txt}\PY{l+s+s1}{\PYZsq{}}\PY{p}{)} \PY{k}{as} \PY{n}{f}\PY{p}{:}
            \PY{n}{read\PYZus{}data} \PY{o}{=} \PY{n}{f}\PY{o}{.}\PY{n}{read}\PY{p}{(}\PY{p}{)}
        \PY{n}{f}\PY{o}{.}\PY{n}{closed}    \PY{c+c1}{\PYZsh{} 它的输出是bool,很奇怪\PYZti{}}
\end{Verbatim}


\begin{Verbatim}[commandchars=\\\{\}]
{\color{outcolor}Out[{\color{outcolor}1}]:} True
\end{Verbatim}
            
    If not using \textbf{with} keyword, then you should call f.close() to
close the file.

After the file is closed, attempts to use the file object will
automatically fail.

    \paragraph{Methods on File Objects}\label{methods-on-file-objects}

    The following examples will assume that a file object called \emph{f}
has already been created.

To read a file's contents, call f.read({[}byte\_size{]}), which reads
some quantity of data and returns it as a string (string mode) or bytes
object (byte mode). byte\_size is an optional numeric argument. When
byte\_size is omitted or negative, the entire contens of the file will
be read and returned.

If the end of the file has been reached, f.read() will return an empty
string ('').

    \textbf{f.readline()} reads a single line from the file; a newline
character (\n) is left at the end of the string, and is only omitted on
the last line of the file if the file doesn't end in a newline.
(换行符号,\n 只有在整个文档的最后一行才会没有)

\begin{itemize}
\tightlist
\item
  .readline() returns an empty string, the end of the file been reached
\item
  .readline() returns '\n', that is a blank line.
\end{itemize}

    For reading lines from a file, you can loop over the file object. This
is memory efficient, fast and leads to simple code:

    \begin{Verbatim}[commandchars=\\\{\}]
{\color{incolor}In [{\color{incolor}23}]:} \PY{n}{f} \PY{o}{=} \PY{n+nb}{open}\PY{p}{(}\PY{l+s+s1}{\PYZsq{}}\PY{l+s+s1}{new\PYZus{}file.txt}\PY{l+s+s1}{\PYZsq{}}\PY{p}{)}
         \PY{k}{for} \PY{n}{line} \PY{o+ow}{in} \PY{n}{f}\PY{p}{:}
             \PY{n+nb}{print}\PY{p}{(}\PY{n+nb}{repr}\PY{p}{(}\PY{n}{line}\PY{p}{)}\PY{p}{)}    \PY{c+c1}{\PYZsh{} 这个最后一行也是以\PYZbs{}n结尾的呀。上边说的也不准呀}
         \PY{n}{f}\PY{o}{.}\PY{n}{close}\PY{p}{(}\PY{p}{)}
\end{Verbatim}


    \begin{Verbatim}[commandchars=\\\{\}]
'This is a test.\textbackslash{}n'
'ile line one.\textbackslash{}n'
'This is line two.\textbackslash{}n'
'This is line three, and line four will be blank.\textbackslash{}n'
'\textbackslash{}n'
'This is line five and the final line.\textbackslash{}n'

    \end{Verbatim}

    \begin{Verbatim}[commandchars=\\\{\}]
{\color{incolor}In [{\color{incolor}15}]:} \PY{n}{f} \PY{o}{=} \PY{n+nb}{open}\PY{p}{(}\PY{l+s+s1}{\PYZsq{}}\PY{l+s+s1}{new\PYZus{}file.txt}\PY{l+s+s1}{\PYZsq{}}\PY{p}{)}
         \PY{k}{for} \PY{n}{i} \PY{o+ow}{in} \PY{n+nb}{range}\PY{p}{(}\PY{l+m+mi}{6}\PY{p}{)}\PY{p}{:}
             \PY{n+nb}{print}\PY{p}{(}\PY{n+nb}{repr}\PY{p}{(}\PY{n}{f}\PY{o}{.}\PY{n}{readline}\PY{p}{(}\PY{p}{)}\PY{p}{)}\PY{p}{)}
         \PY{n}{f}\PY{o}{.}\PY{n}{close}\PY{p}{(}\PY{p}{)}
\end{Verbatim}


    \begin{Verbatim}[commandchars=\\\{\}]
'This is a test file line one.\textbackslash{}n'
'This is line two.\textbackslash{}n'
'This is line three, and line four will be blank.\textbackslash{}n'
'\textbackslash{}n'
'This is line five and the final line.\textbackslash{}n'
''

    \end{Verbatim}

    If want to read all the lines of a file in a list, can also use
\textbf{list(f)} or \textbf{f.readlines()}.

    \textbf{f.write(string)} writes the contents of string to the file,
returning the number of characters written.

.write funciton 从光标所在位置写入,open()默认打开文件开始的位置。

    \begin{Verbatim}[commandchars=\\\{\}]
{\color{incolor}In [{\color{incolor}29}]:} \PY{n}{f} \PY{o}{=} \PY{n+nb}{open}\PY{p}{(}\PY{l+s+s1}{\PYZsq{}}\PY{l+s+s1}{new\PYZus{}file.txt}\PY{l+s+s1}{\PYZsq{}}\PY{p}{,} \PY{l+s+s1}{\PYZsq{}}\PY{l+s+s1}{r+}\PY{l+s+s1}{\PYZsq{}}\PY{p}{)}
         \PY{n}{f}\PY{o}{.}\PY{n}{write}\PY{p}{(}\PY{l+s+s1}{\PYZsq{}}\PY{l+s+s1}{This is a test.}\PY{l+s+se}{\PYZbs{}n}\PY{l+s+s1}{\PYZsq{}}\PY{p}{)}
         \PY{n}{f}\PY{o}{.}\PY{n}{close}\PY{p}{(}\PY{p}{)}
         \PY{n}{f} \PY{o}{=} \PY{n+nb}{open}\PY{p}{(}\PY{l+s+s1}{\PYZsq{}}\PY{l+s+s1}{new\PYZus{}file.txt}\PY{l+s+s1}{\PYZsq{}}\PY{p}{)}
         \PY{n+nb}{print}\PY{p}{(}\PY{n}{f}\PY{o}{.}\PY{n}{read}\PY{p}{(}\PY{p}{)}\PY{p}{)}
\end{Verbatim}


    \begin{Verbatim}[commandchars=\\\{\}]
This This is a test.
is line one.
This is line two.
This is line three, and line four will be blank.

This is line five and the final line.
This is a test.


    \end{Verbatim}

    Other types of objects need to be converted - either to a string or a
bytes object, before writing them.

    \begin{Verbatim}[commandchars=\\\{\}]
{\color{incolor}In [{\color{incolor} }]:} \PY{n}{value} \PY{o}{=} \PY{p}{(}\PY{l+s+s1}{\PYZsq{}}\PY{l+s+s1}{the answer}\PY{l+s+s1}{\PYZsq{}}\PY{p}{,} \PY{l+m+mi}{42}\PY{p}{)}
        \PY{n}{s} \PY{o}{=} \PY{n+nb}{str}\PY{p}{(}\PY{n}{value}\PY{p}{)}
        \PY{n}{f} \PY{o}{=} \PY{n+nb}{open}\PY{p}{(}\PY{l+s+s1}{\PYZsq{}}\PY{l+s+s1}{new\PYZus{}file.txt}\PY{l+s+s1}{\PYZsq{}}\PY{p}{,} \PY{l+s+s1}{\PYZsq{}}\PY{l+s+s1}{r+}\PY{l+s+s1}{\PYZsq{}}\PY{p}{)}
        \PY{n}{f}\PY{o}{.}\PY{n}{write}\PY{p}{(}\PY{n}{s}\PY{p}{)}
\end{Verbatim}


    f.tell() returns an integer giving the file object's current position in
the file represented as number of bytes from the beginning of the file
when in binary mode and an opaque number when in text mode.

To change the file object's position, use f.seek(offset, from\_what).
The position is computed from adding \textbf{offset} to a reference
point; the reference point is selected by the \textbf{from\_what}
argument:

\textbf{from\_what}: * 0: measures from the beginning of the file * 1:
measures from the current position * 2: the end of the file.

    \begin{Verbatim}[commandchars=\\\{\}]
{\color{incolor}In [{\color{incolor} }]:} \PY{n}{f} \PY{o}{=} \PY{n+nb}{open}\PY{p}{(}\PY{l+s+s1}{\PYZsq{}}\PY{l+s+s1}{workfile}\PY{l+s+s1}{\PYZsq{}}\PY{p}{,} \PY{l+s+s1}{\PYZsq{}}\PY{l+s+s1}{rb+}\PY{l+s+s1}{\PYZsq{}}\PY{p}{)}    \PY{c+c1}{\PYZsh{} 注意这里是binary mode}
        \PY{n}{f}\PY{o}{.}\PY{n}{write}\PY{p}{(}\PY{l+s+sa}{b}\PY{l+s+s1}{\PYZsq{}}\PY{l+s+s1}{0123456789abcdef}\PY{l+s+s1}{\PYZsq{}}\PY{p}{)}
        \PY{n}{f}\PY{o}{.}\PY{n}{seek}\PY{p}{(}\PY{l+m+mi}{5}\PY{p}{)}    \PY{c+c1}{\PYZsh{} Go to the 6th byte in the file}
        \PY{n}{f}\PY{o}{.}\PY{n}{seek}\PY{p}{(}\PY{l+m+mi}{0}\PY{p}{,} \PY{l+m+mi}{2}\PY{p}{)}    \PY{c+c1}{\PYZsh{} go to the end of the file}
        \PY{n}{f}\PY{o}{.}\PY{n}{seek}\PY{p}{(}\PY{o}{\PYZhy{}}\PY{l+m+mi}{3}\PY{p}{,} \PY{l+m+mi}{2}\PY{p}{)}   \PY{c+c1}{\PYZsh{} Go to the 3rd byte before the end}
\end{Verbatim}


    In \textbf{text files} (those opened without a \emph{b} in the mode
string), only seeks relative to the beginnning of the file are allowed
(the exception being seeking to the very file end with seek(0, 2)) and
the only valid offset values are those returned from the f.tell(), or
zero. Any other \textbf{offset} value produces undefined behaviour.

    \paragraph{Saving structured data with json (JavaScript Object
Notation)}\label{saving-structured-data-with-json-javascript-object-notation}

Strings can be easily written to and read from a file.

Numbers take a bit more effort, since the read() method only returns
strings, which will have to be passed to a function like int()。

Rather than having users constantly writing and debugging code to save
complicated data types to files, Python allows you to use the popular
data interchange format called JSON (JavaScript Object Notation). The
standard module called json can take Python data hierarchies, and
convert them to string representations; this process is called
\textbf{serializing}. Reconstructing the data from the string
representation is called deserializing. Between serializing and
deserializing, the string representing the object may have been stored
in a file or data, or sent over a network connection to some distant
machine.

The JSON format is commonly used by modern applications to allow for
\textbf{data exchange}.

    If you have an object \textbf{x}, to view its JSON string:

    \begin{Verbatim}[commandchars=\\\{\}]
{\color{incolor}In [{\color{incolor}30}]:} \PY{k+kn}{import} \PY{n+nn}{json}
         \PY{n}{json}\PY{o}{.}\PY{n}{dumps}\PY{p}{(}\PY{p}{[}\PY{l+m+mi}{1}\PY{p}{,} \PY{l+s+s1}{\PYZsq{}}\PY{l+s+s1}{simple}\PY{l+s+s1}{\PYZsq{}}\PY{p}{,} \PY{l+s+s1}{\PYZsq{}}\PY{l+s+s1}{list}\PY{l+s+s1}{\PYZsq{}}\PY{p}{]}\PY{p}{)}
\end{Verbatim}


\begin{Verbatim}[commandchars=\\\{\}]
{\color{outcolor}Out[{\color{outcolor}30}]:} '[1, "simple", "list"]'
\end{Verbatim}
            
    Another variant of the dumps() function, called dump(), simply
serializes the object to a text file. So we can write x to a text file
like this:

    \begin{Verbatim}[commandchars=\\\{\}]
{\color{incolor}In [{\color{incolor}31}]:} \PY{n}{json}\PY{o}{.}\PY{n}{dump}\PY{p}{(}\PY{n}{x}\PY{p}{,} \PY{n}{f}\PY{p}{)}
\end{Verbatim}


    \begin{Verbatim}[commandchars=\\\{\}]

        ---------------------------------------------------------------------------

        NameError                                 Traceback (most recent call last)

        <ipython-input-31-83ec06b0df75> in <module>()
    ----> 1 json.dump(x, f)
    

        NameError: name 'x' is not defined

    \end{Verbatim}

    To decode the object again, if f is a text file object which has been
opened for reading:

    \begin{Verbatim}[commandchars=\\\{\}]
{\color{incolor}In [{\color{incolor} }]:} \PY{n}{x} \PY{o}{=} \PY{n}{json}\PY{o}{.}\PY{n}{load}\PY{p}{(}\PY{n}{f}\PY{p}{)}
\end{Verbatim}


    The above simple serialization technique can handle lists and
dictionaries, but serializing arbitrary class instances in JSON requires
a bit of extra effort. the json module explains more.

    \textbf{pickle} - the pickle module

Contrary to JSON, pickle is a protocol which allows the serialization of
arbitrarily complex Python objects. It is specific to Python and cannot
be used to communicate with applications written in other languages. It
is also insecure by default.

    \subsection{Errors and Exceptions}\label{errors-and-exceptions}

    Two distinguishable kinds of errors: * syntax errors * exceptions

    \subsubsection{Syntax Errors}\label{syntax-errors}

Also known as parsing errors, the most common errors.

    \begin{Verbatim}[commandchars=\\\{\}]
{\color{incolor}In [{\color{incolor}32}]:} \PY{k}{while} \PY{k+kc}{True} \PY{n+nb}{print}\PY{p}{(}\PY{l+s+s1}{\PYZsq{}}\PY{l+s+s1}{Hello world}\PY{l+s+s1}{\PYZsq{}}\PY{p}{)}
\end{Verbatim}


    \begin{Verbatim}[commandchars=\\\{\}]

          File "<ipython-input-32-2b688bc740d7>", line 1
        while True print('Hello world')
                       \^{}
    SyntaxError: invalid syntax


    \end{Verbatim}

    The parser repeats the offending line and displays a little 'arrow'
pointing at the earliest point in the line where the error was detected.
The error is caused by (or at least detected at) the token
\textbf{preceding} the arrow:

in the above example, the error is detected at the function print(),
since a colon is missing before it. File name and line number are
printed so you know where to look in case the input came from a script.

    \paragraph{Exceptions}\label{exceptions}

Even if a statement or expression is syntactically correct, it may cause
an error when an attempt is made to execute it. Errors detected during
execution are called \textbf{exceptions} and are not unconditionally
fatal.

    \begin{Verbatim}[commandchars=\\\{\}]
{\color{incolor}In [{\color{incolor}33}]:} \PY{l+m+mi}{10} \PY{o}{*} \PY{p}{(}\PY{l+m+mi}{1}\PY{o}{/}\PY{l+m+mi}{0}\PY{p}{)}
\end{Verbatim}


    \begin{Verbatim}[commandchars=\\\{\}]

        ---------------------------------------------------------------------------

        ZeroDivisionError                         Traceback (most recent call last)

        <ipython-input-33-0b280f36835c> in <module>()
    ----> 1 10 * (1/0)
    

        ZeroDivisionError: division by zero

    \end{Verbatim}

    \begin{Verbatim}[commandchars=\\\{\}]
{\color{incolor}In [{\color{incolor}34}]:} \PY{l+m+mi}{4} \PY{o}{+} \PY{n}{spam}\PY{o}{*}\PY{l+m+mi}{3}
\end{Verbatim}


    \begin{Verbatim}[commandchars=\\\{\}]

        ---------------------------------------------------------------------------

        NameError                                 Traceback (most recent call last)

        <ipython-input-34-c98bb92cdcac> in <module>()
    ----> 1 4 + spam*3
    

        NameError: name 'spam' is not defined

    \end{Verbatim}

    \begin{Verbatim}[commandchars=\\\{\}]
{\color{incolor}In [{\color{incolor}35}]:} \PY{l+s+s1}{\PYZsq{}}\PY{l+s+s1}{2}\PY{l+s+s1}{\PYZsq{}} \PY{o}{+} \PY{l+m+mi}{2}
\end{Verbatim}


    \begin{Verbatim}[commandchars=\\\{\}]

        ---------------------------------------------------------------------------

        TypeError                                 Traceback (most recent call last)

        <ipython-input-35-d2b23a1db757> in <module>()
    ----> 1 '2' + 2
    

        TypeError: must be str, not int

    \end{Verbatim}

    The last line of the error message indicates what happened. Exceptions
come in different types, and the type is printed as part of the message:
* ZeroDivisionError * NameError * TypeError

Go to take a look at the Built-in Exceptions lists to see the exceptions
and their meaning.

    \subsubsection{Handling Exceptions}\label{handling-exceptions}

The following example: asks the user for input until a valid integer has
been entered, but allows the user to interrupt the program; note that a
user-generated interruption is signalled by raising the
\textbf{KeyboardInterrupt} exception.

    \begin{Verbatim}[commandchars=\\\{\}]
{\color{incolor}In [{\color{incolor}49}]:} \PY{k}{while} \PY{k+kc}{True}\PY{p}{:}
             \PY{k}{try}\PY{p}{:}
                 \PY{n}{x} \PY{o}{=} \PY{n+nb}{int}\PY{p}{(}\PY{n+nb}{input}\PY{p}{(}\PY{l+s+s2}{\PYZdq{}}\PY{l+s+s2}{Please enter a number: }\PY{l+s+s2}{\PYZdq{}}\PY{p}{)}\PY{p}{)}
                 \PY{n+nb}{print}\PY{p}{(}\PY{n}{f}\PY{l+s+s1}{\PYZsq{}}\PY{l+s+s1}{you entered }\PY{l+s+si}{\PYZob{}x\PYZcb{}}\PY{l+s+s1}{\PYZsq{}}\PY{p}{)}
                 \PY{k}{break}
             \PY{k}{except} \PY{n+ne}{ValueError}\PY{p}{:}
                 \PY{n+nb}{print}\PY{p}{(}\PY{l+s+s2}{\PYZdq{}}\PY{l+s+s2}{Oops! That was no valid number. Try again...}\PY{l+s+s2}{\PYZdq{}}\PY{p}{)}
\end{Verbatim}


    \begin{Verbatim}[commandchars=\\\{\}]
Please enter a number: a
Oops! That was no valid number. Try again{\ldots}
Please enter a number: 12
you entered 12

    \end{Verbatim}

    The try statement works as follows.

\begin{itemize}
\tightlist
\item
  First, the \textbf{try} clause (the statements between the try and
  except keywords) is executed.
\item
  If no exception occurs, the \textbf{except clause} is skipped and
  execution of the try statment is finished.
\item
  If an exception occurs during execution of the \textbf{try} clause,
  the rest of the clause is skipped. Then if its type matches the
  exception named after the \textbf{except} keyword, the except clause
  is executed, and then execution continues after the \textbf{try}
  statement.
\item
  If an exception occurs which does not match the exception named in the
  except clause, it is passed on to outer \textbf{try} statements; if no
  handler is found, it is an unhandled exception and execution stops
  with a message as shown above.
\end{itemize}

    A \textbf{try} statement can have more than one except clause, to
specify handlers for different exceptions. At most one handler will be
executed. An except clause may name multiple exceptions as a
parenthesized tuple, for example:

    \begin{Verbatim}[commandchars=\\\{\}]
{\color{incolor}In [{\color{incolor}39}]:} \PY{k}{try}\PY{p}{:}
             \PY{k}{pass}
         \PY{k}{except} \PY{p}{(}\PY{n+ne}{RuntimeError}\PY{p}{,} \PY{n+ne}{TypeError}\PY{p}{,} \PY{n+ne}{NameError}\PY{p}{)}\PY{p}{:}
             \PY{k}{pass}
\end{Verbatim}


    A class in an \textbf{except} clause is compatible with an exception if
it is the same class or a base class thereof (but not the other way
around - an except clause listing a derived class is not compatible with
a base class).

多个except
只有第一个满足条件的会被执行,如下两个例子,D是C的子类,C是B的子类。对比两个离子的结果即可分清。

    \begin{Verbatim}[commandchars=\\\{\}]
{\color{incolor}In [{\color{incolor}41}]:} \PY{k}{class} \PY{n+nc}{B}\PY{p}{(}\PY{n+ne}{Exception}\PY{p}{)}\PY{p}{:}
             \PY{k}{pass}
         
         \PY{k}{class} \PY{n+nc}{C}\PY{p}{(}\PY{n}{B}\PY{p}{)}\PY{p}{:}
             \PY{k}{pass}
         
         \PY{k}{class} \PY{n+nc}{D}\PY{p}{(}\PY{n}{C}\PY{p}{)}\PY{p}{:}
             \PY{k}{pass}
         
         \PY{k}{for} \PY{n+nb+bp}{cls} \PY{o+ow}{in} \PY{p}{[}\PY{n}{B}\PY{p}{,} \PY{n}{C}\PY{p}{,} \PY{n}{D}\PY{p}{]}\PY{p}{:}
             \PY{k}{try}\PY{p}{:}
                 \PY{k}{raise} \PY{n+nb+bp}{cls}\PY{p}{(}\PY{p}{)}    \PY{c+c1}{\PYZsh{} 这个raise的功能暂时不懂,下边会讲。}
             \PY{k}{except} \PY{n}{D}\PY{p}{:}
                 \PY{n+nb}{print}\PY{p}{(}\PY{l+s+s2}{\PYZdq{}}\PY{l+s+s2}{D}\PY{l+s+s2}{\PYZdq{}}\PY{p}{)}
             \PY{k}{except} \PY{n}{C}\PY{p}{:}
                 \PY{n+nb}{print}\PY{p}{(}\PY{l+s+s2}{\PYZdq{}}\PY{l+s+s2}{C}\PY{l+s+s2}{\PYZdq{}}\PY{p}{)}
             \PY{k}{except} \PY{n}{B}\PY{p}{:}
                 \PY{n+nb}{print}\PY{p}{(}\PY{l+s+s2}{\PYZdq{}}\PY{l+s+s2}{B}\PY{l+s+s2}{\PYZdq{}}\PY{p}{)}
\end{Verbatim}


    \begin{Verbatim}[commandchars=\\\{\}]
B
C
D

    \end{Verbatim}

    \begin{Verbatim}[commandchars=\\\{\}]
{\color{incolor}In [{\color{incolor}42}]:} \PY{k}{class} \PY{n+nc}{B}\PY{p}{(}\PY{n+ne}{Exception}\PY{p}{)}\PY{p}{:}
             \PY{k}{pass}
         
         \PY{k}{class} \PY{n+nc}{C}\PY{p}{(}\PY{n}{B}\PY{p}{)}\PY{p}{:}
             \PY{k}{pass}
         
         \PY{k}{class} \PY{n+nc}{D}\PY{p}{(}\PY{n}{C}\PY{p}{)}\PY{p}{:}
             \PY{k}{pass}
         
         \PY{k}{for} \PY{n+nb+bp}{cls} \PY{o+ow}{in} \PY{p}{[}\PY{n}{D}\PY{p}{,} \PY{n}{C}\PY{p}{,} \PY{n}{B}\PY{p}{]}\PY{p}{:}
             \PY{k}{try}\PY{p}{:}
                 \PY{k}{raise} \PY{n+nb+bp}{cls}\PY{p}{(}\PY{p}{)}
             \PY{k}{except} \PY{n}{D}\PY{p}{:}
                 \PY{n+nb}{print}\PY{p}{(}\PY{l+s+s2}{\PYZdq{}}\PY{l+s+s2}{D}\PY{l+s+s2}{\PYZdq{}}\PY{p}{)}
             \PY{k}{except} \PY{n}{C}\PY{p}{:}
                 \PY{n+nb}{print}\PY{p}{(}\PY{l+s+s2}{\PYZdq{}}\PY{l+s+s2}{C}\PY{l+s+s2}{\PYZdq{}}\PY{p}{)}
             \PY{k}{except} \PY{n}{B}\PY{p}{:}
                 \PY{n+nb}{print}\PY{p}{(}\PY{l+s+s2}{\PYZdq{}}\PY{l+s+s2}{B}\PY{l+s+s2}{\PYZdq{}}\PY{p}{)}
\end{Verbatim}


    \begin{Verbatim}[commandchars=\\\{\}]
D
C
B

    \end{Verbatim}

    The last except clause may omit the exception name(s), to serve as a
wildcard. Use this with extreme caution, since it is easy to mask a real
programming error in this way! It can also be used to print an error
message and then re-raise the exception (allowing a caller to handle the
exception as well):

    \begin{Verbatim}[commandchars=\\\{\}]
{\color{incolor}In [{\color{incolor}48}]:} \PY{k+kn}{import} \PY{n+nn}{sys}
         
         \PY{k}{try}\PY{p}{:}
             \PY{n}{f} \PY{o}{=} \PY{n+nb}{open}\PY{p}{(}\PY{l+s+s1}{\PYZsq{}}\PY{l+s+s1}{new\PYZus{}file.txt}\PY{l+s+s1}{\PYZsq{}}\PY{p}{)}
             \PY{n}{s} \PY{o}{=} \PY{n}{f}\PY{o}{.}\PY{n}{readline}\PY{p}{(}\PY{p}{)}
             \PY{n}{i} \PY{o}{=} \PY{n+nb}{int}\PY{p}{(}\PY{n}{s}\PY{o}{.}\PY{n}{strip}\PY{p}{(}\PY{p}{)}\PY{p}{)}
         \PY{k}{except} \PY{n+ne}{OSError} \PY{k}{as} \PY{n}{err}\PY{p}{:}
             \PY{n+nb}{print}\PY{p}{(}\PY{l+s+s2}{\PYZdq{}}\PY{l+s+s2}{OS error: }\PY{l+s+si}{\PYZob{}0\PYZcb{}}\PY{l+s+s2}{\PYZdq{}}\PY{o}{.}\PY{n}{format}\PY{p}{(}\PY{n}{err}\PY{p}{)}\PY{p}{)}
         \PY{k}{except} \PY{n+ne}{ValueError}\PY{p}{:}
             \PY{n+nb}{print}\PY{p}{(}\PY{l+s+s2}{\PYZdq{}}\PY{l+s+s2}{Could not convert data to an integer.}\PY{l+s+s2}{\PYZdq{}}\PY{p}{)}
         \PY{k}{except}\PY{p}{:}    \PY{c+c1}{\PYZsh{} raise暂时不懂 下边还会讲}
             \PY{n+nb}{print}\PY{p}{(}\PY{l+s+s2}{\PYZdq{}}\PY{l+s+s2}{Unexcepted error:}\PY{l+s+s2}{\PYZdq{}}\PY{p}{,} \PY{n}{sys}\PY{o}{.}\PY{n}{exc\PYZus{}info}\PY{p}{(}\PY{p}{)}\PY{p}{[}\PY{l+m+mi}{0}\PY{p}{]}\PY{p}{)}    \PY{c+c1}{\PYZsh{}这里不懂}
             \PY{k}{raise}
\end{Verbatim}


    \begin{Verbatim}[commandchars=\\\{\}]
Could not convert data to an integer.

    \end{Verbatim}

    The try ... except statement has an optional else clause, which, when
present, must follow all except clauses. It is useful for code that must
be executed if the try clause \textbf{does not} raise an exception:

    \begin{Verbatim}[commandchars=\\\{\}]
{\color{incolor}In [{\color{incolor}46}]:} \PY{k}{for} \PY{n}{arg} \PY{o+ow}{in} \PY{n}{sys}\PY{o}{.}\PY{n}{argv}\PY{p}{[}\PY{l+m+mi}{1}\PY{p}{:}\PY{p}{]}\PY{p}{:}
             \PY{k}{try}\PY{p}{:}
                 \PY{n}{f} \PY{o}{=} \PY{n+nb}{open}\PY{p}{(}\PY{n}{arg}\PY{p}{,} \PY{l+s+s1}{\PYZsq{}}\PY{l+s+s1}{r}\PY{l+s+s1}{\PYZsq{}}\PY{p}{)}
             \PY{k}{except} \PY{n+ne}{OSError}\PY{p}{:}
                 \PY{n+nb}{print}\PY{p}{(}\PY{l+s+s1}{\PYZsq{}}\PY{l+s+s1}{cannot open}\PY{l+s+s1}{\PYZsq{}}\PY{p}{,} \PY{n}{arg}\PY{p}{)}
             \PY{k}{else}\PY{p}{:}
                 \PY{n+nb}{print}\PY{p}{(}\PY{n}{arg}\PY{p}{,} \PY{l+s+s1}{\PYZsq{}}\PY{l+s+s1}{has}\PY{l+s+s1}{\PYZsq{}}\PY{p}{,} \PY{n+nb}{len}\PY{p}{(}\PY{n}{f}\PY{o}{.}\PY{n}{readlines}\PY{p}{(}\PY{p}{)}\PY{p}{)}\PY{p}{,} \PY{l+s+s1}{\PYZsq{}}\PY{l+s+s1}{lines}\PY{l+s+s1}{\PYZsq{}}\PY{p}{)}
                 \PY{n}{f}\PY{o}{.}\PY{n}{close}\PY{p}{(}\PY{p}{)}
\end{Verbatim}


    \begin{Verbatim}[commandchars=\\\{\}]
cannot open -f
/Users/jianbinliu/Library/Jupyter/runtime/kernel-355154af-82cb-47be-9ddf-a3ea4edb8558.json has 12 lines

    \end{Verbatim}

    The use of the else clause is better than adding additional code to the
try clause because it avoids accidentally catching an exception that
wasn't raised by the code being protected by the try ... except
statement.

When an exception occurs, it may have an associated value, also known as
the exception's argument. The presence and type of the argument depend
on the exception type.

The except clause may specify a variable after the exception name. The
variable is bound to an exception instance with the arguements stored in
\textbf{instance.args}. For convenience, the exception instance defines
\textbf{\_\_str\_\_()} so the arguments can be printed directly without
having to reference .args.

这部分讲的东西有些还是不太懂,先就这样了。以后用得到时再仔细研究。

    \subsubsection{Raising Exceptions}\label{raising-exceptions}

The \textbf{raise} statement allows the programmer to force a specified
exception to occur:

    \begin{Verbatim}[commandchars=\\\{\}]
{\color{incolor}In [{\color{incolor}50}]:} \PY{k}{raise} \PY{n+ne}{NameError}\PY{p}{(}\PY{l+s+s1}{\PYZsq{}}\PY{l+s+s1}{HiThere}\PY{l+s+s1}{\PYZsq{}}\PY{p}{)}
\end{Verbatim}


    \begin{Verbatim}[commandchars=\\\{\}]

        ---------------------------------------------------------------------------

        NameError                                 Traceback (most recent call last)

        <ipython-input-50-72c183edb298> in <module>()
    ----> 1 raise NameError('HiThere')
    

        NameError: HiThere

    \end{Verbatim}

    The sole argument indicates the exception to be raised. This must be
either an exception instance or an exception class (a class that derives
from \textbf{Exception}). If an exception class is passed, it will be
implicitly instantiated by calling its constructor with no arguments:

    \begin{Verbatim}[commandchars=\\\{\}]
{\color{incolor}In [{\color{incolor}51}]:} \PY{k}{raise} \PY{n+ne}{ValueError}    \PY{c+c1}{\PYZsh{} shorthand for \PYZsq{}raise ValueError()\PYZsq{}}
\end{Verbatim}


    \begin{Verbatim}[commandchars=\\\{\}]

        ---------------------------------------------------------------------------

        ValueError                                Traceback (most recent call last)

        <ipython-input-51-45550f83e55d> in <module>()
    ----> 1 raise ValueError    \# shorthand for 'raise ValueError()'
    

        ValueError: 

    \end{Verbatim}

    If you need to determine whether an exception was raised but don't
intend to handle it, a simpler form of the raise statement allows you to
re-raise the exception:

    \begin{Verbatim}[commandchars=\\\{\}]
{\color{incolor}In [{\color{incolor}52}]:} \PY{k}{try}\PY{p}{:}
             \PY{k}{raise} \PY{n+ne}{NameError}\PY{p}{(}\PY{l+s+s1}{\PYZsq{}}\PY{l+s+s1}{HiThere}\PY{l+s+s1}{\PYZsq{}}\PY{p}{)}
         \PY{k}{except} \PY{n+ne}{NameError}\PY{p}{:}
             \PY{n+nb}{print}\PY{p}{(}\PY{l+s+s1}{\PYZsq{}}\PY{l+s+s1}{An exception flew by!}\PY{l+s+s1}{\PYZsq{}}\PY{p}{)}
             \PY{k}{raise}
\end{Verbatim}


    \begin{Verbatim}[commandchars=\\\{\}]
An exception flew by!

    \end{Verbatim}

    \begin{Verbatim}[commandchars=\\\{\}]

        ---------------------------------------------------------------------------

        NameError                                 Traceback (most recent call last)

        <ipython-input-52-bf6ef4926f8c> in <module>()
          1 try:
    ----> 2     raise NameError('HiThere')
          3 except NameError:
          4     print('An exception flew by!')
          5     raise


        NameError: HiThere

    \end{Verbatim}

    \subsubsection{User-defined Exceptions}\label{user-defined-exceptions}

Exceptions should typically be derived from the \emph{Exception} class,
either directly or indirectly.

Exception classes can be defined which do anything any other class can
do, but are usually kept simple, often only offering a number of
attributes that allow information about the error to be extracted by
handlers for he exception.

Many standard modules define their own exceptions to report errors that
may occur in functions they define.
详细内容去看tutorial。用到的时候再来看。

    \subsubsection{Defining Clean-up
Actions}\label{defining-clean-up-actions}

The \textbf{try} statement has another optional clause which is intended
to define clean-up actions that must be executed under all
circumstances:

    \begin{Verbatim}[commandchars=\\\{\}]
{\color{incolor}In [{\color{incolor}53}]:} \PY{k}{try}\PY{p}{:}
             \PY{k}{raise} \PY{n}{keyboardInterrupt}
         \PY{k}{finally}\PY{p}{:}
             \PY{n+nb}{print}\PY{p}{(}\PY{l+s+s1}{\PYZsq{}}\PY{l+s+s1}{Goodbye, world!}\PY{l+s+s1}{\PYZsq{}}\PY{p}{)}
\end{Verbatim}


    \begin{Verbatim}[commandchars=\\\{\}]
Goodbye, world!

    \end{Verbatim}

    \begin{Verbatim}[commandchars=\\\{\}]

        ---------------------------------------------------------------------------

        NameError                                 Traceback (most recent call last)

        <ipython-input-53-49d64e6afd61> in <module>()
          1 try:
    ----> 2     raise keyboardInterrupt
          3 finally:
          4     print('Goodbye, world!')


        NameError: name 'keyboardInterrupt' is not defined

    \end{Verbatim}

    \textbf{finally} statement is \textbf{always} executed before leaving
the \textbf{try} statement, whether an exception has occurred or not.

When an exception has occurred in the try clause and has not been
handled by an except clause, it is re-raised afer the finally clause has
been executed.

The finally clause is also executed on the way out when any other clause
of the try statement is left via a break, continue or return statement.

    \subsubsection{Predifined clean-up
Actions}\label{predifined-clean-up-actions}

The following example has a problem that it leaves the file open for an
indeterminate amount of time after this part of the code has finished
executing. Could be an issue for larger applications.

    \begin{Verbatim}[commandchars=\\\{\}]
{\color{incolor}In [{\color{incolor}54}]:} \PY{k}{for} \PY{n}{line} \PY{o+ow}{in} \PY{n+nb}{open}\PY{p}{(}\PY{l+s+s1}{\PYZsq{}}\PY{l+s+s1}{new\PYZus{}file.txt}\PY{l+s+s1}{\PYZsq{}}\PY{p}{)}\PY{p}{:}
             \PY{n+nb}{print}\PY{p}{(}\PY{n}{line}\PY{p}{,} \PY{n}{end}\PY{o}{=}\PY{l+s+s1}{\PYZsq{}}\PY{l+s+s1}{\PYZsq{}}\PY{p}{)}
\end{Verbatim}


    \begin{Verbatim}[commandchars=\\\{\}]
This This is a test.
is line one.
This is line two.
This is line three, and line four will be blank.

This is line five and the final line.
This is a test.

    \end{Verbatim}

    The \textbf{with} statement allows objects like files to be used in a
way that ensures they are always cleaned up promptly and correctly.

    \begin{Verbatim}[commandchars=\\\{\}]
{\color{incolor}In [{\color{incolor}55}]:} \PY{k}{with} \PY{n+nb}{open}\PY{p}{(}\PY{l+s+s1}{\PYZsq{}}\PY{l+s+s1}{new\PYZus{}file.txt}\PY{l+s+s1}{\PYZsq{}}\PY{p}{)} \PY{k}{as} \PY{n}{f}\PY{p}{:}
             \PY{k}{for} \PY{n}{line} \PY{o+ow}{in} \PY{n}{f}\PY{p}{:}
                 \PY{n+nb}{print}\PY{p}{(}\PY{n}{line}\PY{p}{,} \PY{n}{end}\PY{o}{=}\PY{l+s+s1}{\PYZsq{}}\PY{l+s+s1}{\PYZsq{}}\PY{p}{)}
\end{Verbatim}


    \begin{Verbatim}[commandchars=\\\{\}]
This This is a test.
is line one.
This is line two.
This is line three, and line four will be blank.

This is line five and the final line.
This is a test.

    \end{Verbatim}

    After the statement is executed, the file \textbf{f} is always closed,
even if a problem was encountered while processing the lines.

    \subsection{Classes}\label{classes}

Creating a new class creates a new type of object, allowing new
instances of that type to be made.

Each class instance can have attributes for maintaining its state.Class
instances can also have mehtods for modifying its state.

Pythone classes provide all standard features of OOP: * allows multiple
base classes * a derived class can override any methods of its base
classes * a method can call the method of a base class with the same
name


    % Add a bibliography block to the postdoc
    
    
    
    \end{document}
